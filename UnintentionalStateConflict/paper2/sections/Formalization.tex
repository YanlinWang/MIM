\section{Overview}
\subsection{Motivation}
Combine the work of FHJ and ClasslessJava.
\begin{itemize}
	\item Use abstract state operations to mimic state
	\item If two fields conflict from triangle inheritance, keep the two fields
	\item Override and covariant type refine of fields
\end{itemize}

Example of unintentional fields confliction:
\begin{lstlisting}
interface Payment {
	String Check(); // represnets an entity 'check'
}
interface Verify {
	boolean Check(); // represents a flag of whether the verification is checked or not.
}
interface VerifiedPayment extends Payment, Verify {}
\end{lstlisting}

C++ partly support this, just like unintentional method conflicts. However, c++ cannot handle the case where fields need to be refined or two conflicted fields need to be merged.
\begin{lstlisting}
class A {};
class B:A {};
class Payment {
	public:
	A* check;
};
class Verify {
	public:
	bool check;
};
class VerifiedPayment : public Payment, public Verify {
	public:
	B* check;
};

int main()
{
VerifiedPayment* vp = new VerifiedPayment();
vp->check = new B();
vp->Verify::check = true;
vp->check;
cout << vp->Payment::check << endl;
cout << vp->check << endl;
}
\end{lstlisting}


Example program in a desired language:
\begin{lstlisting}
interface A { int x(); int y(); }
interface B { int x(); }
interface C extends A, B {}
new B(B.x = 0).GET_x();   // 0
(A)new C(A.x = 1, A.y = 2, B.x = 3).GET_x();  // 1
\end{lstlisting}

\section{Formalization}~\label{sec:formalization}
This version:
\begin{itemize}
	\item void setters
	\item statements
	\item explicit constructor. (?)
\end{itemize}
\begin{figure*}
	\saveSpaceFig
	\begin{displaymath}
		\begin{array}{l}
			\begin{array}{llrl}
				\text{Program}		& P & \Coloneqq  & \overline{IL} \ \overline{S}  \\
				\text{Interfaces}   & IL & \Coloneqq & \interfaceMC{I}{I}{MC?}{M} \\
				\text{Constructors} & MC & \Coloneqq & \constructor{I_x}{J.x} \\
				\text{Methods}      & M  & \Coloneqq & \method{I}{m}{I_x}{x}{J}{e}  \mid
													   \absmethod{I}{m}{I_x}{x}{J} \\
				\text{Statements}   & S & \Coloneqq  & e; \mid I \; x; \mid x = e; \\
				\text{Expressions}  & e  & \Coloneqq & 
				x \mid
				e.m(\overline{e}) \mid
				(I)e \mid 
				\kwnew \; I(\overline{e}) \mid \\
				\text{Context}      & \Gamma & \Coloneqq & \overline{x}:\overline{I} \\
				\text{Values}       & v & \Coloneqq & (I)o \\
			\end{array}
		\end{array}
	\end{displaymath}
	\caption{Syntax}\label{fig:syntax}
	\saveSpaceFig
\end{figure*}

%\begin{figure*}[]
%	\saveSpaceFig
%	\begin{mathpar}
%		\framebox{$ I <: J $} \hspace{.5in} \subid \\
%		\subtrans \hspace{.5in} \subextends \\
%		
%		\framebox{$ \judgeewf \Gamma {e:I} $} \hspace{.5in}
%		\tvar \\
%		\tinvk \\
%		% \tpathinvk \\
%		% \tsuperinvk \\
%		% \tstaticinvk  \\
%		\tnew \\
%		\tanno \\
%		\tmethod \\
%		\tabsmethod \\
%		\tintf
%	\end{mathpar}
%	\saveSpaceFig
%	\caption{Subtyping and Typing Rules}
%		\label{fig:typingrules}
%	\end{figure*}
	
\begin{figure}
	\begin{mathpar}
		\inferrule* [left=(T-Var)]
		{\judgeewf {} {\Delta \; OK} \\ \judgeewf \Delta {\Gamma \; OK} }
		{\judgeewf {\Delta; \Gamma, x:I}{x:I}}
		
		\inferrule* [left=(T-Invk)]
		{  \judgeewf {\Delta; \Gamma} {e_0:I_0}
			\\ \mbody(m, I_0, I_0) = (K, \overline{I_x} \; \overline{x}, I \; \_) 
			\\ \judgeewf {\Delta; \Gamma} \overline{e}:\overline{I}
			\\ \overline{I} \subtype \overline{I_x}
		}
		{\judgeewf {\Delta, \Gamma} e_0.m(\overline{e}):I} 
		
		\inferrule* [left=(T-Constructor)]
		{  \mconstr(I_0) = \overline{I_x} \; \overline{x} 
			\\ \judgeewf {\Delta; \Gamma} \overline{e}:\overline{I}
			\\ \overline{I} \subtype \overline{I_x}
		}
		{\judgeewf {\Delta, \Gamma} \newI{I_0}{\overline{e}}:I_0} 
		
		\inferrule* [left=(T-Cast)]
		{  \judgeewf {\Delta, \Gamma}\Gamma e:I  \\
			I \subtype J  }
		{ \judgeewf {\Delta, \Gamma} {(J)e : J} }
		
		\inferrule* [left=(T-Method)]
		{  I <: J \\
				\mostSpecific(m, I, J) = \{J\} \\
				\mbody(m, J, J) = (K, \overline{I_x} \; \overline{x}, I_e \; \_) \\
				\judgeewf {\Delta; \Gamma, \overline{x}:\overline{I_x}, this:I  } {e_0:I_0} \\
				I_0 \subtype I_e
		}
		{\method{I_e}{m}{I_x}{x}{J}{e_0} \text{ OK IN } I}
	
		\inferrule* [left=(T-AbsMethod)]
		{  I <: J \quad
			\mostSpecific(m, I, J) = \{J\} \\
			\mbody(m, J, J) = (K, \overline{I_x} \; \overline{x}, I_e \; \_)
		}
		{\absmethod{I_e}{m}{I_x}{x}{J} \text{ OK IN } I}
		
		\inferrule* [left=(T-MC)]
		{  \validMC(I, \overline{I_x}, \overline{J.x}) 
		}
		{\constructor{I_x}{J.x} \text{ OK IN } I}
		
		
		\inferrule* [left=(T-Intf)]
		{  MC \text{ OK IN }  I \\
			\overline{M} \text{ OK IN } I \\
			\forall J >: I \text{ and } m, \mbody(m, J, J) \text{ is defined} \Rightarrow \mbody(m, I, J) \text{ is defined} \\ 
			\forall J >: I \text{ and } m, I[m \; \kwoverride \; I] \text{ and } J[m \; \kwoverride \; J] \text{ defined}  \Rightarrow \canOverride(m, I, J)
		}
		{ \interfaceMC{I}{I}{MC?}{M} \text{ OK }}
		
		\inferrule* [left=(T-Prog)]
		{  \forall I \in \overline{IL}, I \text{ OK} \\
		   \forall S \in \overline{S}, S \text{ OK}
		}
		{ \prog{IL}{S} \text{ OK }}
		
		\inferrule* [left=(T-ExprStatement)]
		{ \judgeewf {\Delta, \Gamma} {e : J} }
		{e\;;\text{ OK}}

		\inferrule* [left=(T-VarIntro)]
		{I \text{ OK}}
		{I \; x; \text{ OK}}

		\inferrule* [left=(T-VarAssign)]
		{\judgeewf {\Gamma} {x : I} \\ 
		 \judgeewf {\Delta, \Gamma} {e : J} \\
		 J \subtype I
		}
		{x = e; \text{ OK}}
		% \inferrule* [left=(T-Let)]
		% {  	\judgeewf {\Delta; \Gamma} {e_1:I_1} \\
		% 	I_1 \subtype I \\	
		% 	\judgeewf {\Delta; \Gamma, x:I} {e_2:I_2} 
		% }
		% { 	\judgeewf {\Delta; \Gamma} {\mylet{I}{x}{e_1}{e_2}:I_2} }
		
		\end{mathpar}
	\caption{Typing Rules.}
\end{figure}

\begin{figure*}
	\saveSpaceFig
	\begin{displaymath}
		\begin{array}{l}
			\begin{array}{llrl}
				\text{Configuration} & Config & \Coloneqq  & (H, VS, e, FS) \\
				%    VS & \Coloneqq & [MS] \\
				%    MS & \Coloneqq & [BS] \\
				\text{Scope} & VS & \Coloneqq & \text{is a finite partial function from variables to pairs of expression types and values} \\
				\text{Heap} & H & \Coloneqq & \text{is a finite partial function from oids to heap objects} \\
				\text{Heap Objects} & ho & \Coloneqq & (I, F) \\
				\text{Map Function} & F & \Coloneqq &  \text{is a finite partial function from field names to values} \\
				%    FS & \Coloneqq & [F] \\
				\text{Frame Stack} & FS & \Coloneqq &  \concat{F}{FS} \\
				\text{Frame} & F & \Coloneqq &  CF | OF \\
				\text{Closed Frame} & CF & \Coloneqq & x \mid
				e.m(\overline{e}) \mid
				(I)e \mid 
				I.m(\overline{e}) \mid 
				\mylet{I}{x}{e_1}{e_2}  \text{ expressions} \\
				\text{Open Frame} & OF & \Coloneqq &  \text{expressions with holes} \\
			\end{array}
		\end{array}
	\end{displaymath}
	\caption{Configuration}\label{fig:}
	\saveSpaceFig
\end{figure*}

%\begin{figure*}
%	\saveSpaceFig
%	\begin{displaymath}
%	\begin{array}{l}
%	\begin{array}{llrl}
%    \text{Configuration} & Config & \Coloneqq  & (H, Sc, Fr) \\
%%    VS & \Coloneqq & [MS] \\
%%    MS & \Coloneqq & [BS] \\
%    \text{Scope} & Sc & \Coloneqq & \text{is a finite partial function from variables to pairs of expression types and values} \\
%    \text{Heap} & H & \Coloneqq & \text{is a finite partial function from oids to heap objects} \\
%    \text{Heap Objects} & ho & \Coloneqq & (I, F) \\
%    \text{Map Function} & F & \Coloneqq &  \text{is a finite partial function from field names to values} \\
%%    FS & \Coloneqq & [F] \\
%    \text{Frame} & Fr & \Coloneqq &  e\\
%	\end{array}
%	\end{array}
%	\end{displaymath}
%	\caption{Configuration}\label{fig:}
%	\saveSpaceFig
%\end{figure*}

\begin{comment}
\begin{figure*}
	\begin{mathpar}
		\textsc{(E-VarAccess) } 
		{ (H, Sc, x) \to (H, Sc, v)} \\
		\text{ where } eval(Sc, x) = (v, I) \\
		\\
		\textsc{(E-FieldAcess) } 
		{ (H, Sc, o.f()) \to (H, Sc, v)} \\
		\text{ where } o \in dom(H), H(o) = (I, F), F(f) = v \\
		\\
		\textsc{(E-FieldWrite) } 
		{ (H, Sc, o.f(v)) \to (H', Sc, o)} \\
		\text{ where } o \in dom(H), H(o) = (I, F), f \in dom(F), H' = H[o \to (I, F')] \text{ with } F[f \to v] \\
		\\
		\textsc{(E-Cast) } 
		{ (H, Sc, (I_2)o) \to (H, Sc, o)} \\
		\text{ where } o \in dom(H), H(o) = (I_1, F), I_1 \subtype I_2 \\
		\\
		\textsc{(E-Invk) } 
		{ (H, Sc, (J)o.m(\overline{v})) \to (H, Sc' \circ Sc, (I_e)e_0)} \\
		\text{ where } 
		o \in dom(H), H(o) = (I, F), Sc' = \{ this \to (o, I), \overline{x \to (v, I_x)} \} , \\
		\mbody(m, I, J) =  (I_0, \overline{I_x} \; \overline{x}, I_e \; e_0) \\
		\\
		\textsc{(E-MC) } 
		(H, Sc, (J)I.m(\overline{v})) \to (H[o \to (I, F)], Sc' \circ Sc, o) \\
		\text{ where the fresh object } o \notin dom(H), F = \{ f \to v \} \; \forall f \in \fields(I) \\
	\end{mathpar}
	\caption{Operational Semantics.}
\end{figure*}
\end{comment}

\begin{figure*}
	\begin{mathpar}
		\yanlin{\textit{----added rules:------}\\}
		
		\textsc{(E-VarIntro)}
		{(H, \concat{BS}{VS}, I \; x;, FS) \to (H, \concat{BS'}{VS}, ;, FS)}\\
		\text{ where } BS' = BS[x\to(null, I)] \\

		\textsc{(E-VarAssign)}
		{(H, \concat{BS}{VS}, x = v;, FS) \to (H, \concat{BS'}{VS}, ;, FS)}\\
		\text{ where } BS' = update(BS, (x \to v)).\\

		\textsc{(EC-VarAssign)}
		{(H, \concat{BS}{VS}, x = e;, FS) \to (H, \concat{BS}{VS}, e;, \concat{(x = \bullet)}{FS})}\\

		\textsc{(E-ExpStatement)}
		{(H, \concat{BS}{VS}, e;, FS) \to (H, \concat{BS}{VS}, e, FS)}\\

		\textsc{(E-Skip)}
		{(H, VS, ;, \concat{F}{FS}) \to (H, VS, F, FS)}\\

		\yanlin{\textit{----added rules end------}\\}

		\textsc{(E-VarAccess) } 
		{ (H, \concat{BS}{VS}, x, FS) \to (H, \concat{BS}{VS}, v, FS)} \\
		\text{ where } BS(x) = v \\
		\\
	
	   \textsc{(E-Return)} 
	   {(H, \concat{BS}{VS}, v, \concat{(\return{\bullet})}{FS}) \to 
	     (H, VS, v, FS)} \\
	
		\textsc{(E-ReturnLet)}
		{ (H, \concat{BS}{VS}, v, \concat{(\returnLet{\bullet})}{FS}) \to
			(H, \concat{BS'}{VS}, v, FS) }\\
		\text{ where } BS' = \tail(BS)
%	   \textsc{(E-Return)} 
%	   {(H, \concat{BS}{VS}, \return{v}, FS) \to 
%	     (H, VS, v, FS)} \\
	   
%		\textsc{(EC-Return)}
%		{(H, VS, \return{e}, FS) \to 
%			(H, VS, e, \concat{(\return{\bullet})}{FS})			
%		}\\
		
		\textsc{(E-Invk)}
		{ (H, VS, (J)o.m(\overline{v}), FS) \to
			(H, VS', (I_e)e_0, \concat{(\return{\bullet})}{FS})  }\\
		\text{ where } m \text{ is not a getter ($I_0.m$ does not exist in $I.of$)},  
			   H(o) = \obj{I}{\overline{o}}, 
		       VS' = \concat{\{\this \to (I_0)o, \overline{x \to v} \}}{VS},
		       \mbody(m, I, J) = (I_0, \overline{I_x} \; \overline{x}, I_e \; e_0)
		
		\textsc{(E-InvkGetter)}
		{ (H, VS, (J)o.f(), FS) \to (H, VS, (I_F)o_i, FS) } \\
	    \text{ where } H(o) = \obj{I}{o_1, ...., o_i, ...},
	    \mbody(m, I, J) = (I_0, \overline{I_x} \; \overline{x}, I_e \; e_0),
	    (I_F \; I_0.f) \text{ is the i-th element of } I.of.
 		
		\textsc{(EC-Invk1)}
		{(H, VS, e.m(\overline{e}), FS) \to
		  (H, VS, e, \concat{(\bullet.m(\overline{e}))}{FS}) }\\
		
		\textsc{(EC-Invk2)}
		{(H, VS, v.m(v_1, ..., v_{i-1}, e_i, ..., e_n), FS) \to
			(H, VS, e_i, \concat{(v.m(v_1,...,v_{i-1}, \bullet, ..., e_n))}{FS}) }\\
		
		\text{(E-Cast)}
		{(H, VS, (J)((I)o), FS) \to
			(H, VS, (J)o, FS) 
		}\\ %\text{ where } I \subtype J 
		
		\text{(EC-Cast)}
		{(H, VS, (J)e, FS) \to
			(H, VS, e, \concat{((J)\bullet)}{FS})
		}
	\end{mathpar}
	\caption{Small-step Semantics (1).}
\end{figure*}

\begin{figure*}
	\begin{mathpar}
		\textsc{(EC-new)} 
		(H, VS, \newI{I}{v_1, ..., v_{i-1}, e_i, ..., e_n}, FS) \to 
		(H, VS, e_i, \concat{\newI{I}{v_1, ..., v_i-1, \bullet, ..., e_n}}{FS}) \\
		
		\textsc{(E-new)}
		{(H, VS, \newI{I}{\overline{v}}, FS) \to
			(H', VS, (I)o, FS)
		} \\  \text{ where } \overline{v} = \overline{(I)o}, 
	    H' = H[o \to \obj{I}{\overline{o}}], \text{ fresh } o \notin dom(H) \\

		\textsc{(E-Set)}
		{ (H, VS, (I)o.SET\_f((I')o'), FS) \to
			(H', VS, (K)o, FS)
		} \\ 
	    \text{ where } H(o) =\obj{K}{o_1, ..., o_n},
	       H' = H[o \to \obj{K}{o_1, ..., o_{i-1}, o', o_{i+1}, ... o_n}], 
	            f \text{ is the } i^{th} \text{ of } K's \text{ constructor.} \\
		
		\textsc{(EC-Set1)}
		{   (H, VS, e.SET\_f(e'), FS) \to 
			 (H, VS, e, \concat{(\bullet.SET\_f(e'))}{FS})
		}\\
	
	   \textsc{(EC-Set2)}
	   { (H, VS, v.SET\_f(e'), FS) \to
	   	  (H, VS, e', \concat{(v.SET\_f(\bullet))}{FS})
	   }\\
   
     \textsc{(EC-Fill)}
     { (H, VS, v, \concat{F}{FS}) \to (H, VS, F(v), FS)} 
	\end{mathpar}
	\caption{Small-step Semantics (2).}
\end{figure*}

%\begin{figure*}
%	\begin{mathpar}
%		\textsc{(E-VarAccess) } 
%		{ (H, MS \circ VS, x, FS) \to (H, MS \circ VS, v, FS)} \\
%		\text{ where } eval(MS, x) = (v, I) \\
%		\\
%		\textsc{(E-FieldAcess) } 
%		{ (H, VS, o.f(), FS) \to (H, VS, v, FS)} \\
%		\text{ where } o \in dom(H), H(o) = (I, F), F(f) = v \\
%		\\
%		\textsc{(E-FieldWrite) } 
%		{ (H, VS, o.f(v), FS) \to (H', VS, \kwvoid, FS)} \\
%		\text{ where } o \in dom(H), H(o) = (I, F), f \in dom(F), H' = H[o \to (I, F')] with F[f \to v] \\
%		\\
%		\textsc{(E-Cast) } 
%		{ (H, VS, (I_2)o, FS) \to (H, VS, o, FS)} \\
%		\text{ where } o \in dom(H), H(o) = (I_1, F), I_1 \subtype I_2 \\
%		\\
%		\textsc{(E-Invk) } 
%		{ (H, VS, (J)o.m(\overline{v}), FS) \to (H, (BS \circ [])\circ VS, (I_e)e_0, FS)} \\
%		\text{ where } 
%		o \in dom(H), H(o) = (I, F), BS = \{ this \to (o, I), \overline{x \to (v, I_x)} \} , \\
%		\mbody(m, I, J) =  (I_0, \overline{I_x} \; \overline{x}, I_e \; e_0) \\
%		\\
%		\textsc{(E-MC) } 
%		(H, VS, (J)I.m(\overline{v}), FS) \to (H[o \to (I, F)], (BS \circ [])\circ VS, o, FS) \\
%		\text{ where the fresh object } o \notin dom(H), F = \{ f \to v \} \; \forall f \in \fields(I) \\
%		
%		
%%		%%%%%%**** S-Invk ****%%%%%%%
%%		\text{\yanlin{(S-Invk) is similar to FHJ} } \\
%%		\inferrule* [left=(S-Invk)]
%%		{ \mbody(m, I, J) = (I_0, \overline{I_x} \; \overline{x}, I_e \; e_0) }
%%		{ \left((J)\neww{I}(\overline{K.f=v})\right).m(\overline{d}) \to
%%				(I_e)[
%%				\overline{(I_x)d}
%%				/\overline{x}, (I_0)\neww{I}(\overline{K.f=v})/\kwthis]e_0 }
%	\end{mathpar}
%	\caption{Operational Semantics.}
%\end{figure*}
	

\section{Aux}

\subsection{$\Delta$}

Definition of $\Delta(I)(J.f)$:
$\Delta(I)(J.f) = K$, where $K \; J.f \in \overline{K \; J.f}$, where $static \; I \; of (\overline{K \; J.f});$ is the method constructor of interface $I$.

\subsection{calField}
$ calField(I, J, \overline{K.f=e}, f_i)$
\begin{itemize}
	\item if $I.f_i \in \overline{K.f}$, return $e_i$ where $I.f_i = e_i$
	\item if $J.f_i \in \overline{K.f}$, return $e_i$ where $J.f_i = e_i$ 
\end{itemize}

\subsection{updateField}
similar as $calField$.

\subsection{validMC}
$\validMC(I, \overline{I_x}, \overline{J.x}) $:
\begin{itemize}
	\item If there are no abstract state operations, then empty argument list is valid.
	\item $\forall J.x \in \overline{J.x}, J.x \text{ is a valid field of } I, \text{ and can be seen as abstrate state operations, and the type matches } I_x$.
	\item The interface $I$ can be instantiated, meaning that, every abstrate field appears in the argument list of the constructor method.
\end{itemize}

\subsection{The Method Lookup Algorithm in \mbody{}}\label{subsec:mbodydef}
$\mbody(m, I_d, I_s)$ denotes the method body lookup function.
We use $I_d, I_s$, since $\mbody$ is usually invoked by a receiver of a method $m$, with its dynamic type $I_d$ and static type $I_s$. Such a function returns the most specific method implementation. More
accurately, $\mbody$ returns the parameters, returned expression
(empty for abstract methods) and the types for the method. It considers both originally-defined methods and hierarchical overriding methods, so $\mostSpecific$ and $\mostSpecificOverride$ (see the definition in Section~\ref{sec:mostSpecific} and Section~\ref{sec:mostSpecificOverride}) are both invoked.

\saveSpaceFig
\begin{flalign*}
	& \rhd \textit{Definition of } \mbody(m, I_d, I_s): & \\
	& \bullet \mbody(m, I_d, I_s) = (J, \overline{I_x} \; \overline{x}, I_e \; e_0) & \\
	& \indent\indent \textrm{with: } \mostSpecific(m, I_d, I_s) = \{I\} & \\
	& \hspace{.77in} \mostSpecificOverride(m, I_d, I) = \{J\} & \\
	& \hspace{.77in} J[m\ \kwoverride\ I] = \method{I_e}{m}{I_x}{x}{I}{e_0} & \\
	& \bullet \mbody(m, I_d, I_s) = (J, \overline{I_x} \; \overline{x}, I_e \; \o) & \\
	& \indent\indent \textrm{with: } \mostSpecific(m, I_d, I_s) = \{I\} & \\
	& \hspace{.77in} \mostSpecificOverride(m, I_d, I) = \{J\} & \\
	& \hspace{.77in} J[m\ \kwoverride\ I] = \absmethod{I_e}{m}{I_x}{x}{I} & \\
\end{flalign*}
\saveSpaceFig

To calculate $\mbody(m, I_d, I_s)$, the invocation of $\mostSpecific$ looks for the most specific original methods and their interfaces, and expects a singleton set, so as to achieve unambiguity. Furthermore, the invocation of $\mostSpecificOverride$ also expects a unique and most specific hierarchical override. And finally the target method is returned.

\subsection{\mostSpecific}\label{sec:mostSpecific}
We proceed to give the definitions of two core functions that support method lookup, namely \mostSpecific{} and \mostSpecificOverride. Generally,
$\mostSpecific(m, I, J)$ finds the set of most specific interfaces where $m$ is originally defined. Interfaces in this set should be above interface $I$ and along path $J$. Finally with $\prune$ (defined in Section~\ref{sec:otherdefs})
the overridden interfaces will be filtered out.

\saveSpaceFig
\begin{flalign*}
	& \rhd \textit{Definition of } \mostSpecific(m, I, J): & \\
	& \bullet \mostSpecific(m, I, J) = \prune(origins) & \\
	& \indent\indent \textrm{with: } origins = \{K \mid \subt{I}{K}, \textrm{ and } \subt{K}{J} \; \lor \; \subt{J}{K}, &\\
	& \hspace{1.62in} \textrm{ and } K[m\ \kwoverride\ K] \textrm{ is defined} \} &
\end{flalign*}
By the definition, an interface belongs to $\mostSpecific(m, I, J)$ if and only if:
\begin{itemize}
	\item It originally defines $m$;
	\item It is a supertype of $I$ (including $I$);
	\item It is either a supertype or a subtype of $J$ (including $J$);
	\item Any subtype of it does not belong to the same result set because of $\prune$.
\end{itemize}
\saveSpaceFig

\subsection{\mostSpecificOverride}\label{sec:mostSpecificOverride}
The $\mostSpecific$ function only focuses on original method
implementations, where all the hierarchical overriding methods are omitted
during that step. On the other hand, $\mostSpecificOverride(m, I, J)$
has the assumption that $J$ defines an original $m$, and this function
tries to find the interfaces with the most specific implementations that hierarchically overrides such an $m$. Formally,

\saveSpaceFig
\begin{flalign*}
	& \rhd \textit{Definition of } \mostSpecificOverride(m, I, J): & \\
	& \bullet \mostSpecificOverride(m, I, J) = \prune(overrides) & \\
	& \indent\indent \textrm{with: } overrides = \{K \mid \subt{I}{K}, \; \subt{K}{J} \textrm{ and } K[m\ \kwoverride\ J] \textrm{ is defined} &
\end{flalign*}
\saveSpaceFig

By the definition, an interface belongs to $\mostSpecificOverride(m, I, J)$ if and only if:
\begin{itemize}
	\item it is between $I$ and $J$ (including $I$, $J$);
	\item it hierarchically overrides $J.m$;
	\item any subtype of it does not belong to the same set.
\end{itemize}

\subsection{others}\label{sec:otherdefs}
Below we give other minor definitions of the auxiliary functions that are used in previous sections.

%%%%============================ I[m override J] ================%%%%%%%%
\begin{flalign*}
	& \rhd \textit{Definition of } I[m\ \kwoverride\ J]: & \\
	& \bullet I[m\ \kwoverride\ J] = \method{I_e}{m}{I_x}{x}{J}{e_0} & \\
	& \indent\indent \textrm{with: }
	\kwinterface \; I \; \kwextends \; \overline{I} \; \{ \method{I_e}{m}{I_x}{x}{J}{e_0} \ldots \} & \\
	& \bullet I[m\ \kwoverride\ J] = \absmethod{I_e}{m}{I_x}{x}{J} & \\
	& \indent\indent \textrm{with: }
	\kwinterface \; I \; \kwextends \; \overline{I} \; \{ \absmethod{I_e}{m}{I_x}{x}{J} \ldots \} & \\
\end{flalign*}
Here $I[m\ \kwoverride\ J]$ is basically a direct lookup for method $m$ in the body of $I$, where such a method
overrides $J$ (like static dispatch). The method can be either concrete or abstract, and the body of definition is returned. Notice that
by our syntax, $I[m\ \kwoverride\ I]$ is looking for the originally-defined method $m$ in $I$.
%%%%============================ I[m override J] end================%%%%%%%%

%%%%============================ prune(set) ================%%%%%%%%
\saveSpaceFig
\begin{flalign*}
	& \rhd \textit{Definition of } \prune(set): & \\
	& \bullet \prune(set) = \{I \in set \; | \; \nexists J \in set\setminus I, J <: I\} &
\end{flalign*}

The $\prune$ function takes a set of
types, and filters out those that have subtypes in the same set. In the returned set,
none of them has subtyping relation to one another, since all supertypes have been removed.
%%%%============================ prune(set) end ================%%%%%%%%

\subsection{$\Delta$}
\saveSpaceFig
\begin{flalign*}
	& \rhd \textit{Definition of } \Delta_f(I)(J.f) & \\
	& I', \text{ where } J == I  \text{ and } I' f(); \text{ defined in } I. & \\
	& I'', \text{ where } I'' f(); \text{ defined in } J \text{ and } \forall I \subtype J' 
	\subtype J, \text{ there is no field } f \text{ defined in } J'.   &
\end{flalign*}

\subsection{$\eval(Sc, x)$}
\saveSpaceFig
\begin{flalign*}
	& \rhd \textit{Definition of } \eval(Sc, x): & \\
	& \bullet \eval(Sc, x) = Sc(x), x \in dom(Sc) & 
\end{flalign*}

\subsection{example}
\begin{lstlisting}
interface I {
	int x();
	static I of(int I.x);
}

interface A {
	int x();
}
interface B {
	int x();
}
interface C extends A, B {
	static C of (int A.x, int B.x);
}
C.of(3,2)
\end{lstlisting}


