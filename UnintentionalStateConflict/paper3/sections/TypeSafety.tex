\begin{lemma}~\label{lemma:return_expr_typing}
% $\textit{If } \mtype(m, I, J) = \overline{D} \rightarrow D, \textit{ and } \mbody(m, I, J) = \overline{x}.e, 
%  \textit{ then for some } J_0 \textit{ with } I <: J_0, \  \exists C <: D, \textit{ such that }  
%  \judgeewf {\overline{x}:\overline{D}, \kwthis:J_0} {e:C} $.
If $\mbody(m, I_d, I_s) = (J, \overline{I_x} \; \overline{x}, I_e \; e_0)$, then
 $\judgeewf {\overline{x}:\overline{I_x}, \kwthis:J} {e_0 : I_0}$ for some $I_0 \subtype I_e$.
\end{lemma}

\begin{proof}
By the definition of $\mbody$, the target method $m$ is found in $J$. By the method typing rule \textsc{(T-Method)}, there exists some 
$I_0 \subtype I_e$ such that $\judgeewf {\overline{x}:\overline{I_x}, \kwthis:J} {e_0 : I_0}$. 
\end{proof}

\begin{lemma}[Weakening]~\label{lemma:weakening}
	If $\judgeewf {\Gamma} {e : I}$, then $\judgeewf {\Gamma, x : J} {e : I}$.
\end{lemma}

\begin{proof}
Straightforward induction.
\end{proof}


\begin{lemma}[Method Type Preservation]~\label{lemma:mbody_type_preservation}
If $\mbody(m, J, J) = (K, \overline{I_x}\ \_, I_e\ \_)$, then for any $I \subtype J$, $\mbody(m, I, J) = (K', \overline{I_x}\ \_, I_e\ \_)$.
\end{lemma}
\begin{proof}
	
\begin{comment}	
Since $\mbody(m, J, J)$ is defined, by \textsc{(T-Intf)} we derive that $\mbody(m, I, J)$ is also defined. Suppose that $$\mostSpecific(m, J, J) = \{I_0\}$$ $$\mostSpecificOverride(m, J, I_0) = \{K\}$$
$$\mostSpecific(m, I, J) = \{I'_0\}$$ $$\mostSpecificOverride(m, I, I'_0) = \{K'\}$$

Below we use $I[m\uparrow J]$ to denote the type of method $m$ defined in $I$ that overrides $J$. We have to prove that $K'[m\uparrow I'_0] = K[m\uparrow I_0]$.
Two facts:
\begin{itemize}
	\item A. By $\textsc{(T-Intf)}$, $\canOverride$ ensures that an override between any two original methods preserves the method type. Formally, $$I_1 \subtype I_2 \ \ \Rightarrow\ \ I_1[m\uparrow I_1] = I_2[m\uparrow I_2]$$
	\item B. By $\textsc{(T-Method)}$ and $\textsc{(T-AbsMethod)}$, any partial override also preserves method type. Formally,
	  $$I_1 \subtype I_2\ \ \Rightarrow\ \ I_1[m\uparrow I_2] = I_2[m\uparrow I_2]$$
\end{itemize}

By definition of $\mostSpecificOverride$, $K \subtype I_0, K' \subtype I'_0$. By Fact B, $$K[m\uparrow I_0] = I_0[m\uparrow I_0] \quad K'[m\uparrow I'_0] = I'_0[m\uparrow I'_0]$$

Hence it suffices to prove that $I'_0[m\uparrow I'_0] = I_0[m\uparrow I_0]$. Actually when calculating $\mostSpecific(m, J, J)$, by the definition of $\mostSpecific$ we know that $I_0 \subtype J$ and $I_0[m \; \kwoverride \; I_0]$ is defined. So when calculating $\mostSpecific(m, I, J)$ with $I \subtype J$, $I_0$ should also appear in the set before pruned, since the conditions are again satisfied. But after pruning, only $I'_0$ is obtained, by definition of $\prune$ it implies $I'_0 \subtype I_0$. By Fact A, the proof is done.
\end{comment}

\end{proof}


%%%%%%%%%%%%%%%%%%%%%%%%%%%%%%%%%%%%%%%%%%%%%%%%%%%%
\begin{lemma}[Term Substitution Preserves Typing]~\label{lemma:subst_type_preservation}
If $\judgeewf {\Gamma, \overline{x}:\overline{I_x}} { e : I }$, and
$\judgeewf {\Gamma} {\overline{y}:\overline{I_x}}$, then
$\judgeewf {\Gamma} {[\overline{y}/\overline{x}]e : I}$.

\begin{proof}
% We prove by induction. The expression $e$ has the following cases:

% \textbf{Case Var.} Let $e = x$. If $x \notin \overline{x}$, then the substitution does not change anything. Otherwise,
% since $\overline{y}$ have the same types as $\overline{x}$, it immediately finishes the case.

% \textbf{Case Invk.} Let $e = e_0.m(\overline{e})$. By \textsc{(T-Invk)} we can suppose that
% 	$$\judgeewf {\Gamma, \overline{x}:\overline{I_x}} {e_0 : I_0} \quad \mbody(m, I_0, I_0) = (\_, \overline{J}\ \_, I\ \_)$$
% 	$$\judgeewf {\Gamma, \overline{x}:\overline{I_x}} {\overline{e} : \overline{I_e}} \quad 
% 	\overline{I_e} \subtype \overline{J} \quad \judgeewf {\Gamma, \overline{x}:\overline{I_x}} {e : I}$$
	
% By induction hypothesis, 
% 	$$\judgeewf {\Gamma} {[\overline{y}/\overline{x}]e_0 : I_0} \quad
% 	  \judgeewf {\Gamma} {[\overline{y}/\overline{x}]\overline{e} : \overline{I_e}}$$
	  
% Again by \textsc{(T-Invk)}, $\judgeewf {\Gamma} {[\overline{y}/\overline{x}]e : I}$.

% \textbf{Case New.} Straightforward.

% \textbf{Case Anno.} Straightforward by induction hypothesis and \textsc{(T-Anno)}.
\end{proof}

\end{lemma}
%%%%%%%%%%%%%%%%%%%%%%%%%%%%%%%%%%%%%%%%%%%%%%%%%%%%