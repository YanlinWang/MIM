\begin{lemma}~\label{lemma:return_expr_typing}
% $\textit{If } \mtype(m, I, J) = \overline{D} \rightarrow D, \textit{ and } \mbody(m, I, J) = \overline{x}.e, 
%  \textit{ then for some } J_0 \textit{ with } I <: J_0, \  \exists C <: D, \textit{ such that }  
%  \judgeewf {\overline{x}:\overline{D}, \kwthis:J_0} {e:C} $.
If $\mbody(m, I_d, I_s) = (J, \overline{I_x} \; \overline{x}, I_e \; e_0)$, then
 $\judgeewf {\overline{x}:\overline{I_x}, \kwthis:J} {e_0 : I_0}$ for some $I_0 \subtype I_e$.
\end{lemma}

\begin{proof}
By the definition of $\mbody$, the target method $m$ is found in $J$. By the method typing rule \textsc{(T-Method)}, there exists some 
$I_0 \subtype I_e$ such that $\judgeewf {\overline{x}:\overline{I_x}, \kwthis:J} {e_0 : I_0}$. 
\end{proof}

\begin{lemma}[Weakening]~\label{lemma:weakening}
	If $\judgeewf {\Gamma} {e : I}$, then $\judgeewf {\Gamma, x : J} {e : I}$.
\end{lemma}

\begin{proof}
Straightforward induction.
\end{proof}


\begin{lemma}[Method Type Preservation]~\label{lemma:mbody_type_preservation}
If $\mbody(m, J, J) = (K, \overline{I_x}\ \_, I_e\ \_)$, then for any $I \subtype J$, $\mbody(m, I, J) = (K', \overline{I_x}\ \_, I_e\ \_)$.
\end{lemma}
\begin{proof}
	
%\begin{comment}	
Since $\mbody(m, J, J)$ is defined, by \textsc{(T-Intf)} we derive that $\mbody(m, I, J)$ is also defined. Suppose that $$\mostSpecific(m, J, J) = \{I_0\}$$ $$\mostSpecificOverride(m, J, I_0) = \{K\}$$
$$\mostSpecific(m, I, J) = \{I'_0\}$$ $$\mostSpecificOverride(m, I, I'_0) = \{K'\}$$

Below we use $I[m\uparrow J]$ to denote the type of method $m$ defined in $I$ that overrides $J$. We have to prove that $K'[m\uparrow I'_0] = K[m\uparrow I_0]$.
Two facts:
\begin{itemize}
	\item A. By $\textsc{(T-Intf)}$, $\canOverride$ ensures that an override between any two original methods preserves the method type. Formally, $$I_1 \subtype I_2 \ \ \Rightarrow\ \ I_1[m\uparrow I_1] = I_2[m\uparrow I_2]$$
	\item B. By $\textsc{(T-Method)}$ and $\textsc{(T-AbsMethod)}$, any partial override also preserves method type. Formally,
	  $$I_1 \subtype I_2\ \ \Rightarrow\ \ I_1[m\uparrow I_2] = I_2[m\uparrow I_2]$$
\end{itemize}

By definition of $\mostSpecificOverride$, $K \subtype I_0, K' \subtype I'_0$. By Fact B, $$K[m\uparrow I_0] = I_0[m\uparrow I_0] \quad K'[m\uparrow I'_0] = I'_0[m\uparrow I'_0]$$

Hence it suffices to prove that $I'_0[m\uparrow I'_0] = I_0[m\uparrow I_0]$. Actually when calculating $\mostSpecific(m, J, J)$, by the definition of $\mostSpecific$ we know that $I_0 \subtype J$ and $I_0[m \; \kwoverride \; I_0]$ is defined. So when calculating $\mostSpecific(m, I, J)$ with $I \subtype J$, $I_0$ should also appear in the set before pruned, since the conditions are again satisfied. But after pruning, only $I'_0$ is obtained, by definition of $\prune$ it implies $I'_0 \subtype I_0$. By Fact A, the proof is done.
%\end{comment}

\end{proof}


%%%%%%%%%%%%%%%%%%%%%%%%%%%%%%%%%%%%%%%%%%%%%%%%%%%%
\begin{lemma}[Term Substitution Preserves Typing]~\label{lemma:subst_type_preservation}
If $\judgeewf {\Gamma, \overline{x}:\overline{I_x}} { e : I }$, and
$\judgeewf {\Gamma} {\overline{y}:\overline{I_x}}$, then
$\judgeewf {\Gamma} {[\overline{y}/\overline{x}]e : I}$.

\begin{proof}

\begin{comment}
\noindent \textbf{Case Var.}
$ e = x \quad I = \Gamma(x) $. \\
If $x \notin \overline{x}$, then the conclusion is immediate, since $[\overline{y}/\overline{x}]x = x$.
On the other hand, if $x = x_i$ and $I = {I_x}_i$, then since $[\overline{y}/\overline{x}]x = [\overline{y}/\overline{x}]x_i = y_i$,
letting $I' = {I_y}_i$ finishes the case.

\noindent \textbf{Case New.}
$e = \new I$ and there are no term for substitution, the conclusion is obvious.
\noindent \textbf{Case Invk.}
$ e = e_0.m(\overline{e}) \quad
  \judgeewf {\Gamma, \overline{x}:\overline{I_x}} {e_0 : I_0} $
$$ mtype(m, I_0) = \overline{I_e} \rightarrow J $$
$$ \judgeewf {\Gamma, \overline{x}:\overline{I_x}} {\overline{e}:\overline{I}} \quad
    \overline{I} \subtype \overline{I_e} $$
By induction hypothesis, there are some $I_0'$ and $\overline{I_e'}$ such that
    $$ \judgeewf {\Gamma} {[\overline{y}/\overline{x}]e_0 : I_0'} \quad 
        I_0' \subtype I_0 $$
    $$ \judgeewf {\Gamma} {[\overline{y}/\overline{x}]\overline{e} : \overline{I_e'}} \quad  
        \overline{I_e'} \subtype \overline{I}$$    
By lemma~\ref{lemma2}, 
    $mtype(m, I_0') = \overline{I_e} \rightarrow J$.
    
\noindent Then $\overline{I_e'} \subtype \overline{I_e}$ by the transitivity of $\subtype$.
Therefore, by the rule \textsc{(T-Invk)}, 
    $\judgeewf {\Gamma} {[\overline{y}/\overline{x}]e_0.m([\overline{y}/\overline{x}]\overline{e}) : J}$.

\noindent \textbf{Case PathInvk.}
$ e = e_0.I::m(\overline{e}) $ and proof is similar as case Var.

\noindent \textbf{Case SuperInvk.}
$ e = \kwsuper.I::m(\overline{e}) $ \\
Suppose $\judgeewf {\Gamma} {\kwthis : I_0}$, the following proof should be similar as case Var.
\end{comment}
We prove by induction. The expression $e$ has the following cases:

\textbf{Case Var.} Let $e = x$. If $x \notin \overline{x}$, then the substitution does not change anything. Otherwise,
since $\overline{y}$ have the same types as $\overline{x}$, it immediately finishes the case.

\textbf{Case Invk.} Let $e = e_0.m(\overline{e})$. By \textsc{(T-Invk)} we can suppose that
	$$\judgeewf {\Gamma, \overline{x}:\overline{I_x}} {e_0 : I_0} \quad \mbody(m, I_0, I_0) = (\_, \overline{J}\ \_, I\ \_)$$
	$$\judgeewf {\Gamma, \overline{x}:\overline{I_x}} {\overline{e} : \overline{I_e}} \quad 
	\overline{I_e} \subtype \overline{J} \quad \judgeewf {\Gamma, \overline{x}:\overline{I_x}} {e : I}$$
	
By induction hypothesis, 
	$$\judgeewf {\Gamma} {[\overline{y}/\overline{x}]e_0 : I_0} \quad
	  \judgeewf {\Gamma} {[\overline{y}/\overline{x}]\overline{e} : \overline{I_e}}$$
	  
Again by \textsc{(T-Invk)}, $\judgeewf {\Gamma} {[\overline{y}/\overline{x}]e : I}$.

\begin{comment}
\textbf{Case PathInvk.} $e = e_0.J_0?m(\overline{e})$. By \textsc{(T-PathInvk)}, suppose
	$$\judgeewf {\Gamma, \overline{x}:\overline{I_x}} {e_0 : I_0} \quad I_0 \subtype J_0$$
	$$\mbody(m, I_0, J_0) = (\_, \overline{J}\ \_, I\ \_)$$
	$$\judgeewf {\Gamma, \overline{x}:\overline{I_x}} {\overline{e} : \overline{I_e}} \quad 
	  \overline{I_e} \subtype \overline{J} \quad \judgeewf {\Gamma, \overline{x}:\overline{I_x}} {e : I}$$

By induction hypothesis, 
$$\judgeewf {\Gamma} {[\overline{y}/\overline{x}]e_0 : I_0} \quad
\judgeewf {\Gamma} {[\overline{y}/\overline{x}]\overline{e} : \overline{I_e}}$$ 

Again by \textsc{(T-PathInvk)}, $\judgeewf {\Gamma} {[\overline{y}/\overline{x}]e : I}$.
\end{comment}

\textbf{Case New.} Straightforward.

\textbf{Case Anno.} Straightforward by induction hypothesis and \textsc{(T-Anno)}.


\end{proof}

\end{lemma}
%%%%%%%%%%%%%%%%%%%%%%%%%%%%%%%%%%%%%%%%%%%%%%%%%%%%



\subsubsection{Proof for Theorem~\ref{theorem_subject}}
\begin{proof} ~\\
\indent \textbf{Case S-Invk.} Let
\begin{comment}
   $$e = (\angl{J}\new{I}).m(\overline{v}) \quad \judgeewf {\Gamma} {e : I_e}$$
   $$e' = (\angl{J}\new{I}).J?m(\overline{v})$$
   
Since $e$ is well-typed, $\angl{J}\new{I}$ is also well-typed. By \textsc{(T-Anno)},
  $$\judgeewf {\Gamma} {\angl{J}\new{I} : J} \quad I \subtype J$$
  
By \textsc{(T-Invk)} on $e$,
  $$\mbody(m, J, J) = (\_, \overline{I_x}\ \_, I_e\ \_) \quad \judgeewf {\Gamma} {\overline{v} : \overline{I_v}} \quad \overline{I_v}  \subtype \overline{I_x}$$
  
At this moment, \textsc{(T-PathInvk)} is immediately applicable by the above conditions, including that $e'$ also invokes $\mbody(m, J, J)$. Hence $\judgeewf {\Gamma} {e' : I_e}$.
\textbf{Case S-PathInvk.} Let
\end{comment}
  $$e = ((J)\new{I}).m(\overline{v}) \quad \judgeewf {\Gamma} {e : I_e}$$
  $$e' = (I_{e_0})[\overline{(I_x)v}/\overline{x}, (I_0)\new{I}/\kwthis]e_0$$
  $$\mbody(m, I, J) = (I_0, \overline{I_x} \; \overline{x}, I_{e_0} \; e_0)$$

Since $\mbody(m, I, J)$ is defined, the definition of $\mbody$ ensures that $I \subtype J$. And since $e$ is well-typed, by \textsc{(T-Invk)},
  $$\judgeewf {\Gamma} {\overline{v} : \overline{I_v}} \quad \overline{I_v} \subtype \overline{I_x}$$

By the rules \textsc{(T-Anno)} and  \textsc{(T-New)},
  $$\judgeewf {\Gamma} {\overline{(I_x)v} : \overline{I_x}} \quad \judgeewf {\Gamma} {(I_0)\new{I} : I_0}$$
  
On the other hand, by Lemma~\ref{lemma:return_expr_typing},
  $$\judgeewf {\overline{x} : \overline{I_x}, \kwthis: I_0} {e_0 : I'_{e_0}} \quad I'_{e_0} \subtype I_{e_0}$$
  
By Lemma~\ref{lemma:weakening},
  $$\judgeewf {\Gamma, \overline{x} : \overline{I_x}, \kwthis: I_0} {e_0 : I'_{e_0}}$$

Hence by Lemma~\ref{lemma:subst_type_preservation}, the substitution preserves typing, thus
  $$\judgeewf {\Gamma} {[\overline{(I_x)v}/\overline{x}, (I_0)\new{I}/\kwthis]e_0 : I'_{e_0}}$$
  
Since $I'_{e_0} \subtype I_{e_0}$, the conditions of \textsc{(T-Anno)} are satisfied, hence $\judgeewf {\Gamma} {e' : I_{e_0}}$. Now we only need to prove that $I_{e_0} = I_e$. Since $I_{e_0}$ is from $\mbody(m, I, J)$, whereas $I_e$ is from $\mbody(m, J, J)$, by the rule \textsc{(T-Invk)} on $e$. Since $I \subtype J$, by Lemma~\ref{lemma:mbody_type_preservation}, $I_{e_0} = I_e$.

\textbf{Case C-Receiver.} Straightforward induction.

%\textbf{Case C-PathReceiver.} Straightforward induction.

\textbf{Case C-Args.} Straightforward induction.

%\textbf{Case C-PathArgs.} Straightforward induction.

\textbf{Case C-StaticType.} Immediate by \textsc{(T-Anno)}.

\textbf{Case C-FReduce.} Immediate by \textsc{(T-Anno)} and induction.

\textbf{Case C-AnnoReduce.} Immediate by \textsc{(T-Anno)} and transitivity of $\subtype$.

\begin{comment}
\noindent \textbf{Case Invk.} 
let \[ e = \angl{J}\new I.m(\overline{\angl{I_e}}\overline{e}) \] 
Suppose \[ \mbody(m, I, J) = (J', \overline{I_x} \; \overline{x}, I_{e_0} \; e_0) \] 
then \[ e' =  [\overline{\angl{I_x}} \overline{e}/\overline{x}, \; \angl{J}\new I/\kwthis ] e_0 \] 
By rules \textsc{(T-New)} and \textsc{(T-Invk)}, 
  \[ \judgeewf \Gamma {\new I:I} \quad 
     \mtype(m, I, J) = \overline{I_x} \rightarrow I_{e_0} \quad 
     \judgeewf \Gamma {\overline{e} : \overline{I_e'}} \quad
     \overline{I_e'} \subtype \overline{I_x} \quad
     \textit{, for some } \; \overline{I_e'}
  \]
By Lemma~\ref{lemma0},
    \[
    \judgeewf {\Gamma, \overline{x}:\overline{I_x}, \kwthis:J_0} {e_0:I_f} \textit{, for some } J \subtype J_0 \textit{ and } I_f \subtype I_{e_0}
    \]
By Lemma~\ref{lemma1},
    \[
    \judgeewf {\Gamma} {[\overline{\angl{I_x}} \overline{e}/\overline{x}, \; \angl{J}\new I/\kwthis ] e_0  :  I_g} \textit{, for some } I_g \subtype I_f 
    \]
So $I_g <: I_{e_0}$, finally just let $I' = I_g$.

\noindent \textbf{Case PathInvk.}
let \[ e = \angl{J}\new I.K::m(\overline{\angl{I_e}} \overline{e}) \]  
Suppose \[ \mbody(m, I, K) = (J', \overline{I_x} \; \overline{x}, I_{e_0} \; e_0) \] 
then \[ e' =  [\overline{\angl{I_x}} \overline{e}/\overline{x}, \; \angl{K}\new I/\kwthis ] e_0 \] 
By rules \textsc{(T-New)} and \textsc{(T-Invk)}, 
  \[ \judgeewf \Gamma {\new I:I} \quad 
     \mtype(m, I, K) = \overline{I_x} \rightarrow I_{e_0} \quad 
     \judgeewf \Gamma {\overline{e} : \overline{I_e'}} \quad
     \overline{I_e'} \subtype \overline{I_x} \quad
     \textit{, for some } \; \overline{I_e'}
  \]
By Lemma~\ref{lemma0},
    \[
    \judgeewf {\Gamma, \overline{x}:\overline{I_x}, \kwthis:J_0} {e_0:I_f} \textit{, for some } K \subtype J_0 \textit{ and } I_f \subtype I_{e_0}
    \]
By Lemma~\ref{lemma1},
    \[
    \judgeewf {\Gamma} {[\overline{\angl{I_x}} \overline{e}/\overline{x}, \; \angl{K}\new I/\kwthis ] e_0  :  I_g} \textit{, for some } I_g \subtype I_f 
    \]
So $I_g <: I_{e_0}$, finally just let $I' = I_g$.

\noindent \textbf{Case Super-Invk.}
let \[ e = \kwsuper.K::m(\overline{\angl{I_e}} \overline{e}) \]   
Suppose \[ \mbody(m, K, K) = (J', \overline{I_x} \; \overline{x}, I_{e_0} \; e_0) \] 
then \[ e' =  [\overline{\angl{I_x}} \overline{e}/\overline{x}] e_0 \] 
By rules \textsc{(T-New)} and \textsc{(T-Invk)}, 
  \[ 
     \mtype(m, K, K) = \overline{I_x} \rightarrow I_{e_0} \quad 
     \judgeewf \Gamma {\overline{e} : \overline{I_e'}} \quad
     \overline{I_e'} \subtype \overline{I_x} \quad
     \textit{, for some } \; \overline{I_e'}
  \]
By Lemma~\ref{lemma0},
    \[
    \judgeewf {\Gamma, \overline{x}:\overline{I_x}, \kwthis:J_0} {e_0:I_f} \textit{, for some } K \subtype J_0 \textit{ and } I_f \subtype I_{e_0}
    \]
By Lemma~\ref{lemma1},
    \[
    \judgeewf {\Gamma} {[\overline{\angl{I_x}} \overline{e}/\overline{x} ] e_0  :  I_g} \textit{, for some } I_g \subtype I_f 
    \]
So $I_g <: I_{e_0}$, finally just let $I' = I_g$.

\end{comment}
\end{proof}

\subsubsection{Proof for Theorem~\ref{theorem_progress}}
\begin{proof}
	
Since $e$ is well-typed, by \textsc{(T-Invk)} and \textsc{(T-Anno)} we know that
$$I \subtype J\textrm{, and }\mbody(m, J, J) \textrm{ is defined}$$

By \textsc{(T-Intf)}, $\mbody(m, I, J)$ is also defined, and the type checker ensures the expected number of arguments.

On the other hand, since $I \subtype J$, by the definition of $\mostSpecific$, $$\mostSpecific(m, I, J) \subseteq \mostSpecific(m, I, I)$$

By \textsc{(T-New)}, $\canInstantiate(I) = True$. By the definition of $\canInstantiate$, any $J_0\in\mostSpecific(m, I, I)$ satisfies that
$\mostSpecificOverride(m, I, J_0)$ contains only one interface, in which the $m$ that overrides $J_0$ is a concrete method. Therefore $\mbody(m, I, J)$ also provides a concrete method, which finishes the proof.
\end{proof}