\section{Overview}
\subsection{Motivation}
Combine the work of FHJ and ClasslessJava.
\begin{itemize}
	\item Use abstract state operations to mimic state
	\item If two fields conflict from triangle inheritance, keep the two fields
	\item Override and covariant type refine of fields
\end{itemize}

Example of unintentional fields confliction:
\begin{lstlisting}
interface Payment {
	String Check(); // represnets an entity 'check'
}
interface Verify {
	boolean Check(); // represents a flag of whether the verification is checked or not.
}
interface VerifiedPayment extends Payment, Verify {}
\end{lstlisting}

C++ partly support this, just like unintentional method conflicts. However, c++ cannot handle the case where fields need to be refined or two conflicted fields need to be merged.
\begin{lstlisting}
class A {};
class B:A {};
class Payment {
	public:
	A* check;
};
class Verify {
	public:
	bool check;
};
class VerifiedPayment : public Payment, public Verify {
	public:
	B* check;
};

int main()
{
VerifiedPayment* vp = new VerifiedPayment();
vp->check = new B();
vp->Verify::check = true;
vp->check;
cout << vp->Payment::check << endl;
cout << vp->check << endl;
}
\end{lstlisting}


Example program in a desired language:
\begin{lstlisting}
interface A { int x(); int y(); }
interface B { int x(); }
interface C extends A, B {}
new B(B.x = 0).GET_x();   // 0
(A)new C(A.x = 1, A.y = 2, B.x = 3).GET_x();  // 1
\end{lstlisting}

\section{Formalization}~\label{sec:formalization}

\begin{figure*}
	\saveSpaceFig
	\begin{displaymath}
		\begin{array}{l}
			\begin{array}{llrl}
				\text{Program}		& P & \Coloneqq  & \overline{I} \ e \\
				\text{Interfaces}   & IL & \Coloneqq & \interface{I}{I}{M} \\
				\text{Methods}      & M  & \Coloneqq & \method{I}{m}{I_x}{x}{J}{e}  \mid
				\absmethod{I}{m}{I_x}{x}{J} \\
				\text{Expressions}  & e  & \Coloneqq & 
				x \mid
				e.m(\overline{e}) \mid
				(I)e \mid 
				\neww{I}(\overline{J.f=e}) \\
				\text{Context}      & \Gamma & \Coloneqq & \overline{x}:\overline{I} \\
				\text{Values}       & v & \Coloneqq & (J) \neww{I}(\overline{J.f=v}) \\
			\end{array}
		\end{array}
	\end{displaymath}
	\caption{Syntax}\label{fig:syntax}
	\saveSpaceFig
\end{figure*}

%\begin{figure*}[]
%	\saveSpaceFig
%	\begin{mathpar}
%		\framebox{$ I <: J $} \hspace{.5in} \subid \\
%		\subtrans \hspace{.5in} \subextends \\
%		
%		\framebox{$ \judgeewf \Gamma {e:I} $} \hspace{.5in}
%		\tvar \\
%		\tinvk \\
%		% \tpathinvk \\
%		% \tsuperinvk \\
%		% \tstaticinvk  \\
%		\tnew \\
%		\tanno \\
%		\tmethod \\
%		\tabsmethod \\
%		\tintf
%	\end{mathpar}
%	\saveSpaceFig
%	\caption{Subtyping and Typing Rules}
%		\label{fig:typingrules}
%	\end{figure*}
	

\begin{figure}
	\begin{mathpar}
		%%%%%%**** S-Get ****%%%%%%%
		\inferrule* [left=(S-Get)]
		{ calField(I, J, \overline{K.f=e}, f_i) = e }
		{ (J)new \ I(\overline{K.f=e}).GET\_f_i() \to e } \\
		
		%%%%%%**** S-Get ****%%%%%%%
		\inferrule* [left=(S-Set)]
		{ updateField(I, J, \overline{K.f=e}, f_i, e) = (J)new \  I(\overline{K.f=e'}) }
		{ (J)new I(\overline{K.f=e}).SET\_f_i(e) \to (J)new \  I(\overline{K.f=e'}) } \\
		
		%%%%%%**** S-Invk ****%%%%%%%
		\text{\yanlin{(S-Invk) is similar to FHJ} } \\
		\inferrule* [left=(S-Invk)]
		{ \mbody(m, I, J) = (I_0, \overline{I_x} \; \overline{x}, I_e \; e_0) }
		{ \left((J)\neww{I}(\overline{K.f=v})\right).m(\overline{d}) \to
				(I_e)[
				\overline{(I_x)d}
				/\overline{x}, (I_0)\neww{I}(\overline{K.f=v})/\kwthis]e_0 }
	\end{mathpar}
	\caption{Reduction Rules.}
\end{figure}
	
%	\begin{figure*}[]
%		\saveSpaceFig
%		\begin{mathpar}
%			% \sinvk \\
%			\spathinvk \\
%			% \ssuperinvk \\
%			% \sstaticinvk \\
%			\creceiver \hspace{.5in}
%			% \cpathreceiver \\
%			\cargs \\
%			% \cpathargs \\
%			% \csuperargs \\
%			% \cstaticargs \\
%			\cstatictype \\
%			\cfreduce \\
%			\cannoreduce
%		\end{mathpar}
%		\caption{Small-step semantics.}\label{fig:smallstep}
%		\saveSpaceFig
%	\end{figure*}


\section{Aux}
\subsection{calField}
$ calField(I, J, \overline{K.f=e}, f_i)$
\begin{itemize}
	\item if $I.f_i \in \overline{K.f}$, return $e_i$ where $I.f_i = e_i$
	\item if $J.f_i \in \overline{K.f}$, return $e_i$ where $J.f_i = e_i$ 
\end{itemize}

\subsection{updateField}
similar as $calField$.