\section{Related Work}~\label{sec:relatedwork}
We describe related work in four parts. We first discuss mainstream popular multiple 
inheritance models and then some specific models (e.g., C++ and \csharp) which are closest to our work. Then we 
discuss related techniques used in \self. Finally, we discuss the foundation and related work of our formalization.

\subsection{Mainstream Multiple Inheritance Models}
Multiple inheritance is a useful feature in object-oriented
programming, although it is difficult to model and can 
cause various problems (e.g. the diamond problem~\cite{Sak89dis,Singh1995}).  
There are many existing languages/models that support multiple 
inheritance~\cite{ellis1990annotated,scala-overview,bracha90mixin,scharli03traits,malayeri2009cz,csharpdoc,Moon1986,Flatt1998,Ancona2003}. 
The mixin models~\cite{bracha90mixin,Flatt1998,van1996encapsulation,Ancona2003,Hendler86} allow naming components 
that can be applied to various classes as reusable functionality units. However, the linearization (total ordering) of mixin 
inheritance cannot provide a satisfactory resolution in some cases and restricts the flexibility of mixin composition. 
Scala traits~\cite{scala-overview} are in fact linearized mixins and hence have the same problem as mixins.

Simplifying the mixins approach, traits~\cite{scharli03traits,Ducasse:2006:TMF:1119479.1119483} draw a
strong line between units of reuse and object factories. Traits act
as units of reuse, containing functionality code. Classes, on the
other hand, are
assembled from traits and act as object factories. Java 8
interfaces are closely related to traits: concrete method
implementations are allowed (via the \textbf{\texttt{default}}
keyword) inside interfaces, thus allowing for a restricted form 
of multiple inheritance.
There are also proposals such as FeatherTrait Java~\cite{Liquori08ftj} 
for extending Java with traits. Extensions~\cite{reppy2006foundation,Reppy:2007:MT:2394758.2394784} to 
the original trait model exists with various advanced features, such as \emph{renaming}. As discussed in Section~\ref{sec:overview},
the renaming feature gives a workaround to the unintentional method conflicts
problem. However, it breaks structural subtyping.

\begin{comment}
There are also proposals for extending Java with traits. For example, 
FeatherTrait Java (FTJ) [14] extends FJ [13] with statically-typed traits, 
adding trait-based inheritance in Java. Except for few, mostly syntac- tic details, 
their work can be emulated with Java 8 interfaces. There are also extensions 
to the original trait model, with operations (e.g. renaming [18], which breaks 
structural sub- typing) that default methods and interfaces cannot model.
\end{comment}

Malayeri and Aldrich proposed a model CZ~\cite{malayeri2009cz} which
aims to do multiple inheritance without the diamond problem.
Inheritance is divided into two concepts: inheritance dependency and
implementation inheritance.  Using a combination of
\textbf{\texttt{requires}} and \textbf{\texttt{extends}}, a program
with diamond inheritance is transformed to one without
diamonds. Moreover, fields and multiple inheritance can coexist.
However untangling inheritance also untangles the class structure. In
CZ, not only the number of classes but also the class hierarchy
complexity increases.

The above-mentioned models/languages support multiple inheritance, 
focusing on diamond inheritance.
They handle method conflicts in the same way, by simply disallowing
two methods with the same signature from two different units to
coexist. In contrast, our work provides mechanisms that allow 
methods with the same signatures, but different parents to coexist 
in a class. Disambiguation is possible in many cases by using both 
static and dynamic type information during method dispatching.
In the cases where real ambiguity exists, \MIM{}'s type system can
reject interface definitions and/or method calls statically.

\subsection{Resolving Unintentional Method Conflicts}\label{subsec:middleman}
A few language implementations have realized the problem of
unintentional conflicts and provide some support for it.

\begin{comment} %C++ related work with code explaination
\noindent {\bf C++ model.}
C++ supports a very flexible inheritance model and allows programmers to choose either static dispatch or dynamic dispatch for method lookup.
It allows unintentional conflicts and uses static dispatch to resolve them, as discussed in Section~\ref{sec:overview}. For example, given the following code
\begin{lstlisting}[language=cpp]
class A { public: void m() {cout << "MA" << endl;} };
class B { public: void m() {cout << "MB" << endl;} };
class C : public A, public B { 
	void m() {cout << "MC" << endl;}
};
void func(A* a) { a->m(); }
int main() {
	C* c = new C();
	c->B::m();
	func(c); 
	return 0; //Running result: MB MA
}
\end{lstlisting}
The running result is $MB \; MA$, meaning that it uses static dispatch (looks at the static types) for method \lstinline|m()|. 
On calling \texttt{func(c)}, in spite that the dynamic type of \texttt{c} is class \texttt{C}, the method call still dispatches to 
\lstinline|A.m|.
However, we can alter the code a little bit with \textbf{\texttt{virtual}} methods and the result will be totally different:
\begin{lstlisting}[language=c++]
class A { public: virtual void m() {cout << "MA" << endl;} };
class B { public: virtual void m() {cout << "MB" << endl;} };
class C : public A, public B { 
    public: virtual void m() {cout << "MC" << endl;}
};
void func(A* a) { a->m(); }
int main() {
	C* c = new C();
	c->B::m();
	func(c); 
	return 0; //Running result: MB MC
}
\end{lstlisting}
Now the running result will be $MB \; MC$. With virtual methods, dynamic dispatch is used and 
method lookup algorithm will find the most specific method definition of $m$, namely \lstinline|C.m| at this time.
Although C++ support this flexibility, dynamic dispatching on
unintentional conflicting methods is problematic, as discussed in
Section~\ref{sec:overview}.\bruno{where in the overview are we
  discussing this?}
Furthermore, compared to our
model, C++ does not support hierarchical overriding.
\end{comment}

\noindent {\bf C++ model.}
C++ supports a very flexible inheritance model.
C++ allows the existence of unintentional conflicts and users may specify a hierarchical path via casts for disambiguation, as discussed in Section~\ref{sec:overview}. 
With virtual methods, dynamic dispatch is used and the 
method lookup algorithm will find the most specific method definition. 
A contribution of our work is to provide a minimal formal model of
hierarchical dispatching, whereas C++ can be viewed as a real-world
implementation. There are several
formalizations~\cite{Wasserrab2006,ramananandro2012mechanized,Ramalingam1997}
in the literature modeling various C++ features. However, as far as we know, there is no formal model that
captures this aspect of the C++ method dispatching model. Apart from this, as discussed in Section~\ref{subsec:loosen}, \name{} conservatively rejects some interface/class definitions that C++ accepts, and upcasts are never rejected since they are used for ambiguity resolution.
\bruno{The discussion points out further differences. We need to refer
to that discussion here too.}\yanlin{added.}

Although C++ supports hierarchical dispatching, it does not support
hierarchical overriding.  However, there are some possible
workarounds that can mimick hierarchical
overriding, including the \emph{MiddleMan} approach\footnote{\url{https://stackoverflow.com/questions/44632250/can-i-do-mimic-things-likes-this-partial-override-in-c}}, the \emph{interface classes} pattern as described in Section 25.6 of~\cite{stroustrup1995}, the \emph{LotterySimulation} discussion in~\cite{stroustrup1994design}. Since these workarounds share the same spirit, we will discuss in detail the \emph{MiddleMan} approach, with code shown in 
Figure~\ref{fig:middleman}. In this example, classes \texttt{A} and
\texttt{B} are two classes that both define a method with the same
name \texttt{m}
unintentionally.

Class \texttt{MiddleMan}, as its name suggests, acts as a middle man between its class \texttt{C} and its parents \texttt{A, B}. \texttt{MiddleMan} defines a virtual method \texttt{m} that overrides a parent method \texttt{m} and delegates the implementation to another method \texttt{m\_impl} that takes $\kwthis$ as a parameter. C++ supports method overloading, so that multiple \texttt{m\_impl} methods with different parameter types can coexist. When defining class \texttt{C}, we specify the parents to be \texttt{MiddleMan<A>, MiddleMan<B>} instead of \texttt{A, B}. In this way, programmers may define new versions of \texttt{A.m} and \texttt{B.m} in class \texttt{C} by providing the corresponding \texttt{m\_impl} methods. Then in the client code, the method call \texttt{((A*)c)->m()} will print out string \texttt{"MA2"}, as expected. Although this workaround can help us defining partial method overrides to a certain extent, the drawbacks are obvious. Firstly, the approach is complex and requires the pre-knowledge of the programmer to this approach. Moreover, the lack of direct syntax support makes MiddleMan code cumbersome to write. Finally, the approach is ad-hoc, meaning that the class \texttt{MiddleMan} shown in Figure~\ref{fig:middleman} is not general enough to be used in other cases: more middle-mans are needed if partial method overrides happens in other classes; and it is even worse when return types differ. 

\begin{figure*}[t]
\saveSpaceFig
\begin{lstlisting}
class A { public: virtual void m() {cout << "MA" << endl;}};
class B { public: virtual void m() {cout << "MB" << endl;}};
template<class C>
class MiddleMan : public C {
    void m() override final { m_impl(this); }
  protected:
    virtual void m_impl(MiddleMan*) { return this->C::m(); }
};
class C : public MiddleMan<A>, public MiddleMan<B> { 
private:
    void m_impl (MiddleMan<A>*) override {cout << "MA2" << endl;}
    void m_impl (MiddleMan<B>*) override {cout << "MB2" << endl;}
};
int main()
{
    C* c = new C();
    ((A*)c)->m();         //print "MA2"
    return 0;
}
\end{lstlisting}
\caption{The \emph{MiddleMan} approach.}
\label{fig:middleman}
\saveSpaceFig
\end{figure*}

\noindent  {\bf \csharp{} explicit method implementations.}
Explicit method implementations is a special feature supported by
\csharp{}. As described in \csharp{} documentation~\cite{csharpdoc}, a
class that implements an interface can explicitly implement a member
of that interface. When a member is explicitly implemented, it can
only be accessed through an instance of the interface. Explicit
interface implementations allow an interface to inherit multiple 
interfaces that share the same member names and give each interface
member a separate implementation. 

Explicit interface member implementations have two advantages.
Firstly, they allow interface implementations to be excluded 
from the public interface of a class. This is particularly useful when a class implements an internal 
interface that is of no interest to a consumer of that class or struct.
Secondly, they allow disambiguation of interface members with the 
same signature. However, there are two critical differences to \MIM{}:
(1) default method implementations are not allowed in \csharp{} interfaces; 
(2) there is only one level of conflicting method implementations at the
class that implements the multiple parent interfaces. Further
overriding of those methods is not possible in subclasses.

\noindent  {\bf Languages using hygienicity.}
In NextGen/MixGen~\cite{Allen:2003:FAG:949305.949316}, HygJava~\cite{kusmierek2007hygienic} and Magda~\cite{Bono:2012:MNL:2367163.2367198}, \emph{hygienicity} is proposed to deal with unintentional method conflicts, which is basically giving a method a unique identifier by prefixing the name with an unambiguous path. The advantage of this approach, compared to ours, is that it does not require additional notion in the method dispatching. However, the disadvantage is that these approaches depend on name appending, where conflicting methods from different branches cannot be merged because they are regarded as unrelated method, since they have different paths and the path names are appended in method names. Moreover, writing programs in these languages is tedious if not supported by an ad-hoc IDE and the path prefix is mandatory even if it is not needed in some scenarios.

\subsection{Hierarchical Dispatch in \self{}}
As we have discussed before, although the mix of static and dynamic dispatch is 
particularly useful under certain circumstances, it has received little research attention. 
In the prototype-based language \self{}~\cite{Chambers1991}, inheritance is a basic feature.
\self{} does not include classes but instead allows individual objects to inherit from (or delegate to) other objects. 
Although it is different from class-based languages, the multiple inheritance model is somehow similar. The \self{}
language supports multiple (object) inheritance in a clever way. It not only develops the new inheritance
relation with \emph{prioritized parents} but also adopts \emph{sender path tiebreaker rule} for method lookup.
In \self{} ``\emph{if two slots with the same name are defined in equal-priority parents of the receiver, but only 
one of the parents is an ancestor or descendant of the object containing the method that is sending the message,
then that parent's slot takes precedence over the over parent's slot}''. 
Similarly to our model, this sender path tiebreaker rule resolves ambiguities between unrelated slots. However,
it is used in a prototype-based language setting and it does not support method hierarchical overriding as \MIM{} does.

\subsection{Formalization Based on Featherweight Java}
Featherweight Java (FJ)~\cite{Igarashi01FJ} is a minimal core calculus of the Java language, 
proposed by Igarashi et. al. There are many models built on Featherweight Java, 
including FeatherTrait~\cite{Liquori08ftj}, Featherweight defenders~\cite{goetz12fdefenders}, Jx~\cite{Nystrom2004}, 
Featherweight Scala~\cite{Cremet2006}, and so on.
FJ provides the standard model of formalizing Java-like object-oriented languages and 
is easily extensible. In terms of formalization, the key novelty of our model is making use of various types (such as parameter types, method return types, etc) as upcasts along with various terms. As far as we know, this technique has not appeared in the literature before. This notion is of vital importance in our hierarchical dispatch algorithm, and it allows for a
more precise subject-reduction theorem as discussed in
Section~\ref{sec:formalization}.








