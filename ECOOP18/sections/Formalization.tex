\section{Formalization}~\label{sec:formalization}
In this section, we present a formal model called \MIM{} (\emph{\textbf{F}eatherweight \textbf{H}ierarchical \textbf{J}ava}), following a similar style as  
Featherweight Java~\cite{Igarashi01FJ}. \MIM{} is a minimal core calculus that formalizes the core concept of hierarchical dispatching and overriding. The syntax, typing rules and small-step semantics are presented.

% \vspace{-2ex}
\subsection{Syntax}
The abstract syntax of \MIM{} interface declarations, method declarations, and expressions is given in Figure~\ref{fig:syntax}. The multiple
inheritance feature of \MIM{} is inspired by Java 8 interfaces, which supports
method implementations via default methods. This feature is 
closely related to \emph{traits}. To demonstrate how
unintentional method conflicts are untangled in \MIM{}, we only focus on a small subset of the interface model. For example, all methods declared
in an interface are either default methods or abstract methods. Default methods provide default implementations for methods. Abstract methods do not
have a method body. Abstract methods can be overridden with future implementations.

\subsubsection{Notations}
The metavariables $I, J$ range over interface names; $x$ ranges over variables; $m$ ranges over method names; $e$ ranges over expressions; and $M$ ranges over method declarations. Following Featherweight Java, we assume that the set of variables includes the special variable \kwthis, which cannot be used as the name of an argument to a method. We use the same
conventions as FJ; we write $\overline{I}$ as shorthand for a possibly empty sequence $I_1, ..., I_n$, which may be indexed by $I_i$; and write $\overline{M}$ as shorthand for $M_1 .. M_n$ (with no commas). We also abbreviate operations on pairs of sequences in an obvious way, writing $\overline{I} \; \overline{x}$ for $I_1 \; x_1, ..., I_n \; x_n$, where $n$ is the length of $\overline{I}$ and $\overline{x}$.

\subsubsection{Interfaces}
In order to achieve multiple inheritance, an interface can have a set of 
parent interfaces, where such a set can be empty. The interface declaration $\interface{I}{I}{M}$ introduces an interface named $I$ with parent interfaces $\overline{I}$ and a suite of methods $\overline{M}$. The methods of $I$ may either override methods that are already defined in $\overline{I}$ or add new functionality special to $I$, we will illustrate this in more detail later.

\subsubsection{Methods}
Original methods and hierarchically overriding methods share the same syntax in our model for simplicity.
The concrete method declaration $\method{I}{m}{I_x}{x}{J}{e}$ introduces a
method named $m$ with result type $I$, parameters $\overline{x}$ of
type $\overline{I_x}$ and the overriding target $J$. The body of the
method simply includes the returned expression $e$. Notably, we have introduced the
\kwoverride{} keyword for two cases: if the overridden interface is exactly the enclosing
interface itself, then such a method is seen as originally defined; otherwise it is a hierarchical overriding method. 
Note that in an interface $J$, $
I \; m(\overline{I_x} \; \overline{x}) \; {\{} \kwreturn \; e ; {\}} $ is syntactic sugar for $\method{I}{m}{I_x}{x}{J}{e}$, which is the standard way of method definition in Java-like languages. The definition
of abstract methods is written as $\absmethod{I}{m}{I_x}{x}{J}$, which is
similar to a concrete method but without the method body. 
For simplicity, overloading is not modelled for methods, which
implies that we can uniquely identify a method by its name.

\subsubsection{Expressions \& Values}
Expressions can be standard constructs such as variables, method
invocation, object creation, together with cast expressions. 
Object creation is represented by $\new I$\footnote{In Java the corresponding syntax is $\new I\{\}$.}. Fields and primitive types are not modelled in \MIM{}. 
The casts are merely safe upcasts, and in fact, they can be viewed as
annotated expressions, where the annotation indicates its static type.
The coexistence of static and dynamic types is the key to hierarchical dispatch.
A value
``$(I)\new{J}$''
is the final result of multiple reduction steps evaluating an
expression.

For simplicity, \name{} does not formalize statements like assignments and so on because they are orthogonal features to the hierarchical dispatching and overriding feature.
A program in \name{} consists of a list of interface declarations, plus a single expression.

\begin{figure*}[t]
\saveSpaceFig
\begin{displaymath}
\begin{array}{l}
\begin{array}{llrl}
\text{Interfaces}   & IL & \Coloneqq & \interface{I}{I}{M} \\
\text{Methods}      & M  & \Coloneqq & \method{I}{m}{I_x}{x}{J}{e}  \mid
									   \absmethod{I}{m}{I_x}{x}{J} \\
\text{Expressions}  & e  & \Coloneqq & x \mid
e.m(\overline{e}) \mid
\new{I} \mid \; (I)e \\
\text{Context}      & \Gamma & \Coloneqq & \overline{x}:\overline{I} \\
\text{Values}       & v & \Coloneqq & (I) \new{J} \\
%%\\
%%\text{Interface names} & I, J, K & & \\
%%\text{Method names} & m & & \\
%%\text{Variable names} & x & &
\end{array}
\end{array}
\end{displaymath}
\caption{Syntax of \name{}.}\label{fig:syntax}
\saveSpaceFig
\end{figure*}


\begin{figure*}[t]
\saveSpaceFig
\begin{mathpar}
	\framebox{$ I <: J $} \hspace{.5in} \subid \\
	\subtrans \hspace{.5in} \subextends \\
	
	\framebox{$ \judgeewf \Gamma {e:I} $} \hspace{.5in}
	\tvar \\
	\tinvk \\
	% \tpathinvk \\
	% \tsuperinvk \\
	% \tstaticinvk  \\
	\tnew \\
	\tanno \\
	\tmethod \\
	\tabsmethod \\
	\tintf
\end{mathpar}
\saveSpaceFig
\caption{Subtyping and Typing Rules of \name{}.}
\label{fig:typingrules}
\end{figure*}

\subsection{Subtyping and Typing Rules}\label{subsec:typingrules}
\subsubsection{Subtyping}
The subtyping of \MIM{} consists of only a few rules shown at the top of Figure~\ref{fig:typingrules}.
In short, subtyping relations are built from the inheritance in interface
declarations. Subtyping is both reflexive and transitive.

\subsubsection{Type-checking}
Details of type-checking rules are displayed at the bottom of Figure~\ref{fig:typingrules}, including expression
typing, well-formedness of methods and interfaces. As a convention, an environment
$\Gamma$ is maintained to store the types of variables, together with
the self-reference $\kwthis$.
% The three rules for method invocation, \textsc{(T-Invk)}, \textsc{(T-PathInvk)} and \textsc{(T-SuperInvk)}
% are very similar, in the sense that they all check the type of the specific method, by using
% an auxiliary function \mtype. \mtype{} is the function for looking up method types, which we will
% illustrate later in Section~\ref{subsec:auxdefs}. After the method
% type is obtained, they all check that the arguments and the receiver
% have compatible types. Additionally, \textsc{(T-PathInvk)} requires the receiver to be the subtype of the specified
% path type, and \textsc{(T-SuperInvk)} checks if the enclosing type directly extends the specified super type.

\textsc{(T-Invk)} is the typing rule for method invocation.
Naturally, the receiver and the arguments are required to be well-typed.
$\mbody$ is our key function for method lookup that implements the
hierarchical dispatching algorithm. The formal definition will be introduced in Section~\ref{sec:auxdefs}.
Here $\mbody(m, I_0, I_0)$ finds the most specific $m$ above $I_0$. ``Above $I_0$'' specifies
the search space, namely the supertypes of $I_0$ including itself.
For the general case, however, the hierarchical invocation $\mbody(m, I, J)$ finds ``the most specific $m$
above $I$ and along path/branch $J$''. ``Along path $J$'' additionally requires the result to relate to $J$, that is to say,
the most specific interface that has a subtyping relationship with $J$.

In \textsc{(T-Invk)}, as the compilation should not be aware
of the dynamic type, it only requires that invoking $m$ is valid for the static type of the
receiver. The result of $\mbody$ contains the interface that provides the most specific implementation,
the parameters and the return type. We use underscore for the return expression, implying that an empty return expression
from an abstract method is acceptable.
% \textsc{(T-PathInvk)} is the typing judgement for a path invocation. Besides the conditions of \textsc{(T-Invk)}, \textsc{(T-PathInvk)} requires the type of receiver to be the subtype of the specified path type. 
% and \textsc{(T-SuperInvk)} checks if the enclosing type directly extends the specified super type.

\textsc{(T-New)} is the typing rule for object creation $\new{I}$. The
auxiliary function $\canInstantiate(I)$ (see definition in Section~\ref{sec:otherdefs}) checks whether an interface $I$ 
can be instantiated or not. Since \wordfork{} inheritance accepts conflicting branches to coexist, the check requires that the most specific method is concrete for each method on each branch.

\textsc{(T-Method)} is more interesting since a method can either be an original method or a hierarchical overriding, though
they share the same syntax and method typing rule. $\mostSpecific(m, I, J)$ is a fundamental function,
used to find ``the most specific interfaces that are above $I$ and
along path $J$, and originally defines $m$'' (see
Section~\ref{sec:auxdefs} for full definition).
By ``most specific interfaces'',
it implies that the inherited supertypes are excluded. Thus the condition $\mostSpecific(m, I, J) = \{J\}$ indicates a characteristic of a hierarchical overriding: it must override an original method; the overriding is direct and there does not exist any other original method $m$ in between.
Then $\mbody(m, J, J)$ provides the type of the original method, so hierarchical overriding has to preserve the type. Finally the return expression
is type-checked to be subtype of the declared return type. For the definition of an original method, $I$ equals $J$ and the rule is straightforward. \textsc{(T-AbsMethod)} is a similar rule but works on abstract method declarations.

\textsc{(T-Intf)} defines the typing rule on interfaces. The first condition is obvious, namely its methods need to be well checked. The third
condition checks whether the overriding between original methods preserves typing. In this condition we again use some helper functions defined in  Section~\ref{sec:auxdefs}. $I[m\ \kwoverride\ I]$ is defined if $I$ originally defines $m$, and $\canOverride(m, I, J)$ checks whether $I.m$ has the same type as $J.m$. Generally the preservation of method type is required for any supertype $J$ and any method $m$.

The second condition of \textsc{(T-Intf)} is more complex and is the key to type soundness. Unlike C++ which rejects on ambiguous calls,
\MIM{} rejects on the definition of interfaces when they form a diamond. Consider the case when the second condition is broken: $\mbody(m, J, J)$
is defined but $\mbody(m, I, J)$ is undefined for some $J$ and $m$. This indicates that $m$ is available and unambiguous from the perspective of $J$,
but is ambiguous to $I$ on branch $J$. It means that there are multiple overriding paths of $m$ from $J$ to $I$, which form a diamond. Hence rejecting
that case meets our expectation. Below is an example (Figure~\ref{fig:examplesmbody} (e)) that illustrates the reason why this condition is needed:
%\bruno{what is the purpose of this example:
%  state-it upfront please. Is this example meant to ilustrate T-Inf?
%  Then it's better to have the example together with the text
%  explaining T-inf.} \yanlin{revised.please check whether you're happy with it.}
\begin{lstlisting}
interface T                 { T m() override T { return new T(); } }
interface A extends T       { T m() override T { return new A(); } }
interface B extends T       { T m() override T { return new B(); } }
interface C extends A, B {}
((T) new C()).m()
\end{lstlisting}
This program does not compile on interface $C$, because of the second condition in \textsc{(T-Intf)}, where $I$ equals $C$ and $J$ equals $T$.
By the algorithm, $\mbody(m, T, T)$ will refer to $T.m$, but $\mbody(m, C, T)$ is undefined, since both $A.m$ and $B.m$ are most specific
to $C$ along path $T$, which forms a diamond. The expression \lstinline|((T) new C()).m()| is one example of triggering ambiguity, but \MIM{}
simply rejects the definition of $C$. To resolve the issue, the programmer needs to have an overriding method in $C$, to explicitly merge
the conflicting ones.

Finally, rule \textsc{(T-Anno)} is the typing rule for a cast expression. By the rule, only upcasts are valid.

\subsection{Small-step Semantics and Congruence}
Figure~\ref{fig:smallstep} defines the small-step semantics and
congruence rules of \MIM{}. When evaluating an expression, they
are invoked and produce a single value in
the end. %\haoyuan{we need to be consistent on paragraph upper/lower case.}

\subsubsection{Semantic Rules} \textsc{(S-Invk)} is the only computation rule we need for method invocation.
As a small-step rule and by congruence, it assumes that the receiver and the arguments are already values.
Specifically, the receiver $(J)\new{I}$ indicates the dynamic type $I$
together with the static type $J$. Therefore $\mbody(m, I, J)$ carries out hierarchical dispatching, acquires
the types, the return expression $e_0$ and the interface $I_0$ which provides the most specific method.
Here we use $e_0$ to imply that the return expression is forced to be non-empty because it requires a concrete implementation. Now the
rule reduces method invocation to $e_0$ with substitution.
Parameters are substituted with arguments, and the \lstinline|this| reference is substituted with the receiver,
and in the meanwhile the static types are recorded via annotations. Finally, the return type $I_e$ is put in the front as an annotation.
\subsubsection{Congruence Rules} \textsc{(C-Receiver)}, \textsc{(C-Args)} and \textsc{(C-FReduce)} are natural congruence rules
on receivers, arguments, and cast-expressions, respectively. \textsc{(C-StaticType)} automatically adds an annotation $I$ to the new
object $\new{I}$. \textsc{(C-AnnoReduce)} merges nested upcasts into a single upcast with the outermost type.



\begin{comment}
\paragraph{Example} In contrast with the counter-example in Section~\ref{subsec:typingrules}, it is better to understand semantics by
well-compiled examples. Here we abstract a variant of the \lstinline|DrawableDeck| example:

\vspace{3pt}\begin{lstlisting}
interface Void       {}
interface JFrame     {}
interface Deck       { Void draw() override Deck { return new Void(); } }
interface Drawable { JFrame draw() override Drawable; }
interface DrawableDeck extends Drawable, Deck {
  JFrame draw() override Drawable {
    return new JFrame();
  }
}

((Drawable) new DrawableDeck()).draw()
\end{lstlisting}\vspace{3pt}
We put \lstinline|Drawable.draw| as an abstract method instead, but hierarchically override it in \lstinline|DrawableDeck|.
By typing rules, the code is well-compiled. And during runtime,
\begin{align*}
	& ((Drawable) new DrawableDeck()).draw() \\
\rightarrow & (JFrame) new JFrame()
\end{align*}
\end{comment}