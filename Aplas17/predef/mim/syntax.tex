% \newcommand{\keyword}[1]{\texttt{#1}}
% \newcommand{\subtype}   {<:}
%%%%%%%%%%%%%%%%%%%%%%%%%%%%%%%%%%%%%%%%%% syntax.tex begin %%%%%%%%%%%%%%%%%%%%%%
\newcommand{\kwinterface}{\keyword{interface}}
\newcommand{\kwextends}{\keyword{extends}}
\newcommand{\kwreturn}{\keyword{return}}
\newcommand{\kwoverride}{\keyword{override}}
\newcommand{\kwsuper}{\keyword{super}}
\newcommand{\kwthis}{\keyword{this}}
\newcommand{\kwnew}{\keyword{new}}
\newcommand{\kwtrue}{\keyword{true}}
\newcommand{\kwfalse}{\keyword{false}}


\newcommand{\mtype}{\keyword{mtype}}
\newcommand{\ext}{\keyword{ext}}
\newcommand{\definedin}{\keyword{definedin}}
\newcommand{\collectMethods}{\keyword{collectMethods}}
\newcommand{\mbody}{\keyword{mbody}}
\newcommand{\Undefined}{\keyword{Undefined}}
\newcommand{\Error}{\keyword{Error}}
\newcommand{\needed}{\keyword{needed}}
\newcommand{\methods}{\keyword{methods}}
\newcommand{\only}{\keyword{only}}
\newcommand{\pathcheck}{\keyword{pathcheck}}
\newcommand{\mostSpecific}{\keyword{mostSpecific}}
\newcommand{\mostSpecificOverride}{\keyword{mostSpecificOverride}}
\newcommand{\overrideSet}{\keyword{overrideSet}}
\newcommand{\prune}{\keyword{prune}}
\newcommand{\update}{\keyword{update}}
\newcommand{\dom}{\keyword{dom}}

\newcommand{\canInstantiate}{\keyword{canInstantiate}}

\newcommand{\angl}[1]{
    \langle #1 \rangle
}

\newcommand{\num}[1]{
    \#(#1)
}

\newcommand{\new}[1]{
    \kwnew \; #1()
}

\newcommand{\interface}[3]{
  \kwinterface \; #1 \; \kwextends \; \overline{#2} \; {\{} \overline{#3} {\}}
}
\newcommand{\method}[6]{
  #1 \; #2 (\overline{#3} \; \overline{#4}) \; \kwoverride \; #5 \; {\{} \kwreturn \; #6 ; {\}}
}
\newcommand{\absmethod}[5]{
  #1 \; #2 (\overline{#3} \; \overline{#4}) \; \kwoverride \; #5 \; ;
}

\newcommand{\subid} {
\inferrule* [right=]
    {}
    {I \subtype I}
}
\newcommand{\subtrans} {
\inferrule* [left=]
    {I \subtype J \\ J \subtype K}
    {I \subtype K}
}

\newcommand{\subextends} {
\inferrule* [right=]
    {\interface{I}{I}{M}}
    {\forall I_i \in \overline{I}, I \subtype I_i}
}

\newcommand{\tvar} {
\inferrule* [left=(T-Var)]
    {}
    {\judgeewf \Gamma x:\Gamma(x)}
}

\newcommand{\tinvk} {
\inferrule* [left=(T-Invk)]
    {  \judgeewf \Gamma {e_0:I_0}
    \\ \mtype(m, I_0) = \overline{J} \to I
    \\ \judgeewf \Gamma \overline{e}:\overline{I}
    \\ \overline{I} \subtype \overline{J}
    }
    {\judgeewf \Gamma e_0.m(\overline{e}):I}
}

\newcommand{\tpathinvk} {
\inferrule* [left=(T-PathInvk)]
    {  \judgeewf \Gamma {e_0:I_0}
    \\ I_0 \subtype J_0
    \\ \mtype(m, J_0) = \overline{J} \to I
    \\ \judgeewf \Gamma \overline{e}:\overline{I}
    \\ \overline{I} \subtype \overline{J}
    }
    {\judgeewf \Gamma e_0.J_0::m(\overline{e}):I}
}

\newcommand{\tsuperinvk} {
\inferrule* [left=(T-SuperInvk)]
    {  \ext(\Gamma(this), J_0)
    \\ \mtype(m, J_0) = \overline{J} \to I
    \\ \judgeewf \Gamma \overline{e}:\overline{I}
    \\ \overline{I} \subtype \overline{J}
    }
    {\judgeewf \Gamma {\kwsuper.J_0::m(\overline{e})}:I}
}

\newcommand{\tnew} {
\inferrule* [left=(T-New)]
    {\interface{I}{I}{M} \\ \canInstantiate(I) } 
    {\judgeewf \Gamma \new{I}:I}
}

\newcommand{\tmethod} {
\inferrule* [left=(T-Method)]
    {  I <: J \\
    \mostSpecific(m, I, J) = \{J\} \\
    \mtype(m, J) = \overline{I_x} \to I_e \\
        \judgeewf {\overline{x}:\overline{I_x}, this:I  } {e_0:I_0} \\
        I_0 \subtype I_e
    %\\ \definedin(m, J) % m is (directly or indirectly) defined in J
    }
    {\method{I_e}{m}{I_x}{x}{J}{e_0} \text{ OK IN } I}
}

\newcommand{\tabsmethod} {
\inferrule* [left=(T-AbsMethod)]
    {  I <: J \\
    \mostSpecific(m, I, J) = \{J\} \\
    \mtype(m, J) = \overline{I_x} \to I_e 
    %\\ \definedin(m, J) % m is (directly or indirectly) defined in J
    }
    {\absmethod{I_e}{m}{I_x}{x}{J} \text{ OK IN } I}
}

% \newcommand{\tintf} {
% \inferrule* [left=(T-Intf)]
%     {  \overline{I} \text{ OK}
%     \\ \forall m \in \collectMethods(I), \mbody(m, I) \neq \Undefined
%     }
%     { \interface{I}{I}{M} \text{ OK }}
% }
\newcommand{\tintf} {
\inferrule* [left=(T-Intf)]
    {  \overline{I} \text{ OK}
    \\ \overline{M} \text{ OK IN } I
    \\ \forall J >: I \text{ and } m, \mtype(m, J) \text{ is defined} \Rightarrow \mbody(m, I, J) \text{ is defined}
    }
    { \interface{I}{I}{M} \text{ OK }}
}


\newcommand{\sinvk} {
\inferrule* [left=(S-Invk)]
{ \mbody(m, I, J) = (I_0, \overline{I_x} \; \overline{x}, I'_e \; e_0) \\ e_0 \neq \o}
{\left(\angl{J}\new{I}\right).m(\overline{\angl{I_e}} \overline{e}) \to
    \angl{I'_e}[\overline{\angl{I_x}} \overline{e}/\overline{x}, \angl{I_0}\new{I}/\kwthis]e_0}
}

\newcommand{\spathinvk} {
\inferrule* [left=(S-PathInvk)]
{\mbody(m, I, K) = (I_0, \overline{I_x} \; \overline{x}, I'_e \; e_0) \\ e_0 \neq \o}
{\left(\angl{J}\new{I}\right).K::m(\overline{\angl{I_e}} \overline{e}) \to
    \angl{I'_e}[\overline{\angl{I_x}} \overline{e}/\overline{x}, \angl{I_0}\new{I}/\kwthis]e_0}
}

\newcommand{\ssuperinvk} {
\inferrule* [left=(S-SuperInvk)]
{\mbody(m, K, K) = (I_0, \overline{I_x} \; \overline{x}, I'_e \; e_0) \\ e_0 \neq \o}
{\kwsuper.K::m(\overline{\angl{I_e}} \overline{e}) \to
    \angl{I'_e}[\overline{\angl{I_x}} \overline{e}/\overline{x}, \angl{I_0}\new{K}/\kwthis]e_0}
}

\newcommand{\deff} {
\begin{displaymath}
    \begin{array}{l}
        \begin{array}{llrl}
        \text{Annotated Expressions}   & f & \Coloneqq & e \mid \angl{I}e
        \end{array}
    \end{array}
\end{displaymath}
}

\newcommand{\creceiver} {
%----------- v1.0 ---------------
%\inferrule* [left=(C-Receiver)]
%{e \to e'}
%{e.m(\overline{e}) \to e'.m(\overline{e})}
%}
\inferrule* [left=(C-Receiver)]
{e_0 \to e'_0}
{e_0.m(\overline{e}) \to e'_0.m(\overline{e})}
}

\newcommand{\cpathreceiver} {
\inferrule* [left=(C-PathReceiver)]
{e_0 \to e'_0}
{e_0.K::m(\overline{e}) \to e'_0.K::m(\overline{e})}
}

\newcommand{\cargs} {
%---------- v1.0 ---------------
%\inferrule* [left=(C-Args)]
%{e_i \to e_i'}
%{e.m(..., e_i, ...) \to e.m(..., e_i', ...)}
%}
%---------- v2.0 ---------------
%\inferrule* [left=(C-Args)]
%{e_1 \to f}
%{\angl{J}\new{I}.m(<\overline{E}>\overline{e_0}, e_1, \overline{e_2})
%\to
%\angl{J}\new{I}.m(<\overline{E}>\overline{e_0}, f, \overline{e_2})}
%}
%---------- v3.0 ---------------
\inferrule* [left=(C-Args)]
{e \to e'}
{e_0.m(\ldots,e,\ldots)
\to
e_0.m(\ldots,e',\ldots)}
}

\newcommand{\cpathargs} {
\inferrule* [left=(C-PathArgs)]
{e \to e'}
{e_0.I::m(\ldots,e,\ldots)
\to
e_0.I::m(\ldots,e',\ldots)}
}

\newcommand{\csuperargs} {
\inferrule* [left=(C-SuperArgs)]
{e \to e'}
{super.I::m(\ldots,e,\ldots)
\to
super.I::m(\ldots,e',\ldots)}
}

\newcommand{\cstatictype} {
\inferrule* [left=(C-StaticType)]
{}
{\new{I} \to \; \angl{I}\new{I}}
}

\newcommand{\cfreduce} {
\inferrule* [left=(C-FReduce)]
{e \to e' \\ e \neq \new{J}}
{\angl{I}e \to \; \angl{I}e'}
}

\newcommand{\cannoreduce} {
\inferrule* [left=(C-AnnoReduce)]
{}
{\angl{I}(\angl{J} \new K) \to \angl{I} \new K}
}
%%%%%%%%%%%%%%%%%%%%%%%%%%%%%%%%%%%%%%%%%% syntax.tex end %%%%%%%%%%%%%%%%%%%%%%%%
