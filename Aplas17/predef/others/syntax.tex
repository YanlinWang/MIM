\documentclass{article}
\usepackage{amsmath,amsfonts,amsthm,amssymb,algorithm,algpseudocode}
\usepackage{color,chngpage}
\usepackage{extramarks,epsfig}
\usepackage{fancyhdr,float}
\usepackage{graphicx}
\usepackage{indentfirst}
\usepackage{lastpage,listings}
\usepackage{soul,setspace}
\usepackage{wrapfig}
\usepackage[T1]{fontenc}
\usepackage[scaled=0.85]{beramono}

% Homework Specific Information. Change it to your own

\newcommand{\thistitle}{Syntax: Object Algebras}

% In case you need to adjust margins:
\topmargin=-0.45in      %
\evensidemargin=0in     %
\oddsidemargin=0in      %
\textwidth=6.5in        %
\textheight=9.0in       %
\headsep=0.25in         %

\lstset{ %
language=Java,                % choose the language of the code
columns=flexible,
lineskip=-1pt,
basicstyle=\ttfamily\small,       % the size of the fonts that are used for the code
numbers=none,                   % where to put the line-numbers
numberstyle=\ttfamily\tiny,      % the size of the fonts that are used for the line-numbers
stepnumber=1,                   % the step between two line-numbers. If it's 1 each line will be numbered
numbersep=5pt,                  % how far the line-numbers are from the code
backgroundcolor=\color{white},  % choose the background color. You must add \usepackage{color}
showspaces=false,               % show spaces adding particular underscores
showstringspaces=false,         % underline spaces within strings
showtabs=false,                 % show tabs within strings adding particular underscores
frame=single,                   % adds a frame around the code
morekeywords={sig,where,type,algebra,data,from,let,build,in},
tabsize=2,                  % sets default tabsize to 2 spaces
captionpos=none,                   % sets the caption-position to bottom
breaklines=true,                % sets automatic line breaking
breakatwhitespace=false,        % sets if automatic breaks should only happen at whitespace
title=\lstname,                 % show the filename of files included with \lstinputlisting; also try caption instead of title
escapeinside={(*}{*)},          % if you want to add a comment within your code
keywordstyle=\ttfamily\bfseries,
% commentstyle=\color{Gray},
% stringstyle=\color{Green}
}

% Setup the header and footer
\pagestyle{fancy}
\lhead{} \chead{\thistitle} \rhead{}
\lfoot{} \cfoot{}
\rfoot{Page\ \thepage\ of\ \protect\pageref{LastPage}}
\renewcommand\headrulewidth{0.4pt}                                      %
\renewcommand\footrulewidth{0.4pt}                                      %

%%%%%%%%%%%%%%%%%%%%%%%%%%%%%%%%%%%%%%%%%%%%%%%%%%%%%%%%%%%%%
% Some tools

\newcommand{\Answer}{\ \\\textbf{Answer:} }
\newcommand{\todo}{\textbf{Proposition. } }
\newcommand{\lemma}{\textbf{Lemma. } }
\newcommand{\tab}{\ \ }
\newcommand{\red}[1]{\textcolor{red}{#1}}
\newcommand{\blue}[1]{\textcolor{blue}{#1}}
\newcommand{\nb}{\textit{NB. }}

%%%%%%%%%%%%%%%%%%%%%%%%%%%%%%%%%%%%%%%%%%%%%%%%%%%%%%%%%%%%%


%%%%%%%%%%%%%%%%%%%%%%%%%%%%%%%%%%%%%%%%%%%%%%%%%%%%%%%%%%%%%
% Make title
\title{\huge\textmd{\bf\thistitle}}
\date{}
\author{\today}
%%%%%%%%%%%%%%%%%%%%%%%%%%%%%%%%%%%%%%%%%%%%%%%%%%%%%%%%%%%%%

\begin{document}
\begin{spacing}{1.1}
\maketitle \thispagestyle{empty}

%%%%%%%%%%%%%%%%%%%%%%%%%%%%%%%%%%%%%%%%%%%%%%%%%%%%%%%%%%%%%
% Begin edit from here

\section*{Notation}

\begin{align*}
N,\ T,\ \overline{N},\ \overline{T},\ E:&\textrm{ name, type, list of names, list of types, expression.}\\
\red{\&}(\overline{T}):&\textrm{ intersecting a list of types.}\\
\red{(,,)}(\overline{E}):&\textrm{ merging a list of expressions.}\\
\red{[}\blue{A}\red{\ \mapsto B]}\ E:&\textrm{ substituting }A\textrm{ for }B\textrm{ in expression }E.\\
\red{\Sigma}(...):&\textrm{ collect all fields of records, using copy and paste instead of intersection.}
\end{align*}

\section{Object Algebra Interface}

\subsection{Inheritance $\times$}

\subsubsection{Template}

\begin{lstlisting}[numbers=none]
BEFORE:     (*$\Gamma\ \vdash$*) sig (*$N_{AI}[\overline{T_{AI}}]$*) where (*$\overline{N_{CS}:T_{CS}}$*) in E
\end{lstlisting}
\begin{lstlisting}[numbers=none]
THEN:       (*$\Gamma\ \vdash$*) type (*$N_{AI}[\overline{T_{AI}}]$*) = {(*$\overline{N_{CS}:T_{CS}}$*)} in E
\end{lstlisting}
\begin{lstlisting}[numbers=none]
AFTER:      (*$\Gamma,\ N_{AI}[\overline{T_{AI}}]$*) = (*$\{\overline{N_{CS}:T_{CS}}\}\ \vdash$*)
            (*$\red{[}\blue{N_{AI}[\overline{T}]}\red{\ \mapsto[\blue{\overline{T}}\red{\ \mapsto\overline{T_{AI}}}]\ \{\overline{N_{CS}:T_{CS}}\}]}$*)
            ((*$\lambda$*)(merge(*$N_{AI}$*) : forall (*$\overline{T_{AI}^{(1)}}.$*) forall (*$\overline{T_{AI}^{(2)}}.\ N_{AI}[\overline{T_{AI}^{(1)}}]$*) -> (*$N_{AI}[\overline{T_{AI}^{(2)}}]$*) -> (*$N_{AI}[\overline{T_{AI}^{(1)\&(2)}}]$*)). E)
                (*$\Lambda\overline{T_{AI}^{(1)}}$*). (*$\Lambda\overline{T_{AI}^{(2)}}$*). (*$\lambda$*)(alg1 : (*$\overline{T_{AI}^{(1)}}$*)). (*$\lambda$*)(alg2 : (*$\overline{T_{AI}^{(2)}}$*)).
                    (*$\left\{\overline{N_{CS}\ =\ \lambda(\overline{x}:\overline{T_{CS}^{(1)\&(2)}}).\ alg1.N_{CS}\ \overline{x}\ ,,\ alg2.N_{CS}\ \overline{x}}\right\}$*)
\end{lstlisting}

\subsubsection{Example: \lstinline{ExpAlg[E]}}

\begin{lstlisting}[numbers=none]
BEFORE:     sig ExpAlg[E] where
                lit : Int -> E,
                add : E -> E -> E;
            ...
\end{lstlisting}
\begin{lstlisting}[numbers=none]
THEN:       type ExpAlg[E] = {
                lit : Int -> E,
                add : E -> E -> E
            };
            let mergeExpAlg [E1, E2] (alg1 : ExpAlg[E1]) (alg2 : ExpAlg[E2]) = {
                lit = \(x1 : Int) -> alg1.lit x1 ,, alg2.lit x1,
                add = \(x1 : E1 & E2) -> \(x2 : E1 & E2) -> alg1.add x1 x2 ,, alg2.add x1 x2
            };
            ...
\end{lstlisting}

\newpage

\subsection{Inheritance $\surd$}

\subsubsection{Template}

\begin{lstlisting}[numbers=none]
BEFORE:     (*$\Gamma\ \vdash$*) (*$N_{AI}[\overline{T_{AI}}]$*) extends (*$\overline{N_{AI_2}[\overline{T_{AI_2}}]}$*) where (*$\overline{N_{CS}:T_{CS}}$*) in E
\end{lstlisting}
\begin{lstlisting}[numbers=none]
THEN:       (*$\Gamma\ \vdash$*) type (*$N_{AI}[\overline{T_{AI}}]$*) = (*$\red{\Sigma}(\overline{N_{AI_2}[\overline{T_{AI_2}}]},\overline{N_{CS}:T_{CS}})$*) in E
\end{lstlisting}
\begin{lstlisting}[numbers=none]
AFTER:      (*$\Gamma,\ N_{AI}[\overline{T_{AI}}]$*) = (*$\red{\Sigma}(\overline{N_{AI_2}[\overline{T_{AI_2}}]},\overline{N_{CS}:T_{CS}})\ \vdash$*)
            (*$\red{[}\blue{N_{AI}[\overline{T}]}\red{\ \mapsto[\blue{\overline{T}}\red{\ \mapsto\overline{T_{AI}}}]\ \red{\Sigma}(\overline{N_{AI_2}[\overline{T_{AI_2}}]},\overline{N_{CS}:T_{CS}})]}$*) E
\end{lstlisting}

\nb Need to be updated.

\subsubsection{Example: \lstinline{StatAlg[E, S]}}

\begin{lstlisting}[numbers=none]
BEFORE:     sig StatAlg[E, S] extends ExpAlg[E] where
                seq : S -> S -> S,
                asn : String -> E -> S;
            ...
\end{lstlisting}
\begin{lstlisting}[numbers=none]
THEN:       type StatAlg[E, S] = {
                lit : Int -> E,
                add : E -> E -> E,
                seq : S -> S -> S,
                asn : String -> E -> S
            };
            ...
\end{lstlisting}

\section{Object Algebra}

\subsection{Inheritance $\times$}

\subsubsection{Template}

\begin{lstlisting}[numbers=none]
BEFORE:     algebra (*$N_A$*) implements (*$\overline{N_{AI}[\overline{T_A}]}$*) where (*$\overline{t@(N_{CS}\ \overline{x})=E}$*) in F
\end{lstlisting}
\begin{lstlisting}[numbers=none]
THEN:       let (*$N_A$*) = (*$\red{[}\blue{\overline{T_A}}\red{\ \mapsto\overline{T_{AI}}]}$*) (*$\{\overline{N_{CS}=\lambda(\overline{x}:\overline{T_{CS}}).\ \{t=E\}}\}$*) in F
\end{lstlisting}
\begin{lstlisting}[numbers=none]
AFTER:      ((*$\lambda$*)((*$N_A$*) : ???). F) (*$\red{[}\blue{\overline{T_A}}\red{\ \mapsto\overline{T_{AI}}]}$*) (*$\{\overline{N_{CS}=\lambda(\overline{x}:\overline{T_{CS}}).\ \{t=E\}}\}$*)
\end{lstlisting}

\nb ???

\subsubsection{Example: \lstinline{EvalExpAlg}}

\begin{lstlisting}[numbers=none]
BEFORE:     type IEval = { eval : Int };
            algebra EvalExpAlg implements ExpAlg[IEval] where
                eval@(lit x)(*$\ \ \ $*)= x,
                eval@(add x y) = x.eval + y.eval;
\end{lstlisting}
\begin{lstlisting}[numbers=none]
AFTER:      type IEval = { eval : Int };
            let EvalExpAlg = {
                lit = \(x : Int) -> { eval = x },
                add = \(x : IEval) -> \(y : IEval) -> { eval = x.eval + y.eval }
            };
\end{lstlisting}

\nb Note that in F2J, it should be \lstinline{evalExpAlg}.

\subsection{Inheritance $\surd$}

\subsubsection{Template}

\begin{lstlisting}[numbers=none]
BEFORE:     algebra (*$N_A$*) extends (*$\overline{N_{A_2}}$*) implements (*$\overline{N_{AI}[\overline{T_A}]}$*) where (*$\overline{t@(N_{CS}\ \overline{x})=E}$*);
\end{lstlisting}
\begin{lstlisting}[numbers=none]
AFTER:      let (*$N_A$*) =  (*$\left((\red{,,})\left(\overline{N_{A_2}}\right)\right)$*) ,, (*$\red{[}\blue{\overline{T_A}}\red{/\overline{T_{AI}}]}$*)(*$\{\overline{N_{CS}=\lambda(\overline{x}:\overline{T_{CS}}).\ \{t=E\}}\}$*);
\end{lstlisting}

\nb Need to be updated.

\subsubsection{Example: \lstinline{PrintStatAlg}}

\begin{lstlisting}[numbers=none]
BEFORE:     type IPrint = { print : String };
            algebra PrintExpAlg implements ExpAlg[IPrint] where
                print@(lit x)(*$\ \ \ $*)= "\{x}",
                print@(add x y) = "\{x.print} + \{y.print}";
            algebra PrintStatAlg extends PrintExpAlg implements StatAlg[IPrint, IPrint] where
                print@(seq x y) = "\{x.print} || \{y.print}",
                print@(asn x y) = "\{x} = \{y.print}";
\end{lstlisting}
\begin{lstlisting}[numbers=none]
AFTER:      type IPrint = { print : String };
            let PrintExpAlg = {
                lit = \(x : Int) -> { print = "\{x}" },
                add = \(x : IPrint) -> \(y : IPrint) -> { print = "\{x.print} + \{y.print}" }
            };
            let PrintStatAlg = PrintExpAlg ,, {
                seq = \(x : IPrint) -> \(y : IPrint) -> { print = "\{x.print} || \{y.print}" },
                asn = \(x : String) -> \(y : IPrint) -> { print = "\{x} = \{y.print}" }
            };
\end{lstlisting}

\section{Datatype}

\subsection{Template}

\begin{lstlisting}[numbers=none]
BEFORE:     data (*$N_D[\overline{T_D}]$*) from (*$N_{AI}[\overline{T_{AI}}].S$*);
\end{lstlisting}
\begin{lstlisting}[numbers=none]
AFTER:      type (*$N_D[\overline{T_D}]$*) = { accept : forall ((*$\overline{T_{AI}}\backslash\overline{T_D}$*)). (*$N_{AI}[\overline{T_{AI}}]$*) -> (*$S$*) };
\end{lstlisting}

\nb Usually $\overline{T_{AI}}\backslash\overline{T_D}=S$.

\subsection{Example: \lstinline{List[A]}}

\begin{lstlisting}[numbers=none]
BEFORE:     sig ListAlg[A, L] where
                nil(*$\ \ $*): L,
                cons : A -> L -> L;
            data List[A] from ListAlg[A, L].L;
\end{lstlisting}
\begin{lstlisting}[numbers=none]
AFTER:      type ListAlg[A, L] = {
                nil(*$\ \ $*): L,
                cons : A -> L -> L
            };
            type List[A] = { accept : forall L. ListAlg[A, L] -> L };
\end{lstlisting}

\section{Creating a Structure}

\subsection{Simple Structures}

\subsubsection{Template}

\begin{lstlisting}[numbers=none]
BEFORE:     build (*$N_S$*) : (*$N_D[\overline{T}]$*) = E;
\end{lstlisting}
\begin{lstlisting}[numbers=none]
AFTER:      let (*$N_S$*) = { accept = (*$\Lambda$*)((*$\overline{T_{AI}}\backslash\overline{T_D}$*)). (*$\lambda$*)(alg : (*$N_{AI}[\red{[}\blue{\overline{T}}\red{/\overline{T_D}]}\overline{T_{AI}}]$*)). (*$\red{[}\blue{\overline{alg.N_{CS}}}\red{/\overline{N_{CS}}]}$*)E };
\end{lstlisting}

\nb Potentially there could be name conflicts with \lstinline{alg}. Also names of functions and constructors could probably overlap.

\subsubsection{Example: \lstinline{Exp} and \lstinline{List[Int]}}

\begin{lstlisting}[numbers=none]
BEFORE:     data Exp from ExpAlg[E].E;
            data List[A] from ListAlg[A, L].L;
            build exp : Exp = add (lit 3) (lit 5);
            build lst : List[Int] = cons 3 (cons 5 nil);
\end{lstlisting}
\begin{lstlisting}[numbers=none]
AFTER:      type Exp = { accept : forall E. ExpAlg[E] -> E };
            type List[A] = { accept : forall L. ListAlg[A, L] -> L };
            let exp = { accept = /\E -> \(alg : ExpAlg[E]) -> alg.add (alg.lit 3) (alg.lit 5) };
            let lst = { accept = /\L -> \(alg : ListAlg[Int, L]) -> alg.cons 3 (alg.cons 5 alg.nil) };
\end{lstlisting}

\subsection{Complicated Structures Created by Functions}

\subsubsection{Template}

\begin{lstlisting}[numbers=none]
BEFORE:     build (*$N_S$*) (*$(\overline{x}$*) : (*$\overline{T_1\ne N_D[\overline{T}]})\ (\overline{y}$*) : (*$\overline{T_2=N_D[\overline{T}]})$*) : (*$N_D[\overline{T}]$*) = E;
\end{lstlisting}
\begin{lstlisting}[numbers=none]
AFTER:      let (*$N_S$*) = (*$\lambda$*)((*$\overline{x}$*) : (*$\overline{T_1}$*)). (*$\lambda$*)((*$\overline{y}$*) : (*$\overline{T_2}$*)). {
                accept = (*$\Lambda$*)((*$\overline{T_{AI}}\backslash\overline{T_D}$*)). (*$\lambda$*)(alg : (*$N_{AI}[\red{[}\blue{\overline{T}}\red{/\overline{T_D}]}\overline{T_{AI}}]$*)). (*$\red{[}\blue{\overline{alg.N_{CS}}}\red{/\overline{N_{CS}}]}\red{[}\blue{(\overline{y.accept[\overline{T_{AI}}\backslash\overline{T_D}]\ alg})}\red{/\overline{y}]}$*)E
            };
\end{lstlisting}

\subsubsection{Example: \lstinline{myAdd} and \lstinline{myCons}}

\begin{lstlisting}[numbers=none]
BEFORE:     build myAdd (e1 : Exp) (e2 : Exp) : Exp = add e1 e2;
            build myCons (x : Int) (xs : List[Int]) : List[Int] = cons x xs;
\end{lstlisting}
\begin{lstlisting}[numbers=none]
AFTER:      let myAdd = \(e1 : Exp) -> \(e2 : Exp) -> {
                accept = /\E -> \(alg : ExpAlg[E]) -> alg.add (e1.accept[E] alg) (e2.accept[E] alg)
            };
            let myCons = \(x : Int) -> \(xs : List[Int]) -> {
                accept = /\L -> \(alg : ListAlg[Int, L]) -> alg.cons x (xs.accept[L] alg)
            };
\end{lstlisting}

\nb Extension 1: What if there are BigLambdas in those functions?\\

\nb Extension 2: Does it make sense if there are two different instantiations from one datatype in a function? Namely $N_D[\overline{T_1}]$ and $N_D[\overline{T_2}]$?\\

\nb Extension 3: Recursive ones?\\

\nb IMPORTANT: Why don't we generate \lstinline{"add : Exp -> Exp -> Exp"} and \lstinline{"cons : A -> List[A] -> List[A]"} automatically for global use? My intuition is that for these two examples it's easy; however there could be non-trivial ones. For instance, a new constructor for \lstinline{ListAlg} can be \lstinline{"f : L -> A"}, in which case some structures like \lstinline{"cons (f nil)}\lstinline{ nil"} also make sense, but how to design a template for \lstinline{"f : List[Int] -> Int"}? Instead it's easier to have \lstinline{"alg.cons (alg.f alg.nil)}\lstinline{ alg.nil"}.

\section{Instantiation}

\subsection{Template}

\begin{lstlisting}[numbers=none]
BEFORE:     (*$N_S$*)[(*$\overline{[\overline{T}]}$*)]<(*$\overline{N_A}$*)>
\end{lstlisting}
\begin{lstlisting}[numbers=none]
AFTER:      (*$\red{(,,)}(\overline{N_S.accept[\overline{T}]\ N_A})$*)
\end{lstlisting}

\nb Furthermore, the types $[\overline{[\overline{T}]}]$ could potentially be omitted. But since $N_A$ could implement multiple interfaces, it's not easy to infer the types from context.

Another approach is something like \lstinline{"let result : IEval & IPrint = exp<EvalExpAlg, PrintExpAlg>"}.

\subsection{Example: \lstinline{ListAlg[A, L]} and \lstinline{List[A]}}

\begin{lstlisting}[numbers=none]
BEFORE:     type IEval = { eval : Int };
            type IPrint = { print : String };
            sig ListAlg[A, L] where
                nil(*$\ \ $*): L,
                cons : A -> L -> L;
            algebra SumListAlg implements ListAlg[IEval, IEval] where
                eval@(nil)(*$\ \ \ \ \ \ $*)= 0,
                eval@(cons x y) = x.eval + y.eval;
            algebra PrintListAlg implements ListAlg[IPrint, IPrint] where
                print@(nil)(*$\ \ \ \ \ \ $*)= "",
                print@(cons x y) = "\{x.print} \{y.print}";
            data List[A] from ListAlg[A, L].L;
            build lst : List[Int] = cons 1 (cons 2 (cons 3 nil));
            lst[[IEval, IEval], [IPrint, IPrint]]<EvalListAlg, PrintListAlg>
\end{lstlisting}
\begin{lstlisting}[numbers=none]
AFTER:      type IEval = { eval : Int };
            type IPrint = { print : String };
            type ListAlg[A, L] = {
                nil(*$\ \ $*): L,
                cons : A -> L -> L
            };
            let SumListAlg = {
                nil(*$\ \ $*)= { eval = 0 },
                cons = \(x : IEval) -> \(y : IEval) -> { eval = x.eval + y.eval }
            };
            let PrintListAlg = {
                nil(*$\ \ $*)= { print = "" },
                cons = \(x : IPrint) -> \(y : IPrint) -> { print = "\{x.print} \{y.print}" }
            };
            type List[A] = { accept : forall L. ListAlg[A, L] -> L };
            let lst = {
                accept = /\L -> \(alg : ListAlg[Int, L]) -> alg.cons 1 (alg.cons 2 (alg.cons 3 alg.nil))
            };
            (lst.accept[IEval, IEval] SumListAlg) ,, (lst.accept[IPrint, IPrint] PrintListAlg)
\end{lstlisting}

% Transformation?

% End edit to here
%%%%%%%%%%%%%%%%%%%%%%%%%%%%%%%%%%%%%%%%%%%%%%%%%%%%%%%%%%%%%

\end{spacing}
\end{document}

%%%%%%%%%%%%%%%%%%%%%%%%%%%%%%%%%%%%%%%%%%%%%%%%%%%%%%%%%%%%%
