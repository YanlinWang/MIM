\subsection{Type Safety}

\begin{lemma}~\label{lemma0}
$\textit{If } \mtype(m, I, J) = \overline{D} \rightarrow D, \textit{ and } \mbody(m, I, J) = \overline{x}.e, 
 \textit{ then for some } J_0 \textit{ with } I <: J_0, \  \exists C <: D, \textit{ such that }  
 \judgeewf {\overline{x}:\overline{D}, \kwthis:J_0} {e:C} $.

 \begin{proof}~\\
 $\textit{The base case: if } m \textit{ is defined in } I, \textit{ then it is easy since } m 
 \textit{ is defined in } I \textit{ and } 
 \judgeewf {\overline{x}:\overline{D}, \kwthis : I} {e : C} \textit{ by (T-METHOD)}.
 \textit{ The induction step is also straightforward. }
 $
 \end{proof}
\end{lemma}


\begin{lemma}[Method Type Preservation]~\label{lemma2}
$\textit{If } \mtype(m, I) = \overline{I_X} \rightarrow I_E, \textit{ then } \mtype(m, J) = \overline{I_X} \rightarrow I_E 
\textit{ for all } J \subtype I.
$
\begin{proof}~\\
Straight induction on the derivation of $J \subtype I$. Whether $m$ is defined in $J$ or not, $\mtype(m, J)$ should 
be the same as $\mtype(m, K)$ where $J \; \kwextends \; K \{...\}$.
\end{proof}
\end{lemma}


%%%%%%%%%%%%%%%%%%%%%%%%%%%%%%%%%%%%%%%%%%%%%%%%%%%%
\begin{lemma}[Term Substitution Preserves Typing]~\label{lemma1}
$\textit{If } \ \judgeewf {\Gamma, \ \overline{x}:\overline{I_X}} { e : I } \textit{ , and }
\judgeewf {\Gamma} {\overline{y}:\overline{I_Y}} \textit{ where } \overline{I_Y} \subtype \overline{I_X}
\textit{, then } \judgeewf {\Gamma} {[\overline{y}/\overline{x}]e : I'} \textit{ for some } I' \subtype I.
$

\begin{proof}~\\
\noindent \textbf{Case Var.}
$ e = x \quad I = \Gamma(x) $. \\
If $x \notin \overline{x}$, then the conclusion is immediate, since $[\overline{y}/\overline{x}]x = x$.
On the other hand, if $x = x_i$ and $I = {I_X}_i$, then since $[\overline{y}/\overline{x}]x = [\overline{y}/\overline{x}]x_i = y_i$,
letting $I' = {I_Y}_i$ finishes the case.

\noindent \textbf{Case New.}
$e = \new I$ and there are no term for substitution, the conclusion is obvious.

\noindent \textbf{Case Invk.}
$ e = e_0.m(\overline{e}) \quad
  \judgeewf {\Gamma, \overline{x}:\overline{I_X}} {e_0 : I_0} $
$$ mtype(m, I_0) = \overline{I_E} \rightarrow J $$
$$ \judgeewf {\Gamma, \overline{x}:\overline{I_X}} {\overline{e}:\overline{I}} \quad
    \overline{I} \subtype \overline{I_E} $$
By induction hypothesis, there are some $I_0'$ and $\overline{I_E'}$ such that
    $$ \judgeewf {\Gamma} {[\overline{y}/\overline{x}]e_0 : I_0'} \quad 
        I_0' \subtype I_0 $$
    $$ \judgeewf {\Gamma} {[\overline{y}/\overline{x}]\overline{e} : \overline{I_E'}} \quad  
        \overline{I_E'} \subtype \overline{I}$$    
By lemma~\ref{lemma2}, 
    $mtype(m, I_0') = \overline{I_E} \rightarrow J$,
then $\overline{I_E'} \subtype \overline{I_E}$ by the transitivity of $\subtype$.
Therefore, by the rule (T-INVK), 
    $\judgeewf {\Gamma} {[\overline{y}/\overline{x}]e_0.m([\overline{y}/\overline{x}]\overline{e}) : J}$.

\noindent \textbf{Case PathInvk.}
$ e = e_0.I::m(\overline{e}) $ and proof is similar as case Var.

\noindent \textbf{Case SuperInvk.}
$ e = \kwsuper.I::m(\overline{e}) $ \\
Suppose $\judgeewf {\Gamma} {\kwthis : I_0}$, the following proof should be similar as case Var.
\end{proof}

\end{lemma}
%%%%%%%%%%%%%%%%%%%%%%%%%%%%%%%%%%%%%%%%%%%%%%%%%%%%

\begin{theorem}[Subject Reduction]~\label{theorem_subject}
$\textit{If } \ \judgeewf \Gamma {e : I} \; \textit{ and } \ e \rightarrow e',\ 
\textit{then } \judgeewf \Gamma {e' : I'} \ \textit{ for some } \ I' <: I.$
\end{theorem}


\begin{proof} ~\\
\noindent \textbf{Case Invk.} 
let \[ e = <J>\new I.m(<\overline{I_E}>\overline{e}) \] 
Suppose \[ \mbody(m, I, J) = (\overline{I_X} \; \overline{x}, I_{E_0} \; e_0) \] 
then \[ e' =  [<\overline{I_X}>\overline{e}/\overline{x}, \; <J>\new I/\kwthis ] e_0 \] 
By rules (T-NEW) and (T-INVK), 
  \[ \judgeewf \Gamma {\new I:I} \quad 
     \mtype(m, I, J) = \overline{I_X} \rightarrow I_{E_0} \quad 
     \judgeewf \Gamma {\overline{e} : \overline{Y_E}} \quad
     \overline{Y_E} \subtype \overline{X} \quad
     \textit{, for some } \; \overline{Y_E}
  \]
By Lemma~\ref{lemma0},
    \[
    \judgeewf {\Gamma, \overline{x}:\overline{I_X}, \kwthis:J_0} {e_0:I_F} \textit{, for some } J \subtype J_0 \textit{ and } I_F \subtype I_{E_0}
    \]
By Lemma~\ref{lemma1},
    \[
    \judgeewf {\Gamma} {[<\overline{I_X}>\overline{e}/\overline{x}, \; <J>\new I/\kwthis ] e_0  :  I_G} \textit{, for some } I_G \subtype I_F 
    \]
So $I_G <: I_{E_0}$, finally just let $I' = I_G$.

\noindent \textbf{Case PathInvk.}
let \[ e = <J>\new I.K::m(<\overline{I_E}>\overline{e}) \]  
Suppose \[ \mbody(m, I, K) = (\overline{I_X} \; \overline{x}, I_{E_0} \; e_0) \] 
then \[ e' =  [<\overline{I_X}>\overline{e}/\overline{x}, \; <K>\new I/\kwthis ] e_0 \] 
By rules (T-NEW) and (T-INVK), 
  \[ \judgeewf \Gamma {\new I:I} \quad 
     \mtype(m, I, K) = \overline{I_X} \rightarrow I_{E_0} \quad 
     \judgeewf \Gamma {\overline{e} : \overline{Y_E}} \quad
     \overline{Y_E} \subtype \overline{X} \quad
     \textit{, for some } \; \overline{Y_E}
  \]
By Lemma~\ref{lemma0},
    \[
    \judgeewf {\Gamma, \overline{x}:\overline{I_X}, \kwthis:J_0} {e_0:I_F} \textit{, for some } K \subtype J_0 \textit{ and } I_F \subtype I_{E_0}
    \]
By Lemma~\ref{lemma1},
    \[
    \judgeewf {\Gamma} {[<\overline{I_X}>\overline{e}/\overline{x}, \; <K>\new I/\kwthis ] e_0  :  I_G} \textit{, for some } I_G \subtype I_F 
    \]
So $I_G <: I_{E_0}$, finally just let $I' = I_G$.

\noindent \textbf{Case Super-Invk.}
let \[ e = \kwsuper.K::m(<\overline{I_E}>\overline{e}) \]   
Suppose \[ \mbody(m, K, K) = (\overline{I_X} \; \overline{x}, I_{E_0} \; e_0) \] 
then \[ e' =  [<\overline{I_X}>\overline{e}/\overline{x}] e_0 \] 
By rules (T-NEW) and (T-INVK), 
  \[ 
     \mtype(m, K, K) = \overline{I_X} \rightarrow I_{E_0} \quad 
     \judgeewf \Gamma {\overline{e} : \overline{Y_E}} \quad
     \overline{Y_E} \subtype \overline{X} \quad
     \textit{, for some } \; \overline{Y_E}
  \]
By Lemma~\ref{lemma0},
    \[
    \judgeewf {\Gamma, \overline{x}:\overline{I_X}, \kwthis:J_0} {e_0:I_F} \textit{, for some } K \subtype J_0 \textit{ and } I_F \subtype I_{E_0}
    \]
By Lemma~\ref{lemma1},
    \[
    \judgeewf {\Gamma} {[<\overline{I_X}>\overline{e}/\overline{x} ] e_0  :  I_G} \textit{, for some } I_G \subtype I_F 
    \]
So $I_G <: I_{E_0}$, finally just let $I' = I_G$.

\end{proof}

\begin{theorem}[Progress]~\label{theorem_progress}
%$\textit{If } \ \judgeewf {} {e : I}, \textit{ then either } \ e 
%\textit{ is a value or there is an } e' \textit{ with } e \rightarrow e'.$
$\textit{Suppose } e \textit{ is a well-typed expression } $ \\
\begin{enumerate}
\item If $e$ includes $\left(<J>\new{I}\right).m(\overline{<I_E>} \; \overline{e})$ as a subexpression,
    then $ \mbody(m, I, J) = (\overline{I_X} \; \overline{x}, I'_E \; e_0) $ and
         $\num{\overline{x}} = \num{\overline{e}}$ for some $\overline{I_X}, \overline{x}, I_E'$ and $e_0$
\item If $e$ includes $\left(<J>\new{I}\right).K::m(\overline{<I_E> e})$ as a subexpression,
    then $ \mbody(m, I, K) = (\overline{I_X} \; \overline{x}, I'_E \; e_0) $ and 
         $\num{\overline{x}} = \num{\overline{e}}$ for some $\overline{I_X}, \overline{x}, I_E'$ and $e_0$
\item If $e$ includes $\kwsuper.K::m(\overline{<I_E> e})$ as a subexpression,
    then $ \mbody(m, K, K) = (\overline{I_X} \; \overline{x}, I'_E \; e_0) $ and 
         $\num{\overline{x}} = \num{\overline{e}}$ for some $\overline{I_X}, \overline{x}, I_E'$ and $e_0$
\end{enumerate}
\begin{proof}~\\
\noindent \textbf{Case 1.}
Given that $e$ is well-typed, by rule (T-INVK), $\mtype(m, J)$ is defined.
By rule (T-INTF) for interface $I$,
    \[ \forall J >: I, m, \mtype(m, J) \text{ is defined } \Rightarrow \mbody(m, I, J) \text{ is defined } \]
So straightforwardly, we get $ \mbody(m, I, J) = (\overline{I_X} \; \overline{x}, I'_E \; e_0) $ and
         $\num{\overline{x}} = \num{\overline{e}}$ for some $\overline{I_X}, \overline{x}, I_E'$ and $e_0$
         
\noindent \textbf{Case 2 and 3.} Proof for case 2 and 3 is similar to case 1, and we omit it here.
\end{proof}
\end{theorem}


\begin{theorem}[Type Soundness]~\label{theorem_soundness}
If $\judgeewf \o {e : I}$ and $e \to^* e'$ with $e'$ a normal form, then $e'$ is 
a value $v$ with $\judgeewf \o {v:J}$ and $J \subtype I$.
\begin{proof}
Immediate from Theorem~\ref{theorem_subject} and Theroem~\ref{theorem_progress}.
\end{proof}
\end{theorem}