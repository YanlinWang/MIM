\section{Formalization}

In this section, we present a formalization of our MIM calculus, based on
a minimal subset of Java 8 interfaces.\bruno{Mention that it is done 
in a FJ style?} The syntax, typing rules and small-step
semantics are included below.

\subsection{Syntax} \haoyuan{TODO: syntactic sugar}

Figure~\ref{fig:syntax} shows the syntax of MIM. The multiple
inheritance feature of MIM as a basis Java 8 interfaces, which support
method implementations via default methods. This feature is quite
closely related to \emph{traits}\cite{}.  To demonstrate how
unintentional method conflicts are untangled in MIM, we present the
calculus in a straightforward way, hence we only focus on a small
subset of the interface model. For example, all the methods declared
in an interface are default methods, that is to say, they always
provide default implementations.  From this point we can view that we
are actually modelling a class model that supports multiple
inheritance. Then it is straightforward to do object creation like
$\new I$. Fields and primitive types are not modelled as well.

We use uppercase letters like $I, J, K$ to represent identifiers for
interfaces. By multiple inheritance, an interface can have a set of
super interfaces, where such a set can be empty. Inside an interface
is a set of method declarations. Each method body holds a return
statement. As seen in Figure~\ref{fig:syntax}, we have introduced the
\kwoverride{} keyword to override an old implementation of the
method. If the interface that it overrides is exactly the enclosing
interface, then such a method is seen as ``originally defined''.
Again for simplicity, overloading is not modelled for methods, which
implies we can uniquely identify a method by its name.

An expression can be a variable, a method invocation, an object
creation, furthermore, a path-invocation like
``$e.I::m(\overline{e})$'', meaning that the dynamically binded
implementation for method $m$ should be along the path $I$. Another
case is the super-invocation, enabling a method to access an old
implementation from the specified super type. Hence a super-invocation
can only be used inside an interface definition. Finally an expression
can also be a value ``$<I> \new J$''. It is exactly an object instance
of $J$ with annotated static type $I$.  Note that values are only
intended for the small-step semantics of MIM, hence they are not
supposed to appear in the source program.


\begin{figure*}[t]
\begin{displaymath}
\begin{array}{l}
\begin{array}{llrl}
\text{Interfaces}   & IL & \Coloneqq & \interface{I}{I}{M} \\
\text{Methods}      & M  & \Coloneqq & \method{I}{m}{I_x}{x}{J}{e} \\
\text{Expressions}  & e  & \Coloneqq & x \mid
e.m(\overline{e}) \mid
\new{I} \mid
e.I::m(\overline{e}) \mid
\kwsuper.I::m(e) \; \mid \; <I> e \\
\text{Context}      & \Gamma & \Coloneqq & \overline{x:I} \\
\text{Values}       & v & \Coloneqq & <I>\new{J} \\
\\
\text{Interface names} & I, J, K & & \\
\text{Method names} & m & & \\
\text{Variable names} & x & &
\end{array}
\end{array}
\end{displaymath}
\caption{Syntax. \haoyuan{mention in text that annotated expressions are only intended for semantic rules?}}\label{fig:syntax}
\end{figure*}

\subsection{Subtyping and Typing Rules}
The subtyping of MIM consists of only a few rules shown in Figure ?.
In short, subtyping relations are built from the inheritance in interface
declarations. They hold both reflexivity and transitivity.

Details of type-checking rules are displayed in Figure ?, including expression
typing, well-formedness of methods and interfaces. As a convention, an environment
$\Gamma$ is maintained to store the types of variables, together with ``this'' type, namely
the enclosing type. The three rules for method invocation, (T-INVK), (T-PATHINVK) and (T-SUPERINVK)
are very similar, in the sense that they all check the type of the specific method, by using
an auxiliary function \mtype. \mtype is the function for looking up method types, which we will
illustrate later. After the method type is obtained, they all check that arguments and the receiver
have compatible types. Additionally, (T-PATHINVK) requires the receiver to be the subtype of the specified
path type, and (T-SUPERINVK) checks if the enclosing type directly extends the specified super type.

The method typing rule (T-METHOD) is more interesting, since the method can either be an original implementation or a update.
Besides static type-checking for the return expression,
we further use the helper function $\mostSpecific$ to ensure that the method update is legal. The formal definition is available in Section ?
By that we define the legality of method updating: the updated method must be an original method (not another method update), and method updating cannot cross over method overriding (a method overriding cannot appear between the method update and the updated original method in inheritance hierarchy).

Finally, (T-INTF) defines interface type-checking in a straightforward way. Note that it is responsible for checking conflicted paths with a same source. As we illustrated in Section ?, such a case is disallowed for unambiguity.

\begin{figure*}[t]
\begin{mathpar}
	\framebox{$ I <: J $} \hspace{.5in} \subid \\
	\subtrans \hspace{.5in} \subextends
\end{mathpar}
\caption{Subtyping.}\label{fig:subtyping}
\end{figure*}


\begin{figure*}[t]
\begin{mathpar}
	\framebox{$ \judgeewf \Gamma {e:I} $} \hspace{.5in}
	\tvar \\
	\tinvk \\
	\tpathinvk \\
	\tsuperinvk \\
	\tnew \\
	\tmethod \\
	\tintf
\end{mathpar}
\caption{Typing rules.}\label{fig:typingrules}
\end{figure*}
\bruno{Path invocation subsumes regular invocation. In other words:
e0.m(e) = e0.I::m(e) when the type of e0 is I.}
\bruno{Last time I met I told you that the typing of ``this'' is
  wrong, and this is not yet fixed. I've also told you to make all
figures at the top. In T-Method, $\Gamma$ shows up out of nowhere.
}
\bruno{Rule for new is wrong! It says that a new expression that
  creates a class with any name (even if the class does not exist) 
is valid. The paper is not even following the rules of FJ.}

\subsection{Small-step Semantics and Congruence}
Figure ? and Figure ? define small-step semantic rules and congruence rules,
respectively. When evaluating an expression, they are invoked recursively and
alternately to produce a single value in the end. The small-step semantics (S-INVK), (S-PATHINVK) and (S-SUPERINVK) behave similarly, each corresponds to
one kind of method invocation. They all invoke \mbody{} for method body lookup. Generally, one can understand $\mbody(m, I, J)$ in a way that it finds the most specific body of method $m$, when the receiver has dynamic type $I$ and static type $J$.
The three rules require that the receiver and the arguments have been evaluated into values, before substitution is applied.

\begin{figure*}[t]
\begin{mathpar}
	\sinvk \\
	\spathinvk \\
	\ssuperinvk
\end{mathpar}
\caption{Small-step semantics.\haoyuan{fixed rules. Need to change text.}}\label{fig:smallstep}
\end{figure*}


\begin{figure*}[t]
\begin{mathpar}
	\creceiver \hspace{.5in}
	\cpathreceiver \\
	\cargs \\
	\cpathargs \\
	\csuperargs \\
	\cstatictype \\
	\cfreduce \\
	\cannoreduce
\end{mathpar}
\caption{Congruence.}\label{fig:congruence}
\end{figure*}



\subsection{Auxilary Definitions}

To make our formalization concise and expressive, we have defined a list of
auxiliary functions, collected by Figure~\ref{fig:auxfunc}. To begin with, we
introduce the basic functions: $\ext, \updateSet$ and $\prune$. $\ext(I, J)$
simply indicates that interface $I$ directly extends interface $J$. Corresponding
to this is a more general case $I <: J$, meaning that $I$ is a subtype of $J$.
$\updateSet(I, J)$ returns a set of methods defined in $I$ that have ``$\kwoverride \; J$''
in their signatures. Notice that $\updateSet(I, I)$ is a special representative of
the ``originally-defined'' method set from $I$. The $\prune$ function takes a set of
types, and filters out those that have subtypes in the same set. Finally in the returned set,
none of them has a subtyping to one another, since all super types have been removed.

\begin{figure*}[t]
	\begin{mathpar}
	\inferrule* [left=]
		{  \mostSpecific(m, I_d, I_s) = \{I\} \\
			\mostSpecific_2(m, I_d, I) = \{J\} \\
			\kwinterface \; J \; \kwextends \; \overline{J} \; \{\method{I_E}{m}{I_X}{x}{I}{e_0}\ldots\}}
		{\mbody(m, I_d, I_s) = (J, \overline{I_X} \; \overline{x}, I_E \; e_0)}
	
	\inferrule* [left=]
	{ set1 = \; \{ K <: J \; $and$ \; K >: I \; | \; m \in \updateSet(K, K) \} \\
		set2 = \; \{ K >: J \; | \; m \in \updateSet(K, K) \} }
	{\mostSpecific(m, I, J) = \left\{{\begin{tabular}{ll}
				$prune(set1)$ & if $set1$ is not empty \\ $prune(set2)$ & otherwise
			\end{tabular}}\right.}
	
	\inferrule* [left=]
		{ set = \; \{ K <: J \; $and$ \; K >: I \; | \;
			m \in \updateSet(K, J)  \} }
		{\mostSpecific_2(m, I, J) = prune(set)}
	
	prune(set) = \{I \in set \; | \; \nexists J \in set, J <: I, J \neq I \}
	
	\inferrule* [left=]
	{   \interface{I}{I}{M}
		\\ J \in \overline{I} }
	{\ext(I, J)}
	
	\inferrule* [left=]
	{   \kwinterface \; I \; \kwextends \; \overline{I} \; \{ I_E \; m(\overline{I_X} \; \overline{x}) \;
		\kwoverride \; J \ldots \} }
	{m \in \updateSet(I, J)}
	\end{mathpar}
	\caption{Auxiliary functions.}\label{fig:auxfunc}
\end{figure*}

\subsubsection{$\mostSpecific$ and $\mostSpecific_2$}

$\mostSpecific$ is an auxiliary function that finds the most specific original implementations of a method. Let us consider $\mostSpecific(m, I, J)$, what it returns is a set of interfaces, each including its own $m$ as a most specific implementation. Such a set may contain several elements, but that implies ambiguity; what we expect is actually a singleton set. By the definition of $\mostSpecific$ shown in Figure ?, an interface belongs to the return set if and only if:
\begin{itemize}
	\item It has an original definition of $m$;
	\item It is a supertype of $I$;
	\item It is along path $J$, meaning that it is either a supertype or a subtype of $J$ (including $J$ itself);
	\item It does not have a subtype in the same set, because we have used $prune$ to filter out all super types, as the most specific one is always in the sub-most type.
\end{itemize}
We could have put $set1$ and $set2$ together, but the current definition leads a clearer illustration.

The $\mostSpecific$ function only focuses on original method implementations, all the method updates are omitted during that time. On the other hand, another auxiliary function $\mostSpecific_2(m, I, J)$ has the assumption that $J$ defines an original $m$, and this function tries to find the most specific implementations that update such an $m$. Just as $\mostSpecific$, $\mostSpecific_2$ also returns the set of interfaces after pruning. An interface belongs to the return set if and only if:
\begin{itemize}
	\item It is between $I$ and $J$;
	\item It defines a method update for $J.m$;
	\item It does not have a subtype in the same set.
\end{itemize}
The algorithm for finding the most specific method update is quite similar to that for most specific original method. A method update is not allowed to work on another method update, and one can hide another if their interfaces has subtyping relations. If they do not hide each other, the result implies ambiguity.

\subsubsection{$\mbody$ and $\mtype$}

$\mbody(m, I_d, I_s)$, as defined in Figure~\ref{fig:auxfunc}, denotes a method body lookup function.
We use $I_d, I_s$, since $\mbody$ is usually invoked by a receiver of a method $m$, with its dynamic
type $I_d$ and static type $I_s$. Such a function returns the most specific method implementation, more
accurately, its parameters, returned expression and the types. It considers both originally defined methods and method updates, so $\mostSpecific$ and $\mostSpecific_2$ are invoked.

To calculate $\mbody(m, I_d, I_s)$:
\begin{itemize}
	\item First, it invokes $\mostSpecific(m, I_d, I_s)$ and returns a set.
	\item If $\mostSpecific$ returns a singleton set $\{I\}$, then it is good, otherwise $\mbody$ is undefined in
	this case. The set $\{I\}$ implies that we will use the $m$ from $I$ without ambiguity. But moreover, we have to invoke $\mostSpecific_2(m, I_d, I)$, to check if there are updated versions of $m$ between $I_d$ and $I$. Again we forbid ambiguity, so the expected set after pruning is also a singleton set $\{J\}$.
	\item Finally, we fetch the implementation of $m$ in interface $J$ and return its related information.
\end{itemize}
The definition of $\mtype$ simply relies on $\mbody$. In short,
$$\mbody(m, I, I) = (J, \overline{I_x}\ \overline{x}, I_E\ e) \ \Longrightarrow\ \mtype(m, I) = \overline{I_x}\rightarrow I_E$$

\begin{comment}
$mbody(m, I)$ algorithm:
\begin{itemize}
	\item If m is defined in I directly, then return I.m()
	\item Else, let $\overline{I'} = mdefined(fathers(I))$, all ancestors of $I$ that has directly defined $m()$.
	\item $\overline{I''} = needed(\overline{I'})$, keep only interfaces that are needed, which are not super-interface of others.
	\item If $\overline{I''}$ is unique, then return this unique one. Else if any two I1,I2 in $\overline{I''}$ share a parent in $\overline{I'}$, then diamond conflict is detected, report error. Else return multiple $m()$s.
\end{itemize}
\end{comment}

\begin{comment}
\subsubsection{\collectMethods}
\[ \collectMethods(I) = \left( \bigcup_{I_i \in \overline{I}} \methods(I_i) \right) \bigcup \methods(I) \]
\[ \methods(I) = \overline{M}, \text{where } IT(I) = \interface{I}{I}{M} \]
\end{comment}



\subsection{Properties}
Previously the definitions of our model are given, now we should proceed to prove the type soundness of 
the model, which relates typing to computation. We want to prove such a property: if an expression is 
well-typed, then after many reduction steps it must reduces to a value of a subtype of the original type
of the expression. Following the FeatherweightJava paper~\cite{Igarashi01FJ}, the type-soundness theorem 
(Theorem~\ref{theorem_soundness}) is proved by using the standard technique of subject reduction (Theorem~\ref{theorem_subject})
and progress (Theorem~\ref{theorem_progress})~\cite{Wright1994}.

\begin{theorem}[Subject Reduction]~\label{theorem_subject}
$\textit{If } \ \judgeewf \Gamma {e : I} \; \textit{ and } \ e \rightarrow e',\ 
\textit{then } \judgeewf \Gamma {e' : I'} \ \textit{ for some } \ I' <: I.$
\end{theorem}
\begin{proof}
See Appendix~\ref{appendix_proof}.
\end{proof}

\begin{theorem}[Progress]~\label{theorem_progress}
$\textit{Suppose } e \textit{ is a well-typed expression } $ \\
\begin{enumerate}
\item If $e$ includes $\left(<J>\new{I}\right).m(\overline{<I_E>} \; \overline{e})$ as a subexpression,
    then $ \mbody(m, I, J) = (\overline{I_X} \; \overline{x}, I'_E \; e_0) $ and
         $\num{\overline{x}} = \num{\overline{e}}$ for some $\overline{I_X}, \overline{x}, I_E'$ and $e_0$
\item If $e$ includes $\left(<J>\new{I}\right).K::m(\overline{<I_E> e})$ as a subexpression,
    then $ \mbody(m, I, K) = (\overline{I_X} \; \overline{x}, I'_E \; e_0) $ and 
         $\num{\overline{x}} = \num{\overline{e}}$ for some $\overline{I_X}, \overline{x}, I_E'$ and $e_0$
\item If $e$ includes $\kwsuper.K::m(\overline{<I_E> e})$ as a subexpression,
    then $ \mbody(m, K, K) = (\overline{I_X} \; \overline{x}, I'_E \; e_0) $ and 
         $\num{\overline{x}} = \num{\overline{e}}$ for some $\overline{I_X}, \overline{x}, I_E'$ and $e_0$
\end{enumerate}
\end{theorem}
\begin{proof}
See Appendix~\ref{appendix_proof}.
\end{proof}

\begin{theorem}[Type Soundness]~\label{theorem_soundness}
If $\judgeewf \o {e : I}$ and $e \to^* e'$ with $e'$ a normal form, then $e'$ is 
a value $v$ with $\judgeewf \o {v:J}$ and $J \subtype I$.
\end{theorem}
\begin{proof}
Immediate from Theorem~\ref{theorem_subject} and Theroem~\ref{theorem_progress}.
\end{proof}
