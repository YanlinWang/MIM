\section{Introduction}
Inheritance in Object-Oriented Programming (OOP) offers a mechanism
for code reuse. However many OOP languages are restricted to single
inheritance, which is less expressive and flexible than multiple
inheritance. Nevertheless different flavours of multiple inheritance
have been adopted in some popular OOP languages. C++ has had 
multiple inheritance from the start. Scala adapts ideas from traits~\cite{scharli03traits} 
and mixins~\cite{bracha90mixin} to offer a disciplined form of multiple inheritance. Java 8 
offers a variant of traits with disguised of interfaces with default methods~\cite{goetz12fdefenders}.

A reason why programming languages have resisted to multiple
inheritance in the past is that, as Cook~\cite{Cook1987} puts it, 
multiple inheritance is good but ``\emph{there is no good way to do it}''.
One of the most sensitive and critical issues is perhaps the ambiguity
introduced by multiple inheritance. One case is the famous
\textit{diamond problem}~\cite{Sak89dis,Singh1995}. In the diamond problem inheritance could allow
one feature to be inherited from multiple parent classes that share a
common ancestor, hence
conflicts arise. The variety of strategies for resolving such conflicts
urges the occurrence of different multiple inheritance models,
including traits, mixins, CZ~\cite{malayeri2009cz}, and many others. Existing
languages and research have taken care of this case extensively. Other issues
include how multiple inheritance deals with state, 
which also has been discussed quite extensively~\cite{classless,malayeri2009cz,stroustrup1995}.

In contrast with the diamond inheritance, a second case of ambiguity
is \textit{unintentional method conflicts}~\cite{scharli03traits}. That is conflicting 
methods that do not actually refer to the same feature. 
In a nominal system, methods can be designed for different
functionality, but they just happen to have same names (and
signatures).
A simple example of this situation are two \lstinline{draw} methods that
are inherited from a deck of cards and a drawable widget. 
In such context the two draw methods have very different meanings, 
but they happen to share the same name.
%%This issue was proposed by the trait paper, so-called
When inheritance is used to compose those methods an
error happens due to conflicts. However, unlike the diamond problem,
the conflicted methods have very different meanings and do not share a
common parent. We call such a case ``\textit{triangle inheritance}''.
\haoyuan{Fine to have a graph with diamond and triangle shapes if we got space.}

Unintentional method conflicts are less common than the diamond
problem, nonetheless, they have severe effects in practice. This is
unfortunate because such an issue has not received much formal study 
before. In practice existing languages also provide limited support for
it. In most languages, the mechanisms available to deal with this problem are the same as the diamond
inheritance. Unfortunatelly this is often inadequate and can lead 
to tricky problems in practice. This is especially the case
when it is necessary to combine two large modules and their features,
but the inheritance is simply prohibited by a small conflict. As
a workaround from the diamond inheritance side, it is possible to
define a new method in the child class, to override those conflicting
methods. Intuitively using one method to fuse two unrelated features
is unsatisfactory. Therefore we need a better solution to keep both
features separately during inheritance, so as not to break
\emph{independent extensibility}~\cite{zenger05independentlyextensible}.

Some other workarounds or approaches include delegation, and
renaming/exclusion in the trait model. Yet they still have various
drawbacks as we will discuss in Section~\ref{sec:overview}. Closest to our work
are mechanisms available in C++ and C\# that allow for two
unintentionally conflicted methods to coexist in a class. Among them
C++ is a representative that accepts the triangle inheritance and
resolves the ambiguity by \emph{static dispatching}. However, C++ has
limited support for virtual methods with unintentional conflicts, and
it will often throw errors when composing them. This is again
unsatisfactory because virtual methods are pervasive in OOP and used 
for code reuse and extensibility. A problem with the C++ approach is
that it can only use either static or dynamic dispatching separately, but dealing
with unintentional method conflicts seems to require a combination of both. 

%Having tolerance for unintentional method conflicts does not mean to
%sacrifice extensibility, hence in contrast with static dispatch and
%dynamic dispatch, 

This paper proposes \textit{hierarchical dispatch}: a novel approach
to method dispatching, which combines static and dynamic
information. Using hierarchical dispatch, the method binder will look
at both the \emph{static type} and the \emph{dynamic type} of the
receiver during runtime. When there are multiple branches that cause
unintentional conflicts, the static type can specify one branch among
them for unambiguity, and the dynamic type helps to find the most
specific implementation. In that case both unambiguity and
extensibility are preserved. To present this idea, we introduce a
formalized model in Section~\ref{sec:formalization} based on
Featherweight Java~\cite{Igarashi01FJ}, together with theorems and
proofs for type soundness. In the model we also propose
\textit{partial overrides}, where method overriding can be applied
only to one branch of the class hierarchy. Our model can be viewed as
a generalization of the trait model by providing additional support to
the triangle inheritance.

In summary, our contributions are: \haoyuan{will have prototype implementation in Scala if we still have time.}
\begin{itemize}
	\item \textbf{Hierarchical dispatch:} which integrates both the static type and dynamic type for method dispatch, and hence
	ensures unambiguity as well as extensibility.
	\item \textbf{Partial overrides:} a novel notion that allows
          method overrides on individual branches of the class hierarchy.
	\item \textbf{\name:} a formalized model based on Featherweight Java, supporting the above features. Some theorems and their proofs
	are attached to confirm the type soundness (unambiguity?) of
        the model.\bruno{Anything about unambiguity?}
\end{itemize}

 