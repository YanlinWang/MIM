\section{Related Work}

\subsection{Mainstream Multiple Inheritance Models}
Multiple inheritance is a useful feature in object-oriented programming world although it's difficult to model and can 
cause various problems (e.g. the diamond problem~\cite{Sak89dis,Singh1995}).  There are many existing languages/models that support multiple 
inheritance, either coming with multiple inheritance capability or added through evolution. The Mixin model allows naming components 
that can be applied to various classes as reusable functionality units. However, the linearization (total ordering) of mixin 
inheritance cannot provide a satisfactory resolution in some cases and restricts the flexibility of mixin composition. 

Simplifying the mixins approach, traits~\cite{scharli03traits} draw a strong line between units of reuse and object factories. 
Traits act as units of reuse, containing functionality code; while classes, assembled from traits, act as object factories. 

Java 8 interfaces are closely related to traits: concrete method implementations are allowed (via the \textbf{\texttt{default}} keyword) inside interfaces. 
The introduction of default methods opens the gate for various flavors of multiple inheritance in Java.

Malayeri and Aldrich proposed a model CZ~\cite{malayeri2009cz} which aims to do multiple inheritance without the diamond problem. 
Inheritance is divided into two concepts: inheritance dependency and implementation inheritance. 
Using a combination of \textbf{\texttt{requires}} and \textbf{\texttt{extends}}, a program with 
diamond inheritance is transformed to one without diamonds. Moreover, fields and multiple inheritance can coexist. 
However untangling inheritance also untangles the class structure. In CZ, not only the number of classes, but also 
the class hierarchy complexity increases. 

The above mentioned models/languages support multiple inheritance, and they handle method conflicts in the same way, by
simply disallowing two methods with the same signature from two different units to coexist. The ambiguity is resolved either
by programmers explicitly or by setting a linearised composition. This is the common drawback of the above mentioned models versus our model.

\subsection{Resolving Unintended Method Conflicts}
There are still a few languages that partly realize the problem. They provide limited support for unintended conflicting methods. 

\noindent {\bf C++ model.}
C++ supports very flexible inheritance models and allows programmers to choose either static dispatch or dynamic dispatch for method lookup.
It allows unintended conflicts and uses static dispatch to resolve them, as discussed in Section~\ref{sec:overview}. For example, given the following code
\begin{lstlisting}[language=Java]
class A { public: void m() {cout << "MA" << endl;} };
class B { public: void m() {cout << "MB" << endl;} };
class C : public A, public B { 
	void m() {cout << "MC" << endl;}
};
void func(A* a) { a->m(); }
int main() {
	C* c = new C();
	c->B::m();
	func(c); 
	return 0; //Running result: MB MA
}
\end{lstlisting}
The running result is $MB \; MA$, meaning that it uses static dispatch (looks at the static types) for method \lstinline|m()|. 
On calling \texttt{func(c)}, in spite that the dynamic type of \texttt{c} is class \texttt{C}, the method call still dispatches to 
\lstinline|A.m|.
However, we can alter the code a little bit with \textbf{\texttt{virtual}} methods and the result will be totally different:
\begin{lstlisting}[language=c++]
class A { public: virtual void m() {cout << "MA" << endl;} };
class B { public: virtual void m() {cout << "MB" << endl;} };
class C : public A, public B { 
    public: virtual void m() {cout << "MC" << endl;}
};
void func(A* a) { a->m(); }
int main() {
	C* c = new C();
	c->B::m();
	func(c); 
	return 0; //Running result: MB MC
}
\end{lstlisting}
Now the running result will be $MB \; MC$. With virtual methods, dynamic dispatch is used and 
method lookup algorithm will find the most specific method definition of $m$, namely \lstinline|C.m| at this time.
Although C++ support this flexibility, unfortunately, dynamic dispatch on conflicting methods will be problematic.
Furthermore, it does not support partial overrides compared to our model.

\begin{lstlisting}
\end{lstlisting}

\noindent  {\bf \csharp{} Explicit method implementation.}
Explicit method implementation is a special feature supported by \csharp{}. As described in \csharp{}
documentation~\cite{csharpdoc}, a class that implements an interface can explicitly implement a member of that
interface. When a member is explicitly implemented, it can only be accessed through an instance
of the interface. Explicit interface implementation allows the programmer to inherit two interfaces 
that share the same member names and give each interface member a separate implementation.

Explicit interface member implementations have two advantages.
Firstly, they allow interface implementations to be excluded 
from the public interface of a class. This is particularly useful when a class implements an internal 
interface that is of no interest to a consumer of that class or struct.
Secondly, they allow disambiguation of interface members with the 
same signature. However, there are two critical differences to \MIM{}:
(1) default implementations are not allowed in \csharp{} interfaces; 
(2) there are only one-level partial overrides (implementations).

\begin{lstlisting}
\end{lstlisting}

\subsection{Hierarchical Dispatch}
As we have discussed before, the mix of static and dynamic dispatch is particularly useful under certain circumstances,
yet there is little attention from related work. In the prototype-based language \self~\cite{Chambers1991}, inheritance is a basic feature.
It does not include classes but instead allow individual objects to inherit from (or delegate to) other objects. 
Although it is different from class-based languages, the multiple inheritance model is somehow similar. The \self{}
language supports multiple (object) inheritance in a clever way. It not only develops the new inheritance
relation \emph{prioritized parents}, but also adopts \emph{sender path tiebreaker rule} for method lookup, which is similar with our approach, using hierarchical information in method resolution, but in a prototype-based language setting.\haoyuan{Do they have updates?}


\subsection{Formalization Based on Featherweight Java}
Featherweight Java~\cite{Igarashi01FJ} is a minimal core calculus of Java language, proposed by Igarashi et. al\haoyuan{Need cite}. There are
many models built on
Featherweight Java, including FeatherTrait~\cite{Liquori08ftj}, Featherweight defenders~\cite{goetz12fdefenders}, Jx~\cite{Nystrom2004}, and so on.
It provides the standard model of formalizing a Java-like object-oriented language, and 
is easily extensible. In terms of formalization, they key novelty of our model is the use of 
static types as annotations along with various terms. As far as we know, this technique has not appear in literature before.








