\section{Related Work}
Here we describe related work in four parts. We discuss from mainstream popular multiple 
inheritance models to several specific models (C++ and \csharp) which are closest to our work. Then we 
discuss similar techniques used in \self. Finally we discuss the foundation and related work of our formalization.

\subsection{Mainstream Multiple Inheritance Models}
Multiple inheritance is an useful feature in object-oriented
programming, although it is difficult to model and can 
cause various problems (e.g. the diamond problem~\cite{Sak89dis,Singh1995}).  
There are many existing languages/models that support multiple 
inheritance~\cite{ellis1990annotated,scala-overview,bracha90mixin,scharli03traits,malayeri2009cz,csharpdoc,Moon1986,Flatt1998,Ancona2003}. 
The Mixin model~\cite{bracha90mixin} allows naming components 
that can be applied to various classes as reusable functionality units. However, the linearization (total ordering) of mixin 
inheritance cannot provide a satisfactory resolution in some cases and restricts the flexibility of mixin composition. 
Scala traits~\cite{scala-overview} are in fact linearized mixins and hence have the same problem as mixins.

Simplifying the mixins approach, traits~\cite{scharli03traits,Ducasse:2006:TMF:1119479.1119483} draw a
strong line between units of reuse and object factories. Traits act
as units of reuse, containing functionality code; while classes,
assembled from traits, act as object factories. Java 8
interfaces are closely related to traits: concrete method
implementations are allowed (via the \textbf{\texttt{default}}
keyword) inside interfaces and there are also proposals such as FeatherTrait Java~\cite{Liquori08ftj} 
for extending Java with traits. There are extensions~\cite{reppy2006foundation,Reppy:2007:MT:2394758.2394784} to 
the original trait model with various advanced features such as \emph{renaming}. As discussed in Section~\ref{sec:overview},
the renaming feature helps to solve the unintentional method conflicts problem, however, it breaks structural subtyping.
\bruno{Should talk
  about work on traits that support renaming.}\bruno{I remember Yanlin
  had a presentation with alot of related work on multiple
  inheritance. Many of those seem to be missing here}

\begin{comment}
There are also proposals for extending Java with traits. For example, 
FeatherTrait Java (FTJ) [14] extends FJ [13] with statically-typed traits, 
adding trait-based inheritance in Java. Except for few, mostly syntac- tic details, 
their work can be emulated with Java 8 interfaces. There are also extensions 
to the original trait model, with operations (e.g. renaming [18], which breaks 
structural sub- typing) that default methods and interfaces cannot model.
\end{comment}

Malayeri and Aldrich proposed a model CZ~\cite{malayeri2009cz} which
aims to do multiple inheritance without the diamond problem.
Inheritance is divided into two concepts: inheritance dependency and
implementation inheritance.  Using a combination of
\textbf{\texttt{requires}} and \textbf{\texttt{extends}}, a program
with diamond inheritance is transformed to one without
diamonds. Moreover, fields and multiple inheritance can coexist.
However untangling inheritance also untangles the class structure. In
CZ, not only the number of classes, but also the class hierarchy
complexity increases.

The above mentioned models/languages support multiple inheritance, and
they handle method conflicts in the same way, by simply disallowing
two methods with the same signature from two different units to
coexist. The ambiguity is resolved either by programmers explicitly or
by setting a linearised composition. This is the common drawback of
the above mentioned models versus our model.

\subsection{Resolving Unintentional Method Conflicts}
There are a few languages that partly realize the problem of
unintentional conflicts and provide some support for it.

\noindent {\bf C++ model.}
C++ supports a very flexible inheritance model and allows programmers to choose either static dispatch or dynamic dispatch for method lookup.
It allows unintentional conflicts and uses static dispatch to resolve them, as discussed in Section~\ref{sec:overview}. For example, given the following code
\begin{lstlisting}[language=Java]
class A { public: void m() {cout << "MA" << endl;} };
class B { public: void m() {cout << "MB" << endl;} };
class C : public A, public B { 
	void m() {cout << "MC" << endl;}
};
void func(A* a) { a->m(); }
int main() {
	C* c = new C();
	c->B::m();
	func(c); 
	return 0; //Running result: MB MA
}
\end{lstlisting}
The running result is $MB \; MA$, meaning that it uses static dispatch (looks at the static types) for method \lstinline|m()|. 
On calling \texttt{func(c)}, in spite that the dynamic type of \texttt{c} is class \texttt{C}, the method call still dispatches to 
\lstinline|A.m|.
However, we can alter the code a little bit with \textbf{\texttt{virtual}} methods and the result will be totally different:
\begin{lstlisting}[language=c++]
class A { public: virtual void m() {cout << "MA" << endl;} };
class B { public: virtual void m() {cout << "MB" << endl;} };
class C : public A, public B { 
    public: virtual void m() {cout << "MC" << endl;}
};
void func(A* a) { a->m(); }
int main() {
	C* c = new C();
	c->B::m();
	func(c); 
	return 0; //Running result: MB MC
}
\end{lstlisting}
Now the running result will be $MB \; MC$. With virtual methods, dynamic dispatch is used and 
method lookup algorithm will find the most specific method definition of $m$, namely \lstinline|C.m| at this time.
Although C++ support this flexibility, dynamic dispatching on
unintentional conflicting methods is problematic, as discussed in Section~\ref{sec:overview}.\bruno{why? give example of
  something that goes wrong?}
Furthermore, C++ does not support partial overrides compared to our
model.

\noindent  {\bf \csharp{} Explicit method implementation.}
Explicit method implementation is a special feature supported by
\csharp{}. As described in \csharp{} documentation~\cite{csharpdoc}, a
class that implements an interface can explicitly implement a member
of that interface. When a member is explicitly implemented, it can
only be accessed through an instance of the interface. Explicit
interface implementation allows the programmer to inherit two
interfaces that share the same member names and give each interface
member a separate implementation.

Explicit interface member implementations have two advantages.
Firstly, they allow interface implementations to be excluded 
from the public interface of a class. This is particularly useful when a class implements an internal 
interface that is of no interest to a consumer of that class or struct.
Secondly, they allow disambiguation of interface members with the 
same signature. However, there are two critical differences to \MIM{}:
(1) default implementations are not allowed in \csharp{} interfaces; 
(2) there is only one level of partial overrides (implementations).

\subsection{Hierarchical Dispatch}
As we have discussed before, although the mix of static and dynamic dispatch is 
particularly useful under certain circumstances, it has received little research attention. 
In the prototype-based language \self{}~\cite{Chambers1991}, inheritance is a basic feature.
It does not include classes but instead allow individual objects to inherit from (or delegate to) other objects. 
Although it is different from class-based languages, the multiple inheritance model is somehow similar. The \self{}
language supports multiple (object) inheritance in a clever way. It not only develops the new inheritance
relation with \emph{prioritized parents}, but also adopts \emph{sender path tiebreaker rule} for method lookup.
It specifies that ``\emph{if two slots with the same name are defined in equal-priority parents of the receiver, but only 
one of the parents is an ancestor or descendant of the object containing the method that is sending the message,
then that parent's slot takes precedence over the over parent's slot}''. 
Similar to our model, this sender path tiebreaker rule resolves ambiguities between unrelated abstractions. However,
it is used in a prototype-based language setting and it does not support method hierarchical overriding as \MIM{} does.

\subsection{Formalization Based on Featherweight Java}
Featherweight Java (FJ)~\cite{Igarashi01FJ} is a minimal core calculus of Java language, 
proposed by Igarashi et. al. There are many models built on Featherweight Java, 
including FeatherTrait~\cite{Liquori08ftj}, Featherweight defenders~\cite{goetz12fdefenders}, Jx~\cite{Nystrom2004}, 
Featherweight Scala~\cite{Cremet2006}, and so on.
FJ provides the standard model of formalizing a Java-like object-oriented language, and 
is easily extensible. In terms of formalization, the key novelty of our model is the use of 
static types as annotations along with various terms. As far as we
know, this technique has not appeared in the literature before. The static type annotation is of vital 
importance in our hierarchical dispatch algorithm.








