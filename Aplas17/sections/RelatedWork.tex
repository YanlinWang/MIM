\section{Related Work}

\subsection{Mainstream Multiple Inheritance Models}
Multiple inheritance is a useful feature in object-oriented programming world although it's difficult to model and can 
cause various problems (e.g. the diamond problem~\cite{Sak89dis,Singh1995}).  There are many existing languages/models that support multiple 
inheritance, either coming with multiple inheritance capability or added through evolution. The Mixin model allows naming components 
that can be applied to various classes as reusable functionality units. However, the linearization (total ordering) of mixin 
inheritance cannot provide a satisfactory resolution in some cases and restricts the flexibility of mixin composition. 

Simplifying the mixins approach, traits~\cite{scharli03traits} draw a strong line between units of reuse and object factories. 
Traits act as units of reuse, containing functionality code; while classes, assembled from traits, act as object factories. 

Java 8 interfaces are closely related to traits: concrete method implementations are allowed (via the \textbf{\texttt{default}} keyword) inside interfaces. 
The introduction of default methods opens the gate for various flavors of multiple inheritance in Java.

Malayeri and Aldrich proposed a model CZ~\cite{malayeri2009cz} which aims to do multiple inheritance without the diamond problem. 
Inheritance is divided into two concepts: inheritance dependency and implementation inheritance. 
Using a combination of \textbf{\texttt{requires}} and \textbf{\texttt{extends}}, a program with 
diamond inheritance is transformed to one without diamonds. Moreover, fields and multiple inheritance can coexist. 
However untangling inheritance also untangles the class structure. In CZ, not only the number of classes, but also 
the class hierarchy complexity increases. 

The above mentioned models/languages support multiple inheritance, and they handle method confliction in the same way: requiring 
programmers to explicitly resolve ambiguity and disallowing two methods with the same signature from two different libraries 
to co-exit. This is the common drawback of the above mentioned models versus our model.

\subsection{Resolving Unintended Method Confliction}
There are still a few languages that already realized the problem and explored a bit. We discuss them one by one.

\noindent {\bf C++ model.}
As discussed in Section~\ref{sec:overview}, C++ supports very flexible inheritance mode and allows programmers to choose different 
confliction resolution approaches via normal or virtual inheritance. Generally speaking, C++ model is much more complex 
and flexible than our approach. For example, given the following code
\begin{lstlisting}[language=Java]
class A { public: void m() {cout << "MA" << endl;}};
class B { public: void m() {cout << "MB" << endl;}};
class C : public A, public B { 
	void m() {cout << "MC" << endl;}
};
void func(A* a) { a->m(); }
int main() {
	C* c = new C();
	c->B::m();
	func(c); 
	return 0; //Running result: MB MA
}
\end{lstlisting}
The running result is $MB \; MA$, meaning that it uses static dispatch (looks at the static types) when doing method lookup. 
On calling \texttt{func(c)}, no matter the dynamic type of \texttt{c} is class \texttt{C}, the method call still dispatches to 
the method \texttt{m} in class \texttt{A} because of static dispatch.
However, we can alter the code a little bit with \textbf{\texttt{virtual}} methods and the result will be totally different:
\begin{lstlisting}[language=c++]
class A { 
	public: virtual void m() {cout << "MA" << endl;}
};
class B { 
	public: virtual void m() {cout << "MB" << endl;}
};
class C : public A, public B { 
    public: virtual void m() {cout << "MC" << endl;}
};
void func(A* a) { a->m(); }
int main() {
	C* c = new C();
	c->B::m();
	func(c); 
	return 0; //Running result: MB MC
}
\end{lstlisting}
Now the running result will be $MB \; MC$. With virtual method, dynamic dispatch is used and 
method lookup algorithm will find the most specific method definition of $m$, which is the definition 
in class $C$. Although C++ support this flexibility, the drawback compared to our model is
obvious: it does not support updating both $A.m$ and $B.m$ in class $C$. 

\begin{lstlisting}
\end{lstlisting}

\noindent  {\bf \csharp Explicit method implementation.}
Explicit method implementation is a special feature supported by \csharp. As described in \csharp
documentation~\cite{csharpdoc}, a class that implements an interface can explicitly implement a member of that
interface. When a member is explicitly implemented, it can only be accessed through an instance
of the interface. Explicit interface implementation allows the programmer to inherit two interfaces 
that share the same member names and give each interface member a separate implementation. 
Explicit interface member implementations have two advantages:
Explicit interface member implementations allow interface implementations to be excluded 
from the public interface of a class. This is particularly useful when a class implements an internal 
interface that is of no interest to a consumer of that class or struct.
Explicit interface member implementations allow disambiguation of interface members with the 
same signature. This is the the advantage that is similar to ours' model. However, there are two 
diffirencies: firstly, method (default) implementations are not allowed in \csharp interfaces; 
secondly, when handling unintended method confliction, multi-way method updating is not supported 
in \csharp.

\begin{lstlisting}
\end{lstlisting}

\subsection{Hirarchical Dispatch}
As we have discussed before, the mix of static and dynamic dispatch is needed by the literature 
pay little attention to this. In the prototype-based language \self~\cite{Chambers1991}, inheritance is a basic feature.
It does not include classes but instead allow individual objects to inherit from (or delegate to) other objects. 
Although it is different than class-based languages, the multiple inheritance model is somehow similar. The \self 
language support multiple (object) inheritance in a clever way. Not only the new inheritance
relation \emph{prioritized parents} is developed, but also the method lookup rule called 
\emph{sender path tiebreaker rule}, which is similar with our approach, using hirarchical 
information in method resolution, but in a prototype-based language setting.


\subsection{Formalization Based on FeatherweightJava}
FeatherweightJava~\cite{Igarashi01FJ} is a minimal core calculus of Java language, proposed by Igarashi et. al. Many models are based on 
FeatherweightJava, for example FeatherTrait~\cite{Liquori08ftj}, Featherweight defenders~\cite{goetz12fdefenders}, Jx~\cite{Nystrom2004}, as well as our model.
It provides the standard model of formalizing a Java-like object-oriented language and 
people can easily extends it to form a new language. In terms of formalization, they key novelty of our model is the use of 
static types as annotations along with various terms. As far as we know, this technique has not appear in literature before.








