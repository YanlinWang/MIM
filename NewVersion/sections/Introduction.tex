\section{Introduction}
Inheritance in Object-Oriented Programming (OOP) offers a mechanism
for code reuse. However many OOP languages are restricted to single
inheritance, which is less expressive and flexible than multiple
inheritance. Nevertheless, different flavours of multiple inheritance
have been adopted in some popular OOP languages. C++ has had 
multiple inheritance from the start. Scala adapts the ideas from traits~\cite{scharli03traits,Ducasse:2006:TMF:1119479.1119483,Liquori08ftj}
and mixins~\cite{bracha90mixin,Flatt1998,van1996encapsulation,Ancona2003,Hendler86} to offer a disciplined form of multiple inheritance. Java 8 
offers a simple variant of traits, disguised of interfaces with default methods~\cite{goetz12fdefenders}.

A reason why programming languages have resisted to multiple
inheritance in the past is that, as Cook~\cite{Cook1987} puts it, 
``\emph{multiple inheritance is good but there is no good way to do it}''.
One of the most sensitive and critical issues is perhaps the ambiguity
introduced by multiple inheritance. One case is the famous
\textit{diamond problem}~\cite{Sak89dis,Singh1995} (also known as ``fork-join inheritance''~\cite{Sak89dis}). 
In the diamond problem, inheritance allows
one feature to be inherited from multiple parent classes that share a
common ancestor. Hence
conflicts arise. The variety of strategies for resolving such conflicts
urges the occurrence of different multiple inheritance models,
including traits, mixins, CZ~\cite{malayeri2009cz}, and many others. Existing
languages and research have addressed the issue of diamond inheritance extensively. Other issues
including how multiple inheritance deals with state, 
have also been discussed quite extensively~\cite{classless,malayeri2009cz,stroustrup1995}.

In contrast to diamond inheritance, a second case of ambiguity
is \textit{unintentional method conflicts}~\cite{scharli03traits}. That is conflicting 
methods that do not actually refer to the same feature. 
In a nominal system, methods can be designed for different
functionality, but happen to have the same names (and signatures).
A simple example of this situation is two \lstinline{draw} methods that
are inherited from a deck of cards and a drawable widget. 
In such context, the two \lstinline{draw} methods have very different meanings, 
but they happen to share the same name.
%%This issue was proposed by the trait paper, so-called
When inheritance is used to compose these methods, a compilation 
error happens due to conflicts. However, unlike the diamond problem,
the conflicting methods have very different meanings and do not share a
common parent. We call such a case \textit{triangle inheritance}, in
analogy to diamond inheritance.

%Unintentional method conflicts are perhaps less common than the diamond
%problem. 

When unintentional method conflicts happen, they can have severe
effects in practice if no
appropriate mechanisms to deal with them are available. 
Unfortunately, such an issue has not received much formal study 
before. In practice, existing languages only provide limited support for
the issue. In most languages, the mechanisms available to deal with this problem are the same as the diamond
inheritance. However, this is often inadequate and can lead 
to tricky problems in practice. This is especially the case
when it is necessary to combine two large modules and their features,
but the inheritance is simply prohibited by a small conflict. 
As a workaround from the diamond inheritance side, it is possible to
define a new method in the child class to override those conflicting
methods. However, using one method to fuse two unrelated features
is clearly unsatisfactory. Therefore we need a better solution to keep both
features separately during inheritance, so as not to break
\emph{independent extensibility}~\cite{zenger05independentlyextensible}.

C++ and C\# do allow for two
unintentionally conflicting methods to coexist in a class. C\# allows
this by interface multiple inheritance and explicit method
implementations. But since C\# is a single inheritance language, 
it is only possible to \emph{implement} multiple interfaces (but not
multiple classes). %We will have more detailed discussion on this in Section~\ref{sec:relatedwork}.
C++ accepts triangle inheritance and
resolves the ambiguity by specifying the expected path by
\emph{upcasts}. However
neither the C\# or C++ approaches allow
such conflicting methods to be further overriden. 
% However, C++ has
% limited support for virtual methods with unintentional conflicts, and
% it will often throw errors when composing them. This is again
% unsatisfactory because virtual methods are pervasive in OOP and used 
% for code reuse and extensibility. A problem with the C++ approach is
% that programmers can only use either static or dynamic dispatching separately, but dealing
% with unintentional method conflicts seems to require a combination of
% both. 
Some other workarounds or approaches include delegation and
renaming/exclusion in the trait model. However renaming/exclusion 
can break the subtyping relation between a subclass and its parent.
This is not adequate for the class model commonly used in mainstream 
OOP languages, where the subclass is always expected to be a subtype 
of the parent class. 

%%Yet they still have various
%%drawbacks as we will discuss in Section~\ref{sec:overview}.
\bruno{The previous paragraph needs to be revised to better discuss the
  limitations of the C++ approach. Yanlin, Marco can you have a look
  at this?}


%Having tolerance for unintentional method conflicts does not mean to
%sacrifice extensibility, hence in contrast with static dispatch and
%dynamic dispatch, 

This paper proposes two mechanisms to deal with unintentional method
conflicts: \textit{hierarchical dispatching} and \emph{hierarchical
  overriding}. Hierarchical dispatching is inspired by the mechanisms in C++ and
provides an approach
to method dispatching, which combines static and dynamic
information. Using hierarchical dispatching, the method binder will look
at both the \emph{static type} and the \emph{dynamic type} of the
receiver during runtime. When there are multiple branches that cause
unintentional conflicts, the static type can specify one branch among
them for unambiguity, and the dynamic type helps to find the most
specific implementation. In that case, both unambiguity and
extensibility are preserved. The main novelty over existing work is 
the formalization of the essence of a hierarchical dispatching
algorithm, which (as far as we know) has not been formalized before. 
Furthermore, \textit{hierarchical overriding} allows method overriding to be applied
only to one branch of the class hierarchy. Hierarchical overriding 
adds expressive power that is not available in languages such as 
C++ or C\#. To present both ideas, we introduce a
formalized model \MIM{} in Section~\ref{sec:formalization} based on
Featherweight Java~\cite{Igarashi01FJ}, together with theorems and
proofs for type soundness. 

%%Our model can be viewed as
%%a generalization of the simplified trait model by providing additional support to
%%the triangle inheritance.

In summary, our contributions are:
\begin{itemize}
	\item \textbf{Formalization of a hierarchical dispatching algorithm} that integrates both the static type and dynamic type for method dispatch, and hence
	ensures unambiguity as well as extensibility in the presence
        of unintentional method conflicts.
	\item \textbf{Hierarchical overriding:} a novel notion that allows
          methods to override on individual branches of the class hierarchy.
	\item \textbf{\name:} a formalized model based on
          Featherweight Java, supporting the above features. 
          We provide the static and dynamic semantics, and prove the
          type soundness of the model.
	\item \textbf{Prototype implementation\footnote{Available in
              supplementary material attached to the submission.}:} a
          simple implementation of \MIM{} implemented in Scala.
          \bruno{Are the examples in the paper implemented in the interpreter?}
          \yanlin{new formalization has not been implemented yet.}
\end{itemize}

 