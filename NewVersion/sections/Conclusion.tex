\section{Conclusion}

\bruno{I think that the conclusion is not upto date. Please update it and let me know when ready.}
\yanlin{done}
This paper proposes \MIM{} as a new multiple inheritance model for unintentional method conflicts. 
Existing approaches we discussed before either do not support unintentional method conflicts or have to compromise between code reuse and type safety; the closest one (C++) do not support partial method override. Two key mechanisms in this paper are hierarchical dispatching and hierarchical overriding. For hierarchical dispatching, we mimic the method lookup algorithm as in C++ and provide the formal algorithm in \mbody{} in \MIM{}. It makes use of both dynamic type information and path information from either upcasts or parameters' information. 
Such an approach not only offers great code reuse like dynamic dispatch but also ensures unambiguity by our algorithm for method resolution. This paper also introduces hierarchical overriding that refines branches individually.
\MIM{} is formalized following the style of Featherweight Java and a few properties(theorems) of type-soundness are proved on \MIM{}. The prototype is implemented in Scala as a simple interpreter.

Our model can certainly be improved at some aspects. We do not formalize fields as in Featherweight Java for the reason of simplicity. And we do not formalize other common features including method overloading, casts, covariant return types, and so on, some of which are orthogonal in the design space. Moreover, we restrict that hierarchical overriding can only work on original methods. Potentially a looser condition can better support encapsulation and modularity with respect to code design as discussed in Section~\ref{sec:discussion} and could be a good future work.