\section{Conclusion}


This paper proposes \MIM{} as a formalized multiple inheritance model for
unintentional method conflicts. Previous approaches 
either do not support unintentional method conflicts, have to
compromise between code reuse and type safety, or do not fully support
overriding in the presence of unintentional conflicts. To deal with unintentional method conflicts we
introduce two key mechanisms: hierarchical dispatching and
hierarchical overriding. Hierarchical dispatching is inspired by the
mechanisms in C++. We provide a minimal formal model of hierarchical
dispatching in \MIM{}. Such algorithm makes use of both dynamic type
information and path information from either upcasts or parameters'
information. Such an approach not only offers great code reuse like
dynamic dispatch but also ensures unambiguity by our algorithm for
method resolution. Additionally we introduce \emph{hierarchical
  overriding} to allow conflicting methods in different branches to be
individually overriden.

\MIM{} is formalized following the style of
Featherweight Java and type-soundness is proved. A prototype language is
implemented in Scala as a simple interpreter. We believe that the formalization of
hierarchical dispatching features is general and
can be safely embedded to other OO models, so as to have support for the triangle
inheritance.

Our model can certainly be improved at some aspects. 
As discussed in Section~\ref{sec:discussion}, there are orthogonal and
non-orthogonal features that can potentially be added to the design space. 
The future work relates to loosening the model without giving up its soundness,
together with more exploration on fields in the multiple inheritance setting.
