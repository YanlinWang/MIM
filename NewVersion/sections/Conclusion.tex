\section{Conclusion}


This paper proposes \MIM{} as a new multiple inheritance model for
unintentional method conflicts. Previous approaches 
either do not support unintentional method conflicts, have to
compromise between code reuse and type safety, or do not fully support
overriding in the presence of unintentional conflicts. To deal with unintentional method conflicts we
introduce two key mechanisms: hierarchical dispatching and
hierarchical overriding. Hierarchical dispatching is inspired by the
mechanisms in C++. We provide a minimal formal model of hierarchical
dispatching in \MIM{}. Such algorithm makes use of both dynamic type
information and path information from either upcasts or parameters'
information. Such an approach not only offers great code reuse like
dynamic dispatch but also ensures unambiguity by our algorithm for
method resolution. Additionally we introduce \emph{hierarchical
  overriding} to allow conflicting methods in different branches to be
individually overriden. \MIM{} is formalized following the style of
Featherweight Java and type-soundness is proved. A prototype language is
implemented in Scala as a simple interpreter.

Our model can certainly be improved at some aspects. We do not
formalize fields as in Featherweight Java for the reason of
simplicity\bruno{I'm not sure that simplicity is a good excuse; 
following the trait model may be a better excuse.}. And we do not formalize other common features including
method overloading, casts, covariant return types, and so on, some of
which are orthogonal in the design space. Moreover, we restrict that
hierarchical overriding can only work on original methods. Potentially
a looser condition can better support encapsulation and modularity
with respect to code design as discussed in
Section~\ref{sec:discussion} and could be a good future work.
