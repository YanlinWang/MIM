\section{Formalization}~\label{sec:formalization}
In this section, we present the formal model called \MIM{} (\emph{\textbf{F}eatherweight \textbf{H}ierarchical \textbf{J}ava}), following similar style of 
Featherweight Java~\cite{Igarashi01FJ}. \MIM{} is a minimal core calculus that formalizes the core concept of hierarchical dispatch and overriding. The syntax, typing rules,
small-step semantics and theorems for type soundness are presented.

% \vspace{-2ex}
\subsection{Syntax}
The abstract syntax of \MIM{} interface declarations, method declarations, and expressions is given at Figure~\ref{fig:syntax}. The multiple
inheritance feature of \MIM{} is inspired by Java 8 interfaces, which support
method implementations via default methods. This feature is 
closely related to \emph{traits}. To demonstrate how
unintentional method conflicts are untangled in \MIM{}, we only focus on a small subset of the interface model. For example, all the methods declared
in an interface are either default methods or abstract methods. Default methods provide default implementations for methods. Abstract methods do not
have method body, and in fact they can be considered as normal methods that return nothing. Abstract methods can be overridden with future implementations.

\paragraph{Notations}
The metavariables $I, J$ range over interface names; $x$ ranges over variables; $m$ ranges over method names; $e$ ranges over expressions; and $M$ ranges over method declarations. Following Featherweight Java, we assume that the set of variables includes the special variable \kwthis, which cannot be used as the name of an argument to a method. We use the same
conventions as FJ; we write $\overline{I}$ as shorthand for a possibly empty sequence $I_1, ..., I_n$, which may be indexed by $I_i$; and write $\overline{M}$ as shorthand for $M_1 .. M_n$ (with no commas). We also abbreviate operations on pairs of sequences in the obvious way, writing $\overline{I} \; \overline{x}$ for $I_1 \; x_1, ..., I_n \; x_n$, where $n$ is the length of $\overline{I}$ and $\overline{x}$.

\paragraph{Interfaces}
In order to achieve multiple inheritance, an interface can have a set of 
parent interfaces, where such a set can be empty. The interface declaration $\interface{I}{I}{M}$ introduces an interface named $I$ with parent interfaces $\overline{I}$ and a suite of methods $\overline{M}$. The methods of $I$ may either override methods that are already defined in $\overline{I}$ or add new functionality special to $I$, we will illustrate this in more detail later.

\paragraph{Methods}
Original methods and hierarchically overriding methods share the same syntax in our model for simplicity.
The concrete method declaration $\method{I}{m}{I_x}{x}{J}{e}$ introduces a
method named $m$ with result type $I$, parameters $\overline{x}$ of
type $\overline{I_x}$ and the overriding target $J$. The body of the
method simply includes the returned expression $e$. Notably, we have introduced the
\kwoverride{} keyword to override an old implementation of the
method. If the interface that it overrides is exactly the enclosing
interface, then such a method is seen as originally defined; otherwise it is an overriding method. The definition
of abstract methods is done with $\absmethod{I}{m}{I_x}{x}{J}$, which is
similar to a concrete method but without the method body. 
For simplicity, overloading is not modelled for methods, which
implies that we can uniquely identify a method by its name.

\paragraph{Expressions \& Values}
Expressions can be standard constructs such as variables, method
invocation, or object creation, together with cast expressions. 
Object creation is represented by $\new I$. Fields and primitive types are not modelled in \MIM{}. 
The casts are merely safe up-casts, and in fact, they can be viewed as
annotated expressions, where the annotation indicates its static type.
The coexistence of static and dynamic types is the key to hierarchical dispatch.
A value
``$(I)\new{J}$''
is the final result of multiple reduction steps evaluating an
expression.

For simplicity, \name{} does not formalize statements like assignments and so on because they are orthogonal features.
A program consists of a list of interface declarations, plus a single expression.

\begin{figure*}[t]
\saveSpaceFig
\begin{displaymath}
\begin{array}{l}
\begin{array}{llrl}
\text{Interfaces}   & IL & \Coloneqq & \interface{I}{I}{M} \\
\text{Methods}      & M  & \Coloneqq & \method{I}{m}{I_x}{x}{J}{e}  \mid
									   \absmethod{I}{m}{I_x}{x}{J} \\
\text{Expressions}  & e  & \Coloneqq & x \mid
e.m(\overline{e}) \mid
\new{I} \mid \; (I)e \\
\text{Context}      & \Gamma & \Coloneqq & \overline{x}:\overline{I} \\
\text{Values}       & v & \Coloneqq & (I) \new{J} \\
%%\\
%%\text{Interface names} & I, J, K & & \\
%%\text{Method names} & m & & \\
%%\text{Variable names} & x & &
\end{array}
\end{array}
\end{displaymath}
\caption{Syntax of \name{}.}\label{fig:syntax}
\saveSpaceFig
\end{figure*}

\subsection{Subtyping and Typing Rules}
\paragraph{Subtyping}
The subtyping of \MIM{} consists of only a few rules shown at the top of Figure~\ref{fig:typingrules}.
In short, subtyping relations are built from the inheritance in interface
declarations. Subtyping is both symmetric, reflexive and transitive.

\paragraph{Type-checking}
Details of type-checking rules are displayed at the bottom of Figure~\ref{fig:typingrules}, including expression
typing, well-formedness of methods and interfaces. As a convention, an environment
$\Gamma$ is maintained to store the types of variables, together with
the self-reference $\kwthis$.
% The three rules for method invocation, \textsc{(T-Invk)}, \textsc{(T-PathInvk)} and \textsc{(T-SuperInvk)}
% are very similar, in the sense that they all check the type of the specific method, by using
% an auxiliary function \mtype. \mtype{} is the function for looking up method types, which we will
% illustrate later in Section~\ref{subsec:auxdefs}. After the method
% type is obtained, they all check that the arguments and the receiver
% have compatible types. Additionally, \textsc{(T-PathInvk)} requires the receiver to be the subtype of the specified
% path type, and \textsc{(T-SuperInvk)} checks if the enclosing type directly extends the specified super type.

\textsc{(T-Invk)} is the typing rule for method invocation.
Naturally, the receiver and the arguments are required to be well-typed.
$\mbody$ is our key function for method lookup, based on the
hierarchical dispatching algorithm. The formal definition will be introduced in
Section~\ref{subsec:auxdefs}. To overview,
the hierarchical invocation $\mbody(m, I, J)$ means ``finding most specific $m$
above $I$ and along path $J$'', and usually $I$ denotes the dynamic type while $J$
denotes the static type. In \textsc{(T-Invk)}, as the compilation should not be aware
of the dynamic type, it only requires that invoking $m$ is valid for the static type of the
receiver. The result of $\mbody$ contains the interface that provides the most specific implementation,
the parameters and the return type. We use underscore for the return expression, to imply that an empty return expression
from an abstract method is also acceptable.
Note that we accept implicit upcasts on the arguments.
% \textsc{(T-PathInvk)} is the typing judgement for a path invocation. Besides the conditions of \textsc{(T-Invk)}, \textsc{(T-PathInvk)} requires the type of receiver to be the subtype of the specified path type. 
% and \textsc{(T-SuperInvk)} checks if the enclosing type directly extends the specified super type.

\textsc{(T-New)} is the typing rule for object creation $\new{I}$. The auxiliary function $\canInstantiate(I)$ checks whether interface $I$ can be instantiated or not, formalized in Section~\ref{subsec:auxdefs}. Since triangle inheritance accepts conflicting branches to coexist, the check requires that the most specific method is concrete for each method on each branch.

\textsc{(T-Method)} is more interesting since a method can either be an original method or a hierarchical overriding, though
they share the same syntax and method typing rule. $\mostSpecific(m, I, J)$ is a fundamental function from Section~\ref{subsec:auxdefs},
used to find ``the most specific interfaces that are above $I$ and along path $J$, and originally defines $m$''. By ``most specific interfaces'',
it implies that the inherited super types are excluded. Thus the condition $\mostSpecific(m, I, J) = \{J\}$ indicates the characteristics of a hierarchical overriding: it must override an original method; the overriding is direct and there does not exist any other original $m$ in between.
Then $\mbody(m, J, J)$ provides the type of the original method, so hierarchical overriding has to preserve the type. Finally the return expression
needs to be type-checked, and there is also an implicit cast to the declared return type. For the definition of an original method, $I$ equals $J$ and the rule is straightforward. \textsc{(T-AbsMethod)} is a similar rule but works on abstract method declarations.

\textsc{(T-Intf)} defines the typing rule on interfaces. The first condition is obvious, namely its methods need to be well checked. The third
condition checks whether the overriding between original methods preserves typing. In the condition we again use some helper functions from Section~\ref{subsec:auxdefs}. $I[m\ \kwoverride\ I]$ is defined if $I$ originally defines $m$, and $\canOverride(m, I, J)$ checks whether $I.m$ has the same type as $J.m$. Generally the preservation of method type is required for any super type $J$ and any method $m$.

The second condition of \textsc{(T-Intf)} is more complex, and is the key to type soundness. Unlike C++ which rejects on ambiguous calls,
\MIM{} rejects on the definition of interfaces when they reach a diamond. Consider the case when the second condition is broken: $\mbody(m, J, J)$
is defined but $\mbody(m, I, J)$ is undefined for some $J$ and $m$. It indicates that $m$ is available and unambiguous from the perspective of $J$,
but is ambiguous to $I$ on branch $J$. That means there are multiple overriding paths of $m$ from $J$ to $I$, which form a diamond. Hence rejecting
that case meets our expectation.

Finally, rule \textsc{(T-Anno)} is the typing rule for a cast expression. By the rule, only upcasts are valid.

\paragraph{Examples for mbody: TODO}

\begin{figure*}
\saveSpaceFig

\begin{mathpar}
	\framebox{$ I <: J $} \hspace{.5in} \subid \\
	\subtrans \hspace{.5in} \subextends \\
	
	\framebox{$ \judgeewf \Gamma {e:I} $} \hspace{.5in}
	\tvar \\
	\tinvk \\
	% \tpathinvk \\
	% \tsuperinvk \\
	% \tstaticinvk  \\
	\tnew \\
	\tmethod \\
	\tabsmethod \\
	\tintf \\
	\tanno
\end{mathpar}
\saveSpaceFig
\caption{Typing and subtyping of \name{}.}
\label{fig:typingrules}
\end{figure*}

\subsection{Small-step Semantics and Congruence}
Figure~\ref{fig:smallstep} defines the small-step semantic and
congruence rules of \MIM{}. When evaluating an expression, they
are invoked recursively and alternately, and produce a single value in
the end. \haoyuan{we need to be consistent on paragraph upper/lower case.}

\paragraph{Semantic Rule} \textsc{(S-Invk)} is the only computation rule we need for method invocation.
As a small-step rule and by congruence, it assumes that the receiver and the arguments are already values.
Specifically, the receiver $(J)\new{I}$ indicates the dynamic type $I$
together with the static type $J$. Therefore $\mbody(m, I, J)$ carries out hierarchical dispatch, acquires
the types, the return expression $e_0$ and the interface $I_0$ which provides the most specific method.
Here we use $e_0$ to imply that the return expression is forced to be non-empty because it requires a concrete implementation. Now the
rule reduces method invocation to $e_0$ with substitution.
Parameters are substituted with arguments, and \lstinline|this| reference is substituted with the receiver,
and in the meanwhile implicit casts are applied, so that static types are recorded via annotations. Finally
there is again an annotation from the return type.

\paragraph{Congruence Rules} \textsc{(C-Receiver)}, \textsc{(C-Args)} and \textsc{(C-FReduce)} are natural congruence rules
on receivers, arguments and cast-expressions, respectively. \textsc{(C-StaticType)} automatically adds an annotation $I$ to the new
object $\new{I}$. \textsc{(C-AnnoReduce)} merges nested upcasts into a single upcast with the outermost type.

\begin{figure*}[t]
\saveSpaceFig
\begin{mathpar}
	% \sinvk \\
	\spathinvk \\
	% \ssuperinvk \\
	% \sstaticinvk \\
	\creceiver \hspace{.5in}
	% \cpathreceiver \\
	\cargs \\
	% \cpathargs \\
	% \csuperargs \\
	% \cstaticargs \\
	\cstatictype \\
	\cfreduce \\
	\cannoreduce
\end{mathpar}
\caption{Small-step semantics.}\label{fig:smallstep}
\saveSpaceFig
\end{figure*}

\begin{comment}
\begin{figure*}[t]
\begin{mathpar}
\end{mathpar}
\caption{Congruence.}\label{fig:congruence}
\end{figure*}
\end{comment}


\subsection{Properties}

We present the type soundness of the model by a few theorems below, following the standard technique of
subject reduction and progress proposed by~\cite{Wright1994}. The proof, together with some lemmas, are
in Appendix. The type soundness states that if an expression is well-typed, then after many reduction
steps it must reduce to a value, and its annotation is the same as the static type of the original expression.

\begin{theorem}[Subject Reduction]~\label{theorem_subject}
If $\judgeewf \Gamma {e : I}$ and $e \rightarrow e'$, 
then $\judgeewf \Gamma {e' : I}$.
\end{theorem}
\begin{proof}
See Appendix~\ref{appendix_proof}.
\end{proof}

\begin{theorem}[Progress]~\label{theorem_progress}
Suppose $e$ is a well-typed expression, if $e$ includes 
$\left((J)\emph{\kwnew}\;I()\right).m(\overline{v})$ as a sub-expression, then $\mbody(m, I, J) = (I_0, \overline{I_x} \; \overline{x}, I_e\; e_0)$ and $\num{\overline{x}} = \num{\overline{v}}$ for some $I_0$, $\overline{I_x}$, $\overline{x}$, $I_e$ and $e_0$.
\end{theorem}
\begin{proof}
See Appendix~\ref{appendix_proof}.
\end{proof}

\begin{theorem}[Type Soundness]~\label{theorem_soundness}
If $\judgeewf \o {e : I}$ and $e \to^* e'$ with $e'$ a normal form, then $e'$ is 
a value $v$ with $\judgeewf \o {v:I}$.
\end{theorem}
\begin{proof}
Immediate from Theorem~\ref{theorem_subject} and Theroem~\ref{theorem_progress}.
\end{proof}
Note that in Theorem~\ref{theorem_progress}, ``$\#(\overline{x})$'' denotes the length of
$\overline{x}$.

Our theorems are stricter than those of Featherweight Java~\cite{Igarashi01FJ}. Specifically, in both term substitution (a lemma in Appendix)
and subject reduction, the type may go down in FJ, but in \MIM{} it is unchanged. The difference is due to
subtyping; the parameter or \lstinline|this| reference requires some type $I$, but a value with a more specific type,
which is subtype of $I$, can be passed. However, as \MIM{} keeps track of the static types, the annotations in the
substitution actually ensures the type to be unchanged.