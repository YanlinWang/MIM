\documentclass{llncs}
\pagestyle{plain}

\usepackage{listings}

\usepackage{xspace}
\usepackage{multicol}
\usepackage{microtype}%if unwanted, comment out or use option "draft"
%\usepackage[table,xcdraw]{xcolor} -- clash with preamble.tex
\usepackage{color}
% \usepackage{amsthm}
% \usepackage{amsmath}
\usepackage{stmaryrd}
\usepackage{graphicx}
\usepackage{amssymb}
\usepackage{fancyvrb}
\usepackage{url}
%\usepackage{pstricks,pst-node,pst-tree} -- clash with preamble.tex
\usepackage{bbm}
%\usepackage{pgf} -- clash with preamble.tex
\usepackage{multirow}
\usepackage{enumitem}

\usepackage{verbatim}
\usepackage{graphicx}
\usepackage{wrapfig}
\usepackage[normalem]{ulem}

\usepackage[T1]{fontenc}
\usepackage[scaled=0.85]{beramono}
% \usepackage{mathpartir}
\usepackage[utf8]{inputenc}
\usepackage{flushend}
\usepackage{lmodern}

\let\proof\relax
\let\endproof\relax
\let\mathpar\relax

%%\newcommand\bruno[1]{\authornote{bruno}{red}{#1}}

\newcommand{\updates}{\textbf{updates}}
\newcommand{\MIM}{\textbf{MIM}} %multiple inheritance model
\newcommand{\dispatch}{hierarchical dispatch}
\newcommand{\dispatchnameit}{\textit{hierarchical dispatch}}
\newcommand{\dispatchname}{\textbf{hierarchical dispatch}}
\newcommand{\dispatchnamecaptical}{\textbf{Hierarchical Dispatch}}
\newcommand{\csharp}{C$\#\;$}
\newcommand{\self}{\textbf{SELF$\;$}}
\newcommand{\this}{\textbf{\texttt{this}}}

\newcommand{\red}[1]{\textcolor{red}{#1}}
% \newcommand{\keyword}[1]{\texttt{#1}}
% \newcommand{\subtype}   {<:}
%%%%%%%%%%%%%%%%%%%%%%%%%%%%%%%%%%%%%%%%%% syntax.tex begin %%%%%%%%%%%%%%%%%%%%%%
\newcommand{\kwinterface}{\keyword{interface}}
\newcommand{\kwextends}{\keyword{extends}}
\newcommand{\kwreturn}{\keyword{return}}
\newcommand{\kwoverride}{\keyword{override}}
\newcommand{\kwsuper}{\keyword{super}}
\newcommand{\kwthis}{\keyword{this}}
\newcommand{\kwnew}{\keyword{new}}
\newcommand{\kwtrue}{\keyword{true}}
\newcommand{\kwfalse}{\keyword{false}}


\newcommand{\mtype}{\keyword{mtype}}
\newcommand{\ext}{\keyword{ext}}
\newcommand{\definedin}{\keyword{definedin}}
\newcommand{\collectMethods}{\keyword{collectMethods}}
\newcommand{\mbody}{\keyword{mbody}}
\newcommand{\Undefined}{\keyword{Undefined}}
\newcommand{\Error}{\keyword{Error}}
\newcommand{\needed}{\keyword{needed}}
\newcommand{\methods}{\keyword{methods}}
\newcommand{\only}{\keyword{only}}
\newcommand{\pathcheck}{\keyword{pathcheck}}
\newcommand{\mostSpecific}{\keyword{mostSpecific}}
\newcommand{\mostSpecificOverride}{\keyword{mostSpecificOverride}}
\newcommand{\overrideSet}{\keyword{overrideSet}}
\newcommand{\prune}{\keyword{prune}}
\newcommand{\update}{\keyword{update}}
\newcommand{\dom}{\keyword{dom}}

\newcommand{\angl}[1]{
    \langle #1 \rangle
}

\newcommand{\num}[1]{
    \#(#1)
}

\newcommand{\new}[1]{
    \kwnew \; #1()
}

\newcommand{\interface}[3]{
  \kwinterface \; #1 \; \kwextends \; \overline{#2} \; {\{} \overline{#3} {\}}
}
\newcommand{\method}[6]{
  #1 \; #2 (\overline{#3} \; \overline{#4}) \; \kwoverride \; #5 \; {\{} \kwreturn \; #6 ; {\}}
}

\newcommand{\subid} {
\inferrule* [right=]
    {}
    {I \subtype I}
}
\newcommand{\subtrans} {
\inferrule* [left=]
    {I \subtype J \\ J \subtype K}
    {I \subtype K}
}

\newcommand{\subextends} {
\inferrule* [right=]
    {\interface{I}{I}{M}}
    {\forall I_i \in \overline{I}, I \subtype I_i}
}

\newcommand{\tvar} {
\inferrule* [left=(T-Var)]
    {}
    {\judgeewf \Gamma x:\Gamma(x)}
}

\newcommand{\tinvk} {
\inferrule* [left=(T-Invk)]
    {  \judgeewf \Gamma {e_0:I_0}
    \\ \mtype(m, I_0) = \overline{J} \to I
    \\ \judgeewf \Gamma \overline{e}:\overline{I}
    \\ \overline{I} \subtype \overline{J}
    }
    {\judgeewf \Gamma e_0.m(\overline{e}):I}
}

\newcommand{\tpathinvk} {
\inferrule* [left=(T-PathInvk)]
    {  \judgeewf \Gamma {e_0:I_0}
    \\ I_0 \subtype J_0
    \\ \mtype(m, J_0) = \overline{J} \to I
    \\ \judgeewf \Gamma \overline{e}:\overline{I}
    \\ \overline{I} \subtype \overline{J}
    }
    {\judgeewf \Gamma e_0.J_0::m(\overline{e}):I}
}

\newcommand{\tsuperinvk} {
\inferrule* [left=(T-SuperInvk)]
    {  \ext(\Gamma(this), J_0)
    \\ \mtype(m, J_0) = \overline{J} \to I
    \\ \judgeewf \Gamma \overline{e}:\overline{I}
    \\ \overline{I} \subtype \overline{J}
    }
    {\judgeewf \Gamma {\kwsuper.J_0::m(\overline{e})}:I}
}

\newcommand{\tnew} {
\inferrule* [left=(T-New)]
    {\interface{I}{I}{M}}
    {\judgeewf \Gamma \new{I}:I}
}

\newcommand{\tmethod} {
\inferrule* [left=(T-Method)]
    {  I <: J \\
    \mostSpecific(m, I, J) = \{J\} \\
    \mtype(m, J) = \overline{I_X} \to I_E \\
        \judgeewf {\overline{x}:\overline{I_X}, this:I  } {e_0:I_0} \\
        I_0 \subtype I_E
    %\\ \definedin(m, J) % m is (directly or indirectly) defined in J
    }
    {\method{I_E}{m}{I_X}{x}{J}{e_0} \text{ OK IN } I}
}

% \newcommand{\tintf} {
% \inferrule* [left=(T-Intf)]
%     {  \overline{I} \text{ OK}
%     \\ \forall m \in \collectMethods(I), \mbody(m, I) \neq \Undefined
%     }
%     { \interface{I}{I}{M} \text{ OK }}
% }
\newcommand{\tintf} {
\inferrule* [left=(T-Intf)]
    {  \overline{I} \text{ OK}
    \\ \overline{M} \text{ OK IN } I
    \\ \forall J >: I \text{ and } m, \mtype(m, J) \text{ is defined} \Rightarrow \mbody(m, I, J) \text{ is defined}
    }
    { \interface{I}{I}{M} \text{ OK }}
}


\newcommand{\sinvk} {
\inferrule* [left=(S-Invk)]
{ \mbody(m, I, J) = (I_0, \overline{I_X} \; \overline{x}, I'_E \; e_0)}
{\left(\angl{J}\new{I}\right).m(\overline{\angl{I_E}} \overline{e}) \to
    \angl{I'_E}[\overline{\angl{I_X}} \overline{e}/\overline{x}, \angl{I_0}\new{I}/\kwthis]e_0}
}

\newcommand{\spathinvk} {
\inferrule* [left=(S-PathInvk)]
{\mbody(m, I, K) = (I_0, \overline{I_X} \; \overline{x}, I'_E \; e_0)}
{\left(\angl{J}\new{I}\right).K::m(\overline{\angl{I_E}} \overline{e}) \to
    \angl{I'_E}[\overline{\angl{I_X}} \overline{e}/\overline{x}, \angl{I_0}\new{I}/\kwthis]e_0}
}

\newcommand{\ssuperinvk} {
\inferrule* [left=(S-SuperInvk)]
{\mbody(m, K, K) = (I_0, \overline{I_X} \; \overline{x}, I'_E \; e_0)}
{\kwsuper.K::m(\overline{\angl{I_E}} \overline{e}) \to
    \angl{I'_E}[\overline{\angl{I_X}} \overline{e}/\overline{x}, \angl{I_0}\new{I}/\kwthis]e_0}
}

\newcommand{\deff} {
\begin{displaymath}
    \begin{array}{l}
        \begin{array}{llrl}
        \text{Annotated Expressions}   & f & \Coloneqq & e \mid \angl{I}e
        \end{array}
    \end{array}
\end{displaymath}
}

\newcommand{\creceiver} {
%----------- v1.0 ---------------
%\inferrule* [left=(C-Receiver)]
%{e \to e'}
%{e.m(\overline{e}) \to e'.m(\overline{e})}
%}
\inferrule* [left=(C-Receiver)]
{e_0 \to e'_0}
{e_0.m(\overline{e}) \to e'_0.m(\overline{e})}
}

\newcommand{\cpathreceiver} {
\inferrule* [left=(C-PathReceiver)]
{e_0 \to e'_0}
{e_0.K::m(\overline{e}) \to e'_0.K::m(\overline{e})}
}

\newcommand{\cargs} {
%---------- v1.0 ---------------
%\inferrule* [left=(C-Args)]
%{e_i \to e_i'}
%{e.m(..., e_i, ...) \to e.m(..., e_i', ...)}
%}
%---------- v2.0 ---------------
%\inferrule* [left=(C-Args)]
%{e_1 \to f}
%{\angl{J}\new{I}.m(<\overline{E}>\overline{e_0}, e_1, \overline{e_2})
%\to
%\angl{J}\new{I}.m(<\overline{E}>\overline{e_0}, f, \overline{e_2})}
%}
%---------- v3.0 ---------------
\inferrule* [left=(C-Args)]
{e \to e'}
{e_0.m(\ldots,e,\ldots)
\to
e_0.m(\ldots,e',\ldots)}
}

\newcommand{\cpathargs} {
\inferrule* [left=(C-PathArgs)]
{e \to e'}
{e_0.I::m(\ldots,e,\ldots)
\to
e_0.I::m(\ldots,e',\ldots)}
}

\newcommand{\csuperargs} {
\inferrule* [left=(C-SuperArgs)]
{e \to e'}
{super.I::m(\ldots,e,\ldots)
\to
super.I::m(\ldots,e',\ldots)}
}

\newcommand{\cstatictype} {
\inferrule* [left=(C-StaticType)]
{}
{\new{I} \to \; \angl{I}\new{I}}
}

\newcommand{\cfreduce} {
\inferrule* [left=(C-FReduce)]
{e \to e' \\ e \neq \new{J}}
{\angl{I}e \to \; \angl{I}e'}
}

\newcommand{\cannoreduce} {
\inferrule* [left=(C-AnnoReduce)]
{}
{\angl{I}(\angl{J} \new K) \to \angl{I} \new K}
}
%%%%%%%%%%%%%%%%%%%%%%%%%%%%%%%%%%%%%%%%%% syntax.tex end %%%%%%%%%%%%%%%%%%%%%%%%

\input{predef/others/preamble.tex}
\input{predef/macros/common_pl_macros.tex}
\input{predef/f-and/base.tex}
\input{predef/f-and/orthogonality.tex}
\input{predef/f-and/wellformedness.tex}
\input{predef/f-and/subtyping.tex}
\input{predef/f-and/select.tex}
\input{predef/f-and/restrict.tex}
\input{predef/f-and/typing.tex}

\newcommand{\authornote}[3]{{\color{#2} {\sc #1}: #3}}
\newcommand{\authorText}[2]{{\color{#1}#2}}

 \newcommand\bruno[1]{\authornote{bruno}{red}{#1}}
 \newcommand\yanlin[1]{\authornote{yanlin}{purple}{#1}}
 \newcommand\marco[1]{\authornote{marco}{blue}{#1}}
 \newcommand\marcoT[1]{\authorText{blue}{#1}}
 \newcommand\haoyuan[1]{\authornote{haoyuan}{orange}{#1}}

%\newcommand\bruno[1]{}
%\newcommand\yanlin[1]{}
%\newcommand\marco[1]{}
%\newcommand\haoyuan[1]{}

\newcommand\saveSpaceFig{\vspace{-2ex}}

\lstset{ %
	language=Java,                % choose the language of the code
	columns=flexible,
	lineskip=-1pt,
	basicstyle=\ttfamily\small,       % the size of the fonts that are used for the code
	numbers=none,                   % where to put the line-numbers
	numberstyle=\ttfamily\tiny,      % the size of the fonts that are used for the line-numbers
	stepnumber=1,                   % the step between two line-numbers. If it's 1 each line will be numbered
	numbersep=5pt,                  % how far the line-numbers are from the code
	backgroundcolor=\color{white},  % choose the background color. You must add \usepackage{color}
	showspaces=false,               % show spaces adding particular underscores
	showstringspaces=false,         % underline spaces within strings
	showtabs=false,                 % show tabs within strings adding particular underscores
	morekeywords={var,updates},
	%  frame=single,                   % adds a frame around the code
	tabsize=2,                  % sets default tabsize to 2 spaces
	captionpos=none,                   % sets the caption-position to bottom
	breaklines=true,                % sets automatic line breaking
	breakatwhitespace=false,        % sets if automatic breaks should only happen at whitespace
	title=\lstname,                 % show the filename of files included with \lstinputlisting; also try caption instead of title
	escapeinside={(*}{*)},          % if you want to add a comment within your code
	keywordstyle=\ttfamily\bfseries,
	aboveskip=0pt,
	belowskip=0pt
	% commentstyle=\color{Gray},
	% stringstyle=\color{Green}
}

\begin{document}

% \title{Untangling Unintended Conflicts with Hierarchical Dispatch}
% \title{FHJ: A Formal Model for Resolving Unintended Conflicts in Multiple Inheritance}
\title{FHJ: A Formal Model for Hierarchical Dispatching and Overriding}
%\subtitle{\InterfaceBased Programming for the Masses}

% \authorinfo{NoAuthor}
%            {The University of Hong Kong, China}
% %           {\{ylwang,hyzhang,bruno\}@cs.hku.hk}
% %\authorinfo{Marco Servetto}
% %          {Victoria University of Wellington, New Zealand}
% %          {marco.servetto@ecs.vuw.ac.nz}
\author{Authors Omitted}

\institute{}

\maketitle

\vspace{-4ex}

\begin{abstract}
Multiple inheritance is a valuable feature for Object-Oriented
Programming. However, it is also tricky to get right, as illustrated by
the extensive literature on the topic. A key issue 
is the \emph{ambiguity} arising from inheriting multiple parents,
which can have conflicting methods and fields. 
Numerous existing work provides solutions for 
conflicts which arise from \emph{diamond inheritance}: i.e.
conflicts that arise from implementations sharing a common 
ancestor. However, most mechanisms are inadequate to deal 
with \emph{unintentional method conflicts}: conflicts which 
arise from two unrelated methods that happen to share the same name
and signature. 

\begin{comment}
One of the most promising 
approaches to multiple inheritance is the \emph{trait} model. Traits offer a
restricted model of multiple inheritance that is easy to reason and 
have many elegant properties. Traits have good support for method 
conflicts which arise from \emph{diamond inheritance}:
conflicts that arise from method implementations sharing a common 
ancestor. However, the mechanisms of traits are inadequate to deal 
with \emph{unintentional method conflicts}: conflicts which 
arise from two unrelated methods that happen to share the same name
and signature. 
\end{comment}

This paper presents a new model called \emph{\textbf{F}eatherweight
  \textbf{H}ierarchical \textbf{J}ava} (\name{}) that deals with
unintentional method conflicts.  In our new model, which is partly
inspired by C++, conflicting methods arising from unrelated methods
can coexist in the same class, and \emph{hierarchical dispatching}
supports unambigous lookups in the presence of such conflicting
methods.  To avoid ambiguity, hierarchical information is employed in
method dispatching, which uses a combination of static and dynamic
type information to choose the implementation of a method at run-time.
Furthermore, unlike all existing inheritance models, our model
supports \emph{hierarchical method overriding}: that is, methods can
be \emph{independently overridden} along the multiple inheritance
hierarchy. We give illustrative examples of our language and features
and formalize \name{} as a minimal Featherweight-Java style calculus.

\begin{comment}
Furthermore we discuss similarities and differences to 


What ensures unambiguity is the use of information about
the class hierarchy.

\name{} is partly inspired by the method resolution semantics
of C++, but it also incoorporates ideas from the trait model and Java
8's default methods. We discuss the similarities
and differences with and also advantages and disadvantages. 
\end{comment}

\end{abstract}


\section{Introduction}

\begin{itemize}
	\item problem: Unintended method confliction in Multiple inheritance
	\item Existing approaches or models and Their drawbacks
	\item New features (update)
	\item Our contributions
\end{itemize}

In multiple inheritance, naming conflicts often occurs. Among these conflicts, some are real conflicts which needs 
explicit resolve by programmers, however, there are cases where accidental naming conflicts occurs, where the conflicting
methods have completely different meaning/domain which just share the same name. 

Existing OOP models have taken care of the first case intensively. However, few of them supports 
unintended method confliction well. Trait and other 
mainstream OO models do not allow unintended methods confliction to co-exist. 
SELF~\cite{} uses \emph{sender path tiebreaker rule} to automatically resolve 
ambiguities that are almost certainly caused by accidental naming conflicts. C++ allows methods with the same signature 
co-exist in a class via inheritance and programmers can use $::$ operator to select the method wanted. 
However none of them allows refining these unintended conflicting methods in subclasses. We propose a calculus that 
deals with unintended method confliction and meanwhile allows refining these methods. 

Contributions:
\begin{itemize}
    \item A multiple inheritance model formalized as \MIM.
    \item Novel notion \updates for method path updating.
    \item Implementation of a simple typechecker and evaluator in Scala.
\end{itemize}
\section{A Running Example: DrawableDeck}

This section illustrates the features of our \MIM{} model for resolving unintentional method
conflicts. As mentioned before, such a case arises when two inherited methods happen to have the
same signature, but with different semantics and functionalities. This could be quite troublesome
to programmers that use multiple inheritance. Below we illustrate with a running example called \lstinline|DrawableDeck|.
Note that we use Java-like syntax throughout the paper, and all types are defined with the keyword ``\lstinline|interface|'', which
supports multiple inheritance. Since \MIM{} is designed to be simple, we do not allow abstract methods, that is every method
is required to have a body for its implementation. In that case, interfaces can be directly instantiated by the keyword ``\lstinline|new|''
to create an object.

\subsection{Problem: Unintentional Method Conflicts}

Suppose that two components \lstinline|Drawable| and \lstinline|Deck| have been developed in a system.
\lstinline|Drawable| defines an interface for graphics that can be drawn, which includes a method called \lstinline|draw()|
for visual display. While interface \lstinline|Deck| represents a deck of cards, and supports several operations, like
\lstinline|draw()| for drawing a card from the deck.

\vspace{3pt}\begin{lstlisting}
interface Deck {
  void draw() { // draws a card from the Deck
    Stack<Card> cards = this.getStack();
    if (!cards.isEmpty()) {
      Card card = cards.pop();
      ...
    }
  }
}

interface Drawable {
  void draw() { // draws something on the screen
    JFrame frame = new JFrame("Canvas");
    frame.setVisible(true);
    ...
  }
}
\end{lstlisting}\vspace{3pt}
Note that both methods have \lstinline|void| return type (we will not formalize
\lstinline|void| in our calculus afterwards; here is only for illustration). In \lstinline|Deck|, \lstinline|draw()| tries to get the cards as a stack, pops
out the top card, and so on. While in \lstinline|Drawable|, \lstinline|draw()|
creates a blank canvas using \lstinline|JFrame|. Now, a programmer is designing a
card game with GUI. He may want to draw a deck on the screen, so he defines a drawable
deck using multiple inheritance:

\vspace{3pt}\begin{lstlisting}
interface DrawableDeck extends Drawable, Deck {
  ...
} 
\end{lstlisting}\vspace{3pt}
The point of using multiple inheritance is surely for composing the features of
components, achieving great code reuse. It is supported by many mainstream OO
languages. Nevertheless at this point, \lstinline|DrawableDeck| has to throw a compile
error, for the two \lstinline|draw()| methods cause a conflict, though accidentally.

\subsection{Potential fixes}

For that problem, there are several workarounds that quickly come to our mind:

\paragraph{I. Delegation.} As an alternative to multiple inheritance, delegation can be used by
introducing two fields with \lstinline|Drawable| type and \lstinline|Deck| type, respectively. This avoids
method conflicts, nevertheless, delegation itself is too restricted in modularity, and meanwhile
introduces a lot of boilerplate.

\paragraph{II. Creating a \lstinline|draw()| method in \lstinline|DrawableDeck|, which explicitly overrides the old ones.}
This is a non-solution. It does not make any sense to override both methods with totally different functionalities, as old
methods have to be hidden.

\paragraph{III. Choosing one of them as the default method, like Mixins.} The mixin model can be applied to choose a
default one based on linearisation. Similarly, we want to preserve both features, rather than keeping only one of them.

\paragraph{IV. Method exclusion like traits.} Same reason as above.

\paragraph{V. Method renaming like traits.} This is probably what people do in most cases, by simply renaming one to avoid conflicts.
It can indeed preserve both features, however, it is cumbersome in practice, as introducing new names can affect other code blocks.
Certainly this is a workaround, not a solution.\\

What we really expect from the language is we keep both methods without renaming, and the type checker does not complain on the
inheritance. But we need to find an approach to disambiguate on method calls statically.

Certainly the compiler can ignore the conflict when \lstinline|DrawableDeck| is declared, but once an object of \lstinline|DrawableDeck| is created, a method call for \lstinline|draw()| on that object is ambiguous, due to dynamic dispatch. Nonetheless, we can adopt static dispatch for disambiguating. Some languages like C++ use qualified names in that way:

\vspace{3pt}\begin{lstlisting}
void func(Drawable obj) {
  obj.draw();
}

DrawableDeck d = new DrawableDeck();
d.Drawable::draw();       // calling draw() in Drawable
((Drawable) d).draw(); // calling draw() in Drawable
func(d);                 // calling draw() in Drawable
\end{lstlisting}\vspace{3pt}
Thus we have: \paragraph{VI. Static dispatch.} Static dispatch finds out and invokes the most specific method ``by need''.

On the other hand, we also need dynamic dispatch as it is essential and widely used in object-oriented programming.
C++ has the flexibility for choosing either way of dispatch by the ``virtual'' keyword.
Unfortunately, this approach is still unsatisfactory regarding code reuse. For instance, here we redefine \lstinline|Deck| to support
both \lstinline|draw()| and another operation called \lstinline|shuffleAndDraw()|:
\vspace{3pt}\begin{lstlisting}
interface Deck {
  void draw() {...}
  void shuffleAndDraw() {
    shuffle();
    draw();
  }
  ...
}
\end{lstlisting}\vspace{3pt}
\lstinline|shuffleAndDraw()| is a representative method that invokes \lstinline|draw()| in its definition. In principle, we want
that invocation to use dynamic dispatch, because a programmer may define a subtype of \lstinline|Deck|, and override \lstinline|draw()|:
\vspace{3pt}\begin{lstlisting}
interface LoggingDeck {
  void draw() { // overriding
    Stack<Card> cards = this.getStack();
    if (!cards.isEmpty()) {
      Card card = cards.pop();
      println("The card is: " + card.toString());
      ...
    } else {
      println("Empty deck.");
    }
  }
}
\end{lstlisting}\vspace{3pt}
Usually we want to reuse the code of \lstinline|shuffleAndDraw()| during inheritance, hence dynamic dispatch is necessary, otherwise
programmers have to override all the other methods that invoke \lstinline|draw()|. However, as seen before, dynamic dispatch can cause
ambiguity if we have:
\vspace{3pt}\begin{lstlisting}
interface DrawableLoggingDeck extends Drawable, LoggingDeck {
  ...
}

DrawableLoggingDeck d = new DrawableLoggingDeck();
d.shuffleAndDraw(); // ambiguous draw()
\end{lstlisting}\vspace{3pt}
Since the dynamic type of the receiver is \lstinline|DrawableLoggingDeck|, calling \lstinline|shuffleAndDraw()| triggers the ambiguity. When \lstinline|shuffleAndDraw()| invokes \lstinline|draw()|, what we really want is \lstinline|LoggingDeck.draw()|, yet
neither static dispatch nor dynamic dispatch in languages like C++ does so.
 Therefore, we need to find another algorithm for method binding.

\subsection{Solution in \MIM: \dispatchnamecaptical}
Our \MIM{} model uses \dispatchnameit{} for method lookup. A qualified method invocation, for instance, \lstinline|e.I::m()|, is read as ``finding the most specific \lstinline|m()| along path \lstinline|I|''. The meaning of ``along path \lstinline|I|'' is that, if the result of \dispatch{} is \lstinline|J.m()| for some \lstinline|J|, then such a \lstinline|J| must be a super type of \lstinline|e|'s dynamic type, and \lstinline|J| has a subtyping relation with \lstinline|I| (either \lstinline|J <: I| or \lstinline|J >: I|). Intuitively, the most specific \lstinline|m()| must be from branch \lstinline|I|, but it can be an overridden version after \lstinline|I| like dynamic dispatch. The formal definition will be introduced later.

On the other hand, \lstinline|((I) e).m()| behaves the same as \lstinline|e.I::m()| in our model. Such a dispatch make uses of both the static type and the dynamic type of the receiver. Intuitively, the static type specifies one branch of the method to avoid ambiguity, and the dynamic type finds the latest version on that branch. It may still introduce ambiguity when there are multiple paths from the static type to the dynamic type, and those paths cause conflicts. To prevent this, we disallow this kind of diamond inheritance to ensure unambiguity. That is to say, we do not allow two conflicted methods to override a same base method. This is natural as it is no longer an ``unintended'' conflict in that way.

With the old example, below code meets our expectation:
\vspace{3pt}\begin{lstlisting}
interface Deck {
  void draw() {...}
  void shuffleAndDraw() {
    this.Deck::shuffle();
    this.Deck::draw();
  }
  ...
}
\end{lstlisting}\vspace{3pt}
It guarantees that \lstinline|shuffleAndDraw()| are calling \lstinline|shuffle()| and \lstinline|draw()| from its own branch, so that the namesakes
from other branches will not cause conflicts. Now \lstinline|d.shuffleAndDraw()| is no longer ambiguous.

Actually in \MIM{}, we can still write ``\lstinline|shuffle();|'' and ``\lstinline|draw();|'',
because the compiler is able to know that the receiver ``\lstinline|this|'' exactly has static type \lstinline|Deck|, hence hierarchical dispatch eliminates ambiguity.
\textcolor{red}{Haoyuan: More. super call. update.}

\subsection{Method refinement}


\section{Formalization}~\label{sec:formalization}
In this section, we present a formal model called \MIM{} (\emph{\textbf{F}eatherweight \textbf{H}ierarchical \textbf{J}ava}), following a similar style as  
Featherweight Java~\cite{Igarashi01FJ}. \MIM{} is a minimal core calculus that formalizes the core concept of hierarchical dispatching and overriding. The syntax, typing rules and small-step semantics are presented.

% \vspace{-2ex}
\subsection{Syntax}
The abstract syntax of \MIM{} interface declarations, method declarations, and expressions is given in Figure~\ref{fig:syntax}. The multiple
inheritance feature of \MIM{} is inspired by Java 8 interfaces, which supports
method implementations via default methods. This feature is 
closely related to \emph{traits}. To demonstrate how
unintentional method conflicts are untangled in \MIM{}, we only focus on a small subset of the interface model. For example, all methods declared
in an interface are either default methods or abstract methods. Default methods provide default implementations for methods. Abstract methods do not
have a method body. Abstract methods can be overridden with future implementations.

\subsubsection{Notations}
The metavariables $I, J$ range over interface names; $x$ ranges over variables; $m$ ranges over method names; $e$ ranges over expressions; and $M$ ranges over method declarations. Following Featherweight Java, we assume that the set of variables includes the special variable \kwthis, which cannot be used as the name of an argument to a method. We use the same
conventions as FJ; we write $\overline{I}$ as shorthand for a possibly empty sequence $I_1, ..., I_n$, which may be indexed by $I_i$; and write $\overline{M}$ as shorthand for $M_1 .. M_n$ (with no commas). We also abbreviate operations on pairs of sequences in an obvious way, writing $\overline{I} \; \overline{x}$ for $I_1 \; x_1, ..., I_n \; x_n$, where $n$ is the length of $\overline{I}$ and $\overline{x}$.

\subsubsection{Interfaces}
In order to achieve multiple inheritance, an interface can have a set of 
parent interfaces, where such a set can be empty. The interface declaration $\interface{I}{I}{M}$ introduces an interface named $I$ with parent interfaces $\overline{I}$ and a suite of methods $\overline{M}$. The methods of $I$ may either override methods that are already defined in $\overline{I}$ or add new functionality special to $I$, we will illustrate this in more detail later.

\subsubsection{Methods}
Original methods and hierarchically overriding methods share the same syntax in our model for simplicity.
The concrete method declaration $\method{I}{m}{I_x}{x}{J}{e}$ introduces a
method named $m$ with result type $I$, parameters $\overline{x}$ of
type $\overline{I_x}$ and the overriding target $J$. The body of the
method simply includes the returned expression $e$. Notably, we have introduced the
\kwoverride{} keyword for two cases: if the overridden interface is exactly the enclosing
interface itself, then such a method is seen as originally defined; otherwise it is a hierarchical overriding method. 
Note that in an interface $J$, $
I \; m(\overline{I_x} \; \overline{x}) \; {\{} \kwreturn \; e ; {\}} $ is syntactic sugar for $\method{I}{m}{I_x}{x}{J}{e}$, which is the standard way of method definition in Java-like languages. The definition
of abstract methods is written as $\absmethod{I}{m}{I_x}{x}{J}$, which is
similar to a concrete method but without the method body. 
For simplicity, overloading is not modelled for methods, which
implies that we can uniquely identify a method by its name.

\subsubsection{Expressions \& Values}
Expressions can be standard constructs such as variables, method
invocation, object creation, together with cast expressions. 
Object creation is represented by $\new I$\footnote{In Java the corresponding syntax is $\new I\{\}$.}. Fields and primitive types are not modelled in \MIM{}. 
The casts are merely safe upcasts, and in fact, they can be viewed as
annotated expressions, where the annotation indicates its static type.
The coexistence of static and dynamic types is the key to hierarchical dispatch.
A value
``$(I)\new{J}$''
is the final result of multiple reduction steps evaluating an
expression.

For simplicity, \name{} does not formalize statements like assignments and so on because they are orthogonal features to the hierarchical dispatching and overriding feature.
A program in \name{} consists of a list of interface declarations, plus a single expression.

\begin{figure*}[t]
\saveSpaceFig
\begin{displaymath}
\begin{array}{l}
\begin{array}{llrl}
\text{Interfaces}   & IL & \Coloneqq & \interface{I}{I}{M} \\
\text{Methods}      & M  & \Coloneqq & \method{I}{m}{I_x}{x}{J}{e}  \mid
									   \absmethod{I}{m}{I_x}{x}{J} \\
\text{Expressions}  & e  & \Coloneqq & x \mid
e.m(\overline{e}) \mid
\new{I} \mid \; (I)e \\
\text{Context}      & \Gamma & \Coloneqq & \overline{x}:\overline{I} \\
\text{Values}       & v & \Coloneqq & (I) \new{J} \\
%%\\
%%\text{Interface names} & I, J, K & & \\
%%\text{Method names} & m & & \\
%%\text{Variable names} & x & &
\end{array}
\end{array}
\end{displaymath}
\caption{Syntax of \name{}.}\label{fig:syntax}
\saveSpaceFig
\end{figure*}


\begin{figure*}[t]
\saveSpaceFig
\begin{mathpar}
	\framebox{$ I <: J $} \hspace{.5in} \subid \\
	\subtrans \hspace{.5in} \subextends \\
	
	\framebox{$ \judgeewf \Gamma {e:I} $} \hspace{.5in}
	\tvar \\
	\tinvk \\
	% \tpathinvk \\
	% \tsuperinvk \\
	% \tstaticinvk  \\
	\tnew \\
	\tanno \\
	\tmethod \\
	\tabsmethod \\
	\tintf
\end{mathpar}
\saveSpaceFig
\caption{Subtyping and Typing Rules of \name{}.}
\label{fig:typingrules}
\end{figure*}

\subsection{Subtyping and Typing Rules}\label{subsec:typingrules}
\subsubsection{Subtyping}
The subtyping of \MIM{} consists of only a few rules shown at the top of Figure~\ref{fig:typingrules}.
In short, subtyping relations are built from the inheritance in interface
declarations. Subtyping is both reflexive and transitive.

\subsubsection{Type-checking}
Details of type-checking rules are displayed at the bottom of Figure~\ref{fig:typingrules}, including expression
typing, well-formedness of methods and interfaces. As a convention, an environment
$\Gamma$ is maintained to store the types of variables, together with
the self-reference $\kwthis$.
% The three rules for method invocation, \textsc{(T-Invk)}, \textsc{(T-PathInvk)} and \textsc{(T-SuperInvk)}
% are very similar, in the sense that they all check the type of the specific method, by using
% an auxiliary function \mtype. \mtype{} is the function for looking up method types, which we will
% illustrate later in Section~\ref{subsec:auxdefs}. After the method
% type is obtained, they all check that the arguments and the receiver
% have compatible types. Additionally, \textsc{(T-PathInvk)} requires the receiver to be the subtype of the specified
% path type, and \textsc{(T-SuperInvk)} checks if the enclosing type directly extends the specified super type.

\textsc{(T-Invk)} is the typing rule for method invocation.
Naturally, the receiver and the arguments are required to be well-typed.
$\mbody$ is our key function for method lookup that implements the
hierarchical dispatching algorithm. The formal definition will be introduced in Section~\ref{sec:auxdefs}.
Here $\mbody(m, I_0, I_0)$ finds the most specific $m$ above $I_0$. ``Above $I_0$'' specifies
the search space, namely the supertypes of $I_0$ including itself.
For the general case, however, the hierarchical invocation $\mbody(m, I, J)$ finds ``the most specific $m$
above $I$ and along path/branch $J$''. ``Along path $J$'' additionally requires the result to relate to $J$, that is to say,
the most specific interface that has a subtyping relationship with $J$.

In \textsc{(T-Invk)}, as the compilation should not be aware
of the dynamic type, it only requires that invoking $m$ is valid for the static type of the
receiver. The result of $\mbody$ contains the interface that provides the most specific implementation,
the parameters and the return type. We use underscore for the return expression, implying that an empty return expression
from an abstract method is acceptable.
% \textsc{(T-PathInvk)} is the typing judgement for a path invocation. Besides the conditions of \textsc{(T-Invk)}, \textsc{(T-PathInvk)} requires the type of receiver to be the subtype of the specified path type. 
% and \textsc{(T-SuperInvk)} checks if the enclosing type directly extends the specified super type.

\textsc{(T-New)} is the typing rule for object creation $\new{I}$. The
auxiliary function $\canInstantiate(I)$ (see definition in Section~\ref{sec:otherdefs}) checks whether an interface $I$ 
can be instantiated or not. Since \wordfork{} inheritance accepts conflicting branches to coexist, the check requires that the most specific method is concrete for each method on each branch.

\textsc{(T-Method)} is more interesting since a method can either be an original method or a hierarchical overriding, though
they share the same syntax and method typing rule. $\mostSpecific(m, I, J)$ is a fundamental function,
used to find ``the most specific interfaces that are above $I$ and
along path $J$, and originally defines $m$'' (see
Section~\ref{sec:auxdefs} for full definition).
By ``most specific interfaces'',
it implies that the inherited supertypes are excluded. Thus the condition $\mostSpecific(m, I, J) = \{J\}$ indicates a characteristic of a hierarchical overriding: it must override an original method; the overriding is direct and there does not exist any other original method $m$ in between.
Then $\mbody(m, J, J)$ provides the type of the original method, so hierarchical overriding has to preserve the type. Finally the return expression
is type-checked to be subtype of the declared return type. For the definition of an original method, $I$ equals $J$ and the rule is straightforward. \textsc{(T-AbsMethod)} is a similar rule but works on abstract method declarations.

\textsc{(T-Intf)} defines the typing rule on interfaces. The first condition is obvious, namely its methods need to be well checked. The third
condition checks whether the overriding between original methods preserves typing. In this condition we again use some helper functions defined in  Section~\ref{sec:auxdefs}. $I[m\ \kwoverride\ I]$ is defined if $I$ originally defines $m$, and $\canOverride(m, I, J)$ checks whether $I.m$ has the same type as $J.m$. Generally the preservation of method type is required for any supertype $J$ and any method $m$.

The second condition of \textsc{(T-Intf)} is more complex and is the key to type soundness. Unlike C++ which rejects on ambiguous calls,
\MIM{} rejects on the definition of interfaces when they form a diamond. Consider the case when the second condition is broken: $\mbody(m, J, J)$
is defined but $\mbody(m, I, J)$ is undefined for some $J$ and $m$. This indicates that $m$ is available and unambiguous from the perspective of $J$,
but is ambiguous to $I$ on branch $J$. It means that there are multiple overriding paths of $m$ from $J$ to $I$, which form a diamond. Hence rejecting
that case meets our expectation. Below is an example (Figure~\ref{fig:examplesmbody} (e)) that illustrates the reason why this condition is needed:
%\bruno{what is the purpose of this example:
%  state-it upfront please. Is this example meant to ilustrate T-Inf?
%  Then it's better to have the example together with the text
%  explaining T-inf.} \yanlin{revised.please check whether you're happy with it.}
\begin{lstlisting}
interface T                 { T m() override T { return new T(); } }
interface A extends T       { T m() override T { return new A(); } }
interface B extends T       { T m() override T { return new B(); } }
interface C extends A, B {}
((T) new C()).m()
\end{lstlisting}
This program does not compile on interface $C$, because of the second condition in \textsc{(T-Intf)}, where $I$ equals $C$ and $J$ equals $T$.
By the algorithm, $\mbody(m, T, T)$ will refer to $T.m$, but $\mbody(m, C, T)$ is undefined, since both $A.m$ and $B.m$ are most specific
to $C$ along path $T$, which forms a diamond. The expression \lstinline|((T) new C()).m()| is one example of triggering ambiguity, but \MIM{}
simply rejects the definition of $C$. To resolve the issue, the programmer needs to have an overriding method in $C$, to explicitly merge
the conflicting ones.

Finally, rule \textsc{(T-Anno)} is the typing rule for a cast expression. By the rule, only upcasts are valid.

\subsection{Small-step Semantics and Congruence}
Figure~\ref{fig:smallstep} defines the small-step semantics and
congruence rules of \MIM{}. When evaluating an expression, they
are invoked and produce a single value in
the end. %\haoyuan{we need to be consistent on paragraph upper/lower case.}

\subsubsection{Semantic Rules} \textsc{(S-Invk)} is the only computation rule we need for method invocation.
As a small-step rule and by congruence, it assumes that the receiver and the arguments are already values.
Specifically, the receiver $(J)\new{I}$ indicates the dynamic type $I$
together with the static type $J$. Therefore $\mbody(m, I, J)$ carries out hierarchical dispatching, acquires
the types, the return expression $e_0$ and the interface $I_0$ which provides the most specific method.
Here we use $e_0$ to imply that the return expression is forced to be non-empty because it requires a concrete implementation. Now the
rule reduces method invocation to $e_0$ with substitution.
Parameters are substituted with arguments, and the \lstinline|this| reference is substituted with the receiver,
and in the meanwhile the static types are recorded via annotations. Finally, the return type $I_e$ is put in the front as an annotation.
\subsubsection{Congruence Rules} \textsc{(C-Receiver)}, \textsc{(C-Args)} and \textsc{(C-FReduce)} are natural congruence rules
on receivers, arguments, and cast-expressions, respectively. \textsc{(C-StaticType)} automatically adds an annotation $I$ to the new
object $\new{I}$. \textsc{(C-AnnoReduce)} merges nested upcasts into a single upcast with the outermost type.



\begin{comment}
\paragraph{Example} In contrast with the counter-example in Section~\ref{subsec:typingrules}, it is better to understand semantics by
well-compiled examples. Here we abstract a variant of the \lstinline|DrawableDeck| example:

\vspace{3pt}\begin{lstlisting}
interface Void       {}
interface JFrame     {}
interface Deck       { Void draw() override Deck { return new Void(); } }
interface Drawable { JFrame draw() override Drawable; }
interface DrawableDeck extends Drawable, Deck {
  JFrame draw() override Drawable {
    return new JFrame();
  }
}

((Drawable) new DrawableDeck()).draw()
\end{lstlisting}\vspace{3pt}
We put \lstinline|Drawable.draw| as an abstract method instead, but hierarchically override it in \lstinline|DrawableDeck|.
By typing rules, the code is well-compiled. And during runtime,
\begin{align*}
	& ((Drawable) new DrawableDeck()).draw() \\
\rightarrow & (JFrame) new JFrame()
\end{align*}
\end{comment}
\section{Fundamental Algorithms and Auxiliary Definitions}\label{sec:auxdefs}
In this section, we present the fundamental algorithms and auxiliary definitions used in our formalization.
Some of them are auxiliary functions, others are however important ones as they
directly implement our algorithm for method lookup.
%\bruno{I felt like there's not enough explanation/context in this
 % section. It is just dumping definitions without connecting them and
 % their purpose properly.}
 
\bruno{used for what and where?}\haoyuan{todo: used for where}

\subsection{Method Lookup Algorithm in \mbody{}}
Before showing the definition of \mbody{}, we start with several examples of method lookup, as shown in Figure~\ref{fig:examplesmbody}. Note that we use \lstinline|m| to denote an original method, and ``\lstinline|m|$\uparrow$\lstinline|A|'' for hierarchical overriding on \lstinline|A|. For each small example, the result gives the interface to which $m$ is dispatched. (a) is our old friend example for unintentional method conflicts (triangle inheritance); (b) and (c) demonstrate
that hierarchical dispatch can find the most specific original method and hierarchical overriding method. More interesting are the two bad examples (d) and (e), they both fail on $\mbody$. (d) is the well-known diamond inheritance, where $\mbody(m, D, A)$ is \Undefined; (e) is again a diamond conflict because the two hierarchical overriding methods are overriding the same branch. Both counter-examples imply that \lstinline|(A)new D().m()| will
lead to ambiguity, and in order to ensure type soundness, both of them have to be rejected by the type checker. Our model guarantees this by the interface checking rule\textsc{(T-Intf)} in Figure~\ref{fig:typingrules}. \haoyuan{need better explanation on these examples.}

Now we present the formal definition of \mbody{} that gives the expected results of the above mentioned examples. 

\begin{flalign*}
	& \rhd \textit{Definition of } \mbody(m, I_d, I_s): & \\
	& \bullet \mbody(m, I_d, I_s) = (J, \overline{I_x} \; \overline{x}, I_e \; e_0) & \\
	& \indent\indent \textrm{with: } \mostSpecific(m, I_d, I_s) = \{I\} & \\
	& \hspace{.77in} \mostSpecificOverride(m, I_d, I) = \{J\} & \\
	& \hspace{.77in} J[m\ \kwoverride\ I] = \method{I_e}{m}{I_x}{x}{I}{e_0} & \\
	& \bullet \mbody(m, I_d, I_s) = (J, \overline{I_x} \; \overline{x}, I_e \; \o) & \\
	& \indent\indent \textrm{with: } \mostSpecific(m, I_d, I_s) = \{I\} & \\
	& \hspace{.77in} \mostSpecificOverride(m, I_d, I) = \{J\} & \\
	& \hspace{.77in} J[m\ \kwoverride\ I] = \absmethod{I_e}{m}{I_x}{x}{I} & \\
\end{flalign*}
$\mbody(m, I_d, I_s)$ denotes the method body lookup function.
We use $I_d, I_s$, since $\mbody$ is usually invoked by a receiver of a method $m$, with its dynamic type $I_d$ and static type $I_s$. Such a function returns the most specific method implementation, more
accurately, its parameters, returned expression (empty for abstract methods) and the types. It considers both originally-defined methods and hierarchical overriding methods, so $\mostSpecific$ and $\mostSpecificOverride$ (see the definition in Section~\ref{sec:mostSpecific} and Section~\ref{sec:mostSpecificOverride}) are both invoked.

To calculate $\mbody(m, I_d, I_s)$, the invocation of $\mostSpecific$ looks for the most specific original methods and their interfaces, and expects a singleton set, so as to avoid unambiguity. Furthermore, the invocation of $\mostSpecificOverride$ also expects a unique (unambiguous) most specific hierarchical override. And finally the target method is returned.

\subsection{Finding the Most Specific Origin: \mostSpecific}\label{sec:mostSpecific}
We proceed to give the definition of two core functions that support method lookup, namely \mostSpecific{} and \mostSpecificOverride. Generally,
$\mostSpecific(m, I, J)$ finds the set of most specific interfaces where $m$ is originally defined, they should be above $I$ and
along path $J$. ``Along path $J$'' means one should be either a subtype or a super type of $J$. Finally with $\prune$ (defined in Section~\ref{sec:otherdefs})
the overridden interfaces will be filtered out.

\begin{flalign*}
	& \rhd \textit{Definition of } \mostSpecific(m, I, J): & \\
	& \bullet \mostSpecific(m, I, J) = \prune(origins) & \\
	& \indent\indent \textrm{with: } origins = \{K \mid \subt{I}{K}, \textrm{ and } \subt{K}{J} \; \lor \; \subt{J}{K}, &\\
	& \hspace{1.62in} \textrm{ and } K[m\ \kwoverride\ K] \textrm{ is defined} \} &
\end{flalign*}
By the definition, an interface belongs to $\mostSpecific(m, I, J)$ if and only if:
\begin{itemize}
	\item It originally defines $m$;
	\item It is a super type of $I$;
	\item It is either a super type or a subtype of $J$ (including $J$ itself);
	\item Any subtype of it does not belong to the same result set because of $\prune$.
\end{itemize}

\subsection{Finding the Most Specifc Overriding: \mostSpecificOverride}\label{sec:mostSpecificOverride}
The $\mostSpecific$ function only focuses on original method implementations, all the hierarchical overriding methods are omitted during that step. On the other hand, $\mostSpecificOverride(m, I, J)$ has the assumption that $J$ defines an original $m$, and this function tries to find the interfaces with most specific implementations that hierarchically overrides such an $m$. Formally,

\begin{flalign*}
	& \rhd \textit{Definition of } \mostSpecificOverride(m, I, J): & \\
	& \bullet \mostSpecificOverride(m, I, J) = \prune(overrides) & \\
	& \indent\indent \textrm{with: } overrides = \{K \mid \subt{I}{K}, \; \subt{K}{J} \textrm{ and } K[m\ \kwoverride\ J] \textrm{ is defined} &
\end{flalign*}
By the definition, an interface belongs to $\mostSpecific(m, I, J)$ if and only if:
\begin{itemize}
	\item It is between $I$ and $J$;
	\item It hierarchically overrides $J.m$;
	\item Any subtype of it does not belong to the same set.
\end{itemize}


\subsection{Others}\label{sec:otherdefs}
Below we give other minor definitions of the auxiliary functions that are used in previous sections.

%%%%============================ I[m override J] ================%%%%%%%%
\begin{flalign*}
	& \rhd \textit{Definition of } I[m\ \kwoverride\ J]: & \\
	& \bullet I[m\ \kwoverride\ J] = \method{I_e}{m}{I_x}{x}{J}{e_0} & \\
	& \indent\indent \textrm{with: }
	  \kwinterface \; I \; \kwextends \; \overline{I} \; \{ \method{I_e}{m}{I_x}{x}{J}{e_0} \ldots \} & \\
	& \bullet I[m\ \kwoverride\ J] = \absmethod{I_e}{m}{I_x}{x}{J} & \\
	& \indent\indent \textrm{with: }
	\kwinterface \; I \; \kwextends \; \overline{I} \; \{ \absmethod{I_e}{m}{I_x}{x}{J} \ldots \} & \\
\end{flalign*}
Here $I[m\ \kwoverride\ J]$ is basically a direct lookup for method $m$ in the body of $I$, where such a method
overrides $J$ (like static dispatch). The method can be either concrete or abstract, and the body of definition is returned. Notice that
by our syntax, $I[m\ \kwoverride\ I]$ is looking for the originally-defined method $m$ in $I$.
%%%%============================ I[m override J] end================%%%%%%%%

%%%%============================ prune(set) ================%%%%%%%%
\begin{flalign*}
	& \rhd \textit{Definition of } \prune(set): & \\
	& \bullet \prune(set) = \{I \in set \; | \; \nexists J \in set\setminus I, J <: I\} &
\end{flalign*}
The $\prune$ function takes a set of
types, and filters out those that have subtypes in the same set. In the returned set,
none of them has subtyping relation to one another, since all super types have been removed.
%%%%============================ prune(set) end ================%%%%%%%%

%%%%============================ canOverride ================%%%%%%%%
\begin{flalign*}
	& \rhd \textit{Definition of } \canOverride(m, I, J): & \\
	& \bullet \canOverride(m, I, J) = True & \\
	& \indent\indent \textrm{with: } I[m\ \kwoverride\ I] = I_e \; m(\overline{I_x} \; \overline{x}) \; \kwoverride \; I \ldots & \\
	& \hspace{.77in} J[m\ \kwoverride\ J] = I_e \; m(\overline{I_x} \; \overline{y}) \; \kwoverride \; J \ldots &
\end{flalign*}
$\canOverride$ just checks that two original $m$ in $I$ and $J$ have the same type.
%%%%============================ canOverride end ================%%%%%%%%

%%%%============================ canInstantiate ================%%%%%%%%
\begin{flalign*}
	& \rhd \textit{Definition of } \canInstantiate(I): & \\
	& \bullet \canInstantiate(I) = True & \\
	& \indent\indent \textrm{with: } \forall J \in \mostSpecific(m, I, I), \mostSpecificOverride(m, I, J) = \{K\}, & \\
	& \hspace{.77in} \textrm{ and } K[m\ \kwoverride\ J] = \method{I_e}{m}{I_x}{x}{J}{e_0} &
\end{flalign*}
$\canInstantiate(I)$ checks whether interface $I$ can be instantiated by the keyword $\kwnew$.
$\mostSpecific(m, I, I)$ represents the set of branches $I$ inherits on method $m$. $I$ can be instantiated
if and only if for every branch, the most specific implementation is unambiguous and non-abstract.
%%%%============================ canInstantiate end ================%%%%%%%%

\section{Discussion}
In this section, we discuss the design space and some reflections of the work. 
\paragraph{Abstract Methods.}
Why we include 
\paragraph{Static Invocation.}
Why we include
\paragraph{Method Resolution Rules.}
Other resolution possibilities and comparison.
\section{Related Work}

\begin{itemize}
	\item Multiple inheritance models
		\begin{itemize}
			\item Mixin
			\item Scala Mixins
			\item trait
			\item C++ model
			\item Java 8
			\item CZ
			\item \red{C# Explicit method implementation}
		\end{itemize}
	\item Resolve unintended method confliction
		\begin{itemize}
			\item Parents are Shared Parts of Objects: Inheritance and Encapsulation in SELF* (tiebreaker rule)
			\item ???
		\end{itemize}
	\item Static+Dynamic type method lookup (any existing language that supports this?)
	\item Formalization based on FJ (novelty: keep static types <I> in formalization)
		\begin{itemize}
			\item Existing formalizations based on FJ proposed new features and added rules in syntax and semantics. But we not only add rules, but also piggyback static types on almost all semantic rules to model method lookup. 
			\item Featherweight defenders, ...
		\end{itemize}
\end{itemize}
\section{Conclusion}


This paper proposes \MIM{} as a formalized multiple inheritance model for
unintentional method conflicts. Previous approaches 
either do not support unintentional method conflicts, thus have to
compromise between code reuse and type safety, or do not fully support
overriding in the presence of unintentional conflicts. To deal with unintentional method conflicts we
introduce two key mechanisms: hierarchical dispatching and
hierarchical overriding. Hierarchical dispatching is inspired by the
mechanisms in C++. We provide a minimal formal model of hierarchical
dispatching in \MIM{}. Such an algorithm makes use of both dynamic type
information and static information from either upcasts or parameters'
information. It not only offers great code reuse like
dynamic dispatch but also ensures unambiguity by our algorithm for
method resolution. Additionally we introduce \emph{hierarchical
  overriding} to allow conflicting methods in different branches to be
individually overridden.

\MIM{} is formalized following the style of
Featherweight Java and proved to be sound. A prototype interpreter is
implemented in Scala. We believe that the formalization of
hierarchical dispatching features is general and
can be safely embedded in other OO models, so as to have support for the triangle
inheritance.

Our model can certainly be improved in some aspects. 
As discussed in Section~\ref{sec:discussion}, there are orthogonal and
non-orthogonal features that can potentially be added to the design space. 
The future work relates to loosening the model without giving up its soundness,
together with more exploration on supporting fields in the multiple inheritance setting.


\begin{comment}
\section*{Acknowledgments}\label{sec:Acknowledgments}

Authors would like to thank YYYYY.
\end{comment}

% \newpage
% \clearpage
% \bibliographystyle{abbrvnat}
\bibliographystyle{splncs03}
\bibliography{paper}




\newpage
\appendix
\section{Appendix}

\subsection{Proofs}~\label{appendix_proof}
TODO.

% \input{sections/Appendix_Formalization.tex}
% \input{sections/Translation.tex}
% \input{sections/Appendix_Translation.tex}

\end{document}
