\documentclass[a4paper]{article}

\usepackage[top=.2in, bottom=.2in, left=.2in, right=.2in]{geometry}

% Remote packages

% For pdflatex, replaced by fontspec:
% \usepackage{tgpagella}
\usepackage[T1]{fontenc}
\usepackage[utf8]{inputenc}

% For xelatex or lualatex:
% \usepackage{fontspec}
% \setmainfont{Times New Roman}


\usepackage{amsmath}
\usepackage{amsthm}
\usepackage{amssymb}
\usepackage{mathtools} % For \Coloneqq
\usepackage{bm}        % Bold symbols in maths mode
\usepackage{cmll}
\usepackage{fixltx2e}
\usepackage{stmaryrd}
\usepackage[dvipsnames]{xcolor}
\usepackage{listings} % For code listings
% \usepackage{minted}
% \usemintedstyle{murphy}
\usepackage{fancyvrb}
\usepackage{url}
\usepackage{xspace}
\usepackage{comment}
\usepackage{mdwlist}

% Typography
\usepackage[euler-digits,euler-hat-accent]{eulervm}

% Copied from the FCore paper:
\usepackage[colorlinks=true,allcolors=black,breaklinks,draft=false]{hyperref}   % hyperlinks, including DOIs and URLs in bibliography
% known bug: http://tex.stackexchange.com/questions/1522/pdfendlink-ended-up-in-different-nesting-level-than-pdfstartlink

% Figures with borders
% http://en.wikibooks.org/wiki/LaTeX/Floats,_Figures_and_Captions
% \usepackage{float}
% \floatstyle{boxed}
% \restylefloat{figure}

% Local packages

% \usepackage{styles/bcprules}    % by Benjamin C. Pierce
\usepackage{predef/styles/mathpartir}  % by Didier Rémy


% ! Always load mathastext last
% http://mirrors.ibiblio.org/CTAN/macros/latex/contrib/mathastext/mathastext.pdf
% \renewcommand\familydefault\ttdefault
% \usepackage{mathastext}
% \renewcommand\familydefault\rmdefault

% http://tex.stackexchange.com/questions/114151/how-do-i-reference-in-appendix-a-theorem-given-in-the-body
\usepackage{thmtools, thm-restate}
% \newtheorem{theorem}{Theorem}
% \newtheorem{lemma}{Lemma}


\newcommand{\authornote}[3]{{\color{#2} {\sc #1}: #3}}
\newcommand\bruno[1]{\authornote{bruno}{red}{#1}}
\newcommand\yanlin[1]{\authornote{george}{blue}{#1}}
\newcommand\haoyuan[1]{\authornote{bruno}{green}{#1}}
\newcommand{\red}[1]{\textcolor{red}{#1}}

% Define macros immediately before the \begin{document} command
\newcommand{\turns}{\vdash}

\newcommand{\im}[1]{\lvert #1 \rvert}

\newcommand{\hast}{\!:\!}
\newcommand{\subst}[2]  {\lbrack #1 / #2 \rbrack}

% Relations
\newcommand{\subtype}   {<:}

\definecolor{facebook}{HTML}{3B5998}
\newcommand{\yields}[1]{\textcolor{facebook}{\; \hookrightarrow {#1}}}

% Helpers
\newcommand{\ftv}[1]{\textit{ftv}({#1})}

% Spacing
\newcommand{\binderspacing}{\,}
\newcommand{\appspacing}{\;}

% Types
\newcommand{\for}[2]{\forall #1. \binderspacing #2}
\newcommand{\recty}[2]{\{ #1 \hast #2 \}}
% \newcommand{\top}{\{\}}
\newcommand{\andop}{\with}
\newcommand{\pair}[2]{(#1, #2)}

% Expressions
\newcommand{\lam}[3]{\lambda (#1 \hast #2).\binderspacing #3}
\newcommand{\blam}[2]{\Lambda #1.\binderspacing #2}
\newcommand{\app}[2]{#1 \appspacing #2}
\newcommand{\tapp}[2]{#1 \appspacing #2}
\newcommand{\mergeop}{,,}
\newcommand{\reccon}[2]{\{ #1 = #2 \}}
\newcommand{\recupdate}[3]{#1 \; \mathbf{with} \; \{#2 = #3\}}
\newcommand{\proj}[2]{{\code{proj}}_{#1} #2}
\newcommand{\letexpr}[3]{\kwlet \; #1 = #2 \; \kwin \; #3}

% Keywords
\newcommand{\keyword}[1]{\texttt{#1}}

\newcommand{\kwlet}{\keyword{let}}
\newcommand{\kwin}{\keyword{in}}
\newcommand{\kwwhere}{\keyword{where}}


\newcommand{\Int}{\code{Int}}
\newcommand{\String}{\code{String}}
\newcommand{\Bool}{\code{Bool}}
\newcommand{\I}{\code{i}}
\newcommand{\J}{\code{j}}


% Rules

% Couleurs
\colorlet{subcolor}{OliveGreen}
\colorlet{targetcolor}{BrickRed}

% Source/elaboration and labels
\newcommand{\rulelabelerecupd}{\rulelabele\text{rec-upd}}


% Presentation
\definecolor{lightyellow}{HTML}{FFFFE0}
\newcommand{\highlight}[1]{\colorbox{GreenYellow}{$#1$}}


% To be retired
\newcommand{\turnsget}{\vdash_{\textrm{get}}}
\newcommand{\turnsput}{\vdash_{\textrm{put}}}
\newcommand{\turnsrec}{\vdash_{\textrm{rec}}}
\newcommand{\rulename}[1]{(\textrm{#1})}



\newcommand{\fand}{{\bf $F_{\&}$}\xspace}

\newcommand{\restrictop}{\setminus}

\newcommand{\orthog}{\perp}

\newcommand{\rulelabelorthog}{\bm{o}}

\newcommand{\rulelabelorthogvar}{\rulelabelorthog\text{var}}
\newcommand{\ruleorthogvar}{
\inferrule* [right=$\rulelabelorthogvar$]
  {\alpha_1 \neq \alpha_2}
  {\alpha_1 \orthog \alpha_2}
}

\newcommand{\rulelabelorthogfun}{\rulelabelorthog\text{fun}}
\newcommand{\ruleorthogfun}{
\inferrule* [right=$\rulelabelorthogfun$]
  {\tau_1 \orthog \tau_3 \\ \tau_2 \orthog \tau_4}
  {\tau_1 \to \tau_2 \orthog \tau_3 \to \tau_4}
}

\newcommand{\rulelabelorthogforall}{\rulelabelorthog\text{forall}}
\newcommand{\ruleorthogforall}{
\inferrule* [right=$\rulelabelorthogforall$]
  {\tau_1 \orthog \subst {\alpha_1} {\alpha_2} \tau_2}
  {\for {\alpha_1} \tau_1 \orthog \for {\alpha_2} \tau_2}
}

\newcommand{\rulelabelorthogandleft}{\rulelabelorthog{\text{and-left}}}
\newcommand{\ruleorthogandleft}{
\inferrule* [right=$\rulelabelorthogandleft$]
  {\tau_1 \orthog \tau_3 \\ \tau_2 \orthog \tau_3}
  {\tau_1 \andop \tau_2 \orthog \tau_3}
}

\newcommand{\rulelabelorthogandright}{\rulelabelorthog{\text{and-right}}}
\newcommand{\ruleorthogandright}{
\inferrule* [right=$\rulelabelorthogandright$]
  {\tau_1 \orthog \tau_2 \\ \tau_1 \orthog \tau_3}
  {\tau_1 \orthog \tau_2 \andop \tau_3}
}

\newcommand{\rulelabelorthogrec}{\rulelabelorthog\text{rec}}
\newcommand{\ruleorthogrec}{
\inferrule* [right=$\rulelabelorthogrec$]
  {l_1 \neq l_2}
  {\recty {l_1} {\tau_1} \orthog \recty {l_2} {\tau_2}}
}
\newcommand{\judgeewf}[2]{#1 \; {\turns} \; #2}

\newcommand{\rulelabelewf}{\bm{wf}}

\newcommand{\rulelabelewfvar}{\rulelabelewf\text{var}}
\newcommand{\rulelabelewftop}{\rulelabelewf\text{top}}
\newcommand{\rulelabelewffun}{\rulelabelewf\text{fun}}
\newcommand{\rulelabelewfforall}{\rulelabelewf\text{forall}}
\newcommand{\rulelabelewfand}{\rulelabelewf{\text{and}}}
\newcommand{\rulelabelewfrec}{\rulelabelewf\text{rec}}

\newcommand{\ruleewf}{
\inferrule* [right=$\rulelabelewf$]
  {\ftv \tau \in \gamma}
  {\judgeewf \gamma \tau}
}

\newcommand{\rulelabeltwf}{\rulelabelt\text{wf}}
\newcommand{\ruletwf}{
\inferrule* [right=$\rulelabeltwf$]
  {\ftv T \in \Gamma}
  {\judgetwf \Gamma T}
}

% Expanded form of well-formedness

\newcommand{\ruleewfvar}{
\inferrule* [right=$\rulelabelewfvar$]
  {\alpha \in \gamma}
  {\judgeewf \gamma \alpha}
}

\newcommand{\ruleewftop}{
\inferrule* [right=$\rulelabelewftop$]
  { }
  {\judgeewf \gamma \top}
}

\newcommand{\ruleewffun}{
\inferrule* [right=$\rulelabelewffun$]
  {\judgeewf \gamma {\tau_1} \\ \judgeewf \Gamma {\tau_2}}
  {\judgeewf \gamma {\tau_1 \to \tau_2}}
}

\newcommand{\ruleewfforall}{
\inferrule* [right=$\rulelabelewfforall$]
  {\judgeewf {\gamma, \alpha} \tau}
  {\judgeewf \gamma {\for \alpha \tau}}
}

\newcommand{\ruleewfand}{
\inferrule* [right=$\rulelabelewfand$]
  {\judgeewf \gamma {\tau_1} \\ \judgeewf \Gamma {\tau_2} \\ \tau_1 \orthog \tau_2}
  {\judgeewf \gamma {\tau_1 \andop \tau_2}}
}

\newcommand{\ruleewfrec}{
\inferrule* [right=$\rulelabelewfrec$]
  {\judgeewf \gamma \tau}
  {\judgeewf \gamma {\recty l \tau}}
}
\newcommand{\rulelabelsub}{\bm{sub}}

\newcommand{\rulelabelsubvar}{\rulelabelsub\text{var}}
\newcommand{\rulesubvar}{
\inferrule* [right=$\rulelabelsubvar$]
  { }
  {\alpha \subtype \alpha}
}

\newcommand{\rulelabelsubtop}{\rulelabelsub\text{top}}
\newcommand{\rulesubtop}{
\inferrule* [right=$\rulelabelsubtop$]
  { }
  {\tau \subtype \top}
}

\newcommand{\rulelabelsubfun}{\rulelabelsub\text{fun}}
\newcommand{\rulesubfun}{
\inferrule* [right=$\rulelabelsubfun$]
  {\tau_3 \subtype \tau_1 \\ \tau_2 \subtype \tau_4}
  {\tau_1 \to \tau_2 \subtype \tau_3 \to \tau_4}
}

\newcommand{\rulelabelsubforall}{\rulelabelsub\text{forall}}
\newcommand{\rulesubforall}{
\inferrule* [right=$\rulelabelsubforall$]
  {\tau_1 \subtype \subst {\alpha_1} {\alpha_2} \tau_2}
  {\for {\alpha_1} \tau_1 \subtype \for {\alpha_2} \tau_2}
}

\newcommand{\rulelabelsuband}{\rulelabelsub\text{and}}
\newcommand{\rulesuband}{
\inferrule* [right=$\rulelabelsuband$]
  {\tau_1 \subtype \tau_2 \\ \tau_1 \subtype \tau_3}
  {\tau_1 \subtype \tau_2 \andop \tau_3}
}

\newcommand{\rulelabelsubandleft}{\rulelabelsub{\text{and}_1}}
\newcommand{\rulesubandleft}{
\inferrule* [right=$\rulelabelsubandleft$]
  {\tau_1 \subtype \tau_3}
  {\tau_1 \andop \tau_2 \subtype \tau_3}
}

\newcommand{\rulelabelsubandright}{\rulelabelsub{\text{and}_2}}
\newcommand{\rulesubandright}{
\inferrule* [right=$\rulelabelsubandright$]
  {\tau_2 \subtype \tau_3}
  {\tau_1 \andop \tau_2 \subtype \tau_3}
}

\newcommand{\rulelabelsubrec}{\rulelabelsub\text{rec}}
\newcommand{\rulesubrec}{
\inferrule* [right=$\rulelabelsubrec$]
  {\tau_1 \subtype \tau_2}
  {\recty l {\tau_1} \subtype \recty l {\tau_2}}
}
\newcommand{\judgeselect}[3]{#1 \bullet #2 = #3}

% select
\newcommand{\rulelabelselect}{\bm{select}}
\newcommand{\ruleget}{
  \inferrule* [right=$\rulelabelselect$]
  { }
  {\judgeselect {\recty l \tau} l \tau}
}

% select1
\newcommand{\rulelabelselectleft}{{\rulelabelselect}_1}
\newcommand{\rulegetleft}{
  \inferrule* [right=$\rulelabelselectleft$]
  {\judgeselect {\tau_1} l \tau}
  {\judgeselect {\tau_1 \andop \tau_2} l \tau}
}

% select2
\newcommand{\rulelabelselectright}{{\rulelabelselect}_2}
\newcommand{\rulegetright}{
  \inferrule* [right=$\rulelabelselectright$]
  {\judgeselect {\tau_2} l \tau}
  {\judgeselect {\tau_1 \andop \tau_2} l \tau}
}

\newcommand{\judgerestrict}[3]{#1 \bm{\restrictop} #2 = #3}

% restrict
\newcommand{\rulelabelrestrict}{\bm{restrict}}
\newcommand{\rulerestrict}{
  \inferrule* [right=$\rulelabelrestrict$]
  { }
  {\judgerestrict {\recty l \tau} l \top}
}

% restrict1
\newcommand{\rulelabelrestrictleft}{{\rulelabelrestrict}_1}
\newcommand{\rulerestrictleft}{
  \inferrule* [right=$\rulelabelrestrictleft$]
  {\judgerestrict {\tau_1} l \tau}
  {\judgerestrict {\tau_1 \andop \tau_2} l {\tau \andop \tau_2}}
}

% restrict2
\newcommand{\rulelabelrestrictright}{{\rulelabelrestrict}_2}
\newcommand{\rulerestrictright}{
  \inferrule* [right=$\rulelabelrestrictright$]
  {\judgerestrict {\tau_2} l \tau}
  {\judgerestrict {\tau_1 \andop \tau_2} l {\tau_1 \andop \tau}}
}

%%%%%%%%%%%%%%%%%%%%%%%%%%%%%%%%%%%%%%%%%%%%%%%%%%%%%%%%%%%%%%%%%%%%%%%%
% Typing
%%%%%%%%%%%%%%%%%%%%%%%%%%%%%%%%%%%%%%%%%%%%%%%%%%%%%%%%%%%%%%%%%%%%%%%%

\newcommand{\judgee}[3]{#1 \; \turns \; #2 \; : \; #3}
\newcommand{\rulelabele}{\bm{ty}}

% var
\newcommand{\rulelabelevar}{\rulelabele\text{var}}
\newcommand{\ruleevar} {
\inferrule* [right=$\rulelabelevar$]
  {(x,\tau) \in \gamma}
  {\judgee \gamma x \tau}
}

% top
\newcommand{\rulelabeletop}{\rulelabele\text{top}}
\newcommand{\ruleetop} {
\inferrule* [right=$\rulelabeletop$]
  { }
  {\judgee \gamma \top \top}
}

% lam
\newcommand{\rulelabelelam}{\rulelabele\text{lam}}
\newcommand{\ruleelam} {
\inferrule* [right=$\rulelabelelam$]
  {\judgee {\gamma, x \hast \tau} e {\tau_1} \\ \judgeewf \gamma \tau}
  {\judgee \gamma {\lam x \tau e} {\tau \to \tau_1}}
}

% app
\newcommand{\rulelabeleapp}{\rulelabele\text{app}}
\newcommand{\ruleeapp}{
\inferrule* [right=$\rulelabeleapp$]
  {\judgee \gamma {e_1} {\tau_1 \to \tau_2} \\
   \judgee \gamma {e_2} {\tau_3} \\
   \tau_3 \subtype \tau_1}
  {\judgee \gamma {\app {e_1} {e_2}} {\tau_2}}
}

% blam
\newcommand{\rulelabeleblam}{\rulelabele\text{blam}}
\newcommand{\ruleeblam}{
\inferrule* [right=$\rulelabeleblam$]
  {\judgee {\gamma, \alpha} e \tau}
  {\judgee \gamma {\blam \alpha e} {\for \alpha \tau}}
}

% tapp
\newcommand{\rulelabeletapp}{\rulelabele\text{tapp}}
\newcommand{\ruleetapp}{
\inferrule* [right=$\rulelabeletapp$]
  {\judgee \gamma e {\for \alpha {\tau_1}} \\ \judgeewf \gamma \tau}
  {\judgee \gamma {\tapp e \tau} {\subst \tau \alpha \tau_1}}
}

% merge
\newcommand{\rulelabelemerge}{\rulelabele\text{merge}}
\newcommand{\ruleemerge}{
\inferrule* [right=$\rulelabelemerge$]
  {\judgee \gamma {e_1} {\tau_1} \\
   \judgee \gamma {e_2} {\tau_2}}
  {\judgee \gamma {e_1 \mergeop e_2} {\tau_1 \andop \tau_2}}
}

% rec-construct
\newcommand{\rulelabelerecconstruct}{\rulelabele\text{rec-construct}}
\newcommand{\ruleerecconstruct}{
\inferrule* [right=$\rulelabelerecconstruct$]
  {\judgee \gamma e \tau}
  {\judgee \gamma {\reccon l e} {\recty l \tau}}
}

% rec-select
\newcommand{\rulelabelerecselect}{\rulelabele\text{rec-select}}
\newcommand{\ruleerecselect}{
\inferrule* [right=$\rulelabelerecselect$]
  {\judgee \gamma e \tau \\
   \judgeselect \tau l {\tau_1}}
  {\judgee \gamma {e.l} {\tau_1}}
}

% rec-restrict
\newcommand{\rulelabelerecrestrict}{\rulelabele\text{rec-restrict}}
\newcommand{\ruleerecrestrict}{
\inferrule* [right=$\rulelabelerecrestrict$]
  {\judgee \gamma e \tau \\
   \judgerestrict \tau l {\tau_1}}
  {\judgee \gamma {e \restrictop l} {\tau_1}}
}


%%%%%%%%%%%%%%%%%%%%%%%%%%%%%%%%%%%%%%%%%% syntax.tex begin %%%%%%%%%%%%%%%%%%%%%%
\newcommand{\kwinterface}{\keyword{interface}}
\newcommand{\kwextends}{\keyword{extends}}
\newcommand{\kwreturn}{\keyword{return}}
\newcommand{\kwoverride}{\keyword{override}}
\newcommand{\kwsuper}{\keyword{super}}
\newcommand{\kwthis}{\keyword{this}}
\newcommand{\kwnew}{\keyword{new}}
\newcommand{\kwtrue}{\keyword{true}}
\newcommand{\kwfalse}{\keyword{false}}


\newcommand{\mtype}{\keyword{mtype}}
\newcommand{\ext}{\keyword{ext}}
\newcommand{\definedin}{\keyword{definedin}}
\newcommand{\collectMethods}{\keyword{collectMethods}}
\newcommand{\mbody}{\keyword{mbody}}
\newcommand{\Undefined}{\keyword{Undefined}}
\newcommand{\Error}{\keyword{Error}}
\newcommand{\needed}{\keyword{needed}}
\newcommand{\methods}{\keyword{methods}}
\newcommand{\only}{\keyword{only}}
\newcommand{\pathcheck}{\keyword{pathcheck}}

\newcommand{\new}[1]{
    \kwnew \; #1()
}

\newcommand{\interface}[3]{
  \kwinterface \; #1 \; \kwextends \; \overline{#2} \; {\{} \overline{#3} {\}}
}
\newcommand{\method}[6]{
  #1 \; #2 (\overline{#3} \; #4) \; \kwoverride \; #5 \; {\{} \kwreturn \; #6 ; {\}}
}

\newcommand{\subid} {
\inferrule* [right=]
    {}
    {I \subtype I}
}
\newcommand{\subtrans} {
\inferrule* [left=]
    {I \subtype J \\ J \subtype K}
    {I \subtype K}
}

\newcommand{\subextends} {
\inferrule* [right=]
    {\interface{I}{I}{M}}
    {\forall I_i \in \overline{I}, I \subtype I_i}
}

\newcommand{\tvar} {
\inferrule* [left=(T-Var)]
    {}
    {\judgeewf \Gamma x:\Gamma(x)}
}

\newcommand{\tinvk} {
\inferrule* [left=(T-Invk)]
    {  \judgeewf \Gamma {e_0:I_0}
    \\ \mtype(m, I_0) = \overline{J} \to I
    \\ \judgeewf \Gamma \overline{e}:\overline{I}
    \\ \overline{I} \subtype \overline{J}
    }
    {\judgeewf \Gamma e_0.m(\overline{e}):I}
}

\newcommand{\tpathinvk} {
\inferrule* [left=(T-PathInvk)]
    {  \judgeewf \Gamma {e_0:I_0}
    \\ I_0 \subtype J_0
    \\ \mtype(m, J_0) = \overline{J} \to I
    \\ \judgeewf \Gamma \overline{e}:\overline{I}
    \\ \overline{I} \subtype \overline{J}
    }
    {\judgeewf \Gamma e_0.J_0::m(\overline{e}):I}
}

\newcommand{\tsuperinvk} {
\inferrule* [left=(T-SuperInvk)]
    {  \judgeewf \Gamma {this:I_0}
    \\ \ext(I_0, J_0)
    \\ \mtype(m, J_0) = \overline{J} \to I
    \\ \judgeewf \Gamma \overline{e}:\overline{I}
    \\ \overline{I} \subtype \overline{J}
    }
    {\judgeewf \Gamma_{I_0} {\kwsuper.J_0::m(\overline{e})}:I}
}

\newcommand{\tnew} {
\inferrule* [left=(T-New)]
    {}
    {\judgeewf \Gamma \new{I}:I}
}

\newcommand{\tmethod} {
\inferrule* [left=(T-Method)]
    {  \ext(I, J)
    \\ \mtype(m, J) = \overline{I} \to I_0
    \\ \text{If } I=J \text{ then } \only(m, I) = \kwtrue
    %\\ \definedin(m, J) % m is (directly or indirectly) defined in J
    }
    {\method{I_0}{m}{I}{x}{J}{e_0} \text{ OK IN } I}
}

% \newcommand{\tintf} {
% \inferrule* [left=(T-Intf)]
%     {  \overline{I} \text{ OK}
%     \\ \forall m \in \collectMethods(I), \mbody(m, I) \neq \Undefined
%     }
%     { \interface{I}{I}{M} \text{ OK }}
% }
\newcommand{\tintf} {
\inferrule* [left=(T-Intf)]
    {  \overline{I} \text{ OK}
    \\ \forall m \in \collectMethods(I), \red{\pathcheck(I, m)}
    }
    { \interface{I}{I}{M} \text{ OK }}
}


\newcommand{\sinvk} {
\inferrule* [left=(S-Invk)]
{\mbody(m, I, J) = (\overline{X} \; \overline{x}, E' \; e_0)}
{<J>\new{I}.m(<\overline{E}>\overline{e}) \to
    <E'>[<\overline{X}>\overline{e}/\overline{x}, <J>\new{I}/\kwthis]e_0}
}

\newcommand{\spathinvk} {
\inferrule* [left=(S-PathInvk)]
{\mbody(m, I, K) = (\overline{X} \; \overline{x}, E' \; e_0)}
{<J>\new{I}.K::m(<\overline{E}>\overline{e}) \to
    <E'>[<\overline{X}>\overline{e}/\overline{x}, <J>\new{I}/\kwthis]e_0}
}

\newcommand{\ssuperinvk} {
\inferrule* [left=(S-SuperInvk)]
{\mbody(m, K, K) = (\overline{X} \; \overline{x}, E' \; e_0)}
{\kwsuper.K::m(<\overline{E}>\overline{e}) \to
    <E'>[<\overline{X}>\overline{e}/\overline{x}, <J>\new{I}/\kwthis]e_0}
}

\newcommand{\deff} {
\begin{displaymath}
    \begin{array}{l}
        \begin{array}{llrl}
        \text{Annotated Expressions}   & f & \Coloneqq & e \mid <I>e
        \end{array}
    \end{array}
\end{displaymath}
}

\newcommand{\creceiver} {
%----------- v1.0 ---------------
%\inferrule* [left=(C-Receiver)]
%{e \to e'}
%{e.m(\overline{e}) \to e'.m(\overline{e})}
%}
\inferrule* [left=(C-Receiver)]
{f \to f'}
{f.m(\overline{e}) \to f'.m(\overline{e})}
}

\newcommand{\cpathreceiver} {
\inferrule* [left=(C-PathReceiver)]
{f \to f'}
{f.K::m(\overline{e}) \to f'.K::m(\overline{e})}
}

\newcommand{\cargs} {
%---------- v1.0 ---------------
%\inferrule* [left=(C-Args)]
%{e_i \to e_i'}
%{e.m(..., e_i, ...) \to e.m(..., e_i', ...)}
%}
%---------- v2.0 ---------------
%\inferrule* [left=(C-Args)]
%{e_1 \to f}
%{<J>\new{I}.m(<\overline{E}>\overline{e_0}, e_1, \overline{e_2})
%\to
%<J>\new{I}.m(<\overline{E}>\overline{e_0}, f, \overline{e_2})}
%}
%---------- v3.0 ---------------
\inferrule* [left=(C-Args)]
{e_1 \to f}
{<J>\new{I}.m(<\overline{E}>\overline{v}, e_1, \overline{e_2})
\to
<J>\new{I}.m(<\overline{E}>\overline{v}, f, \overline{e_2})}
}

\newcommand{\cpathargs} {
\inferrule* [left=(C-PathArgs)]
{e_1 \to f}
{<J>\new{I}.K::m(<\overline{E}>\overline{v}, e_1, \overline{e_2})
\to
<J>\new{I}.K::m(<\overline{E}>\overline{v}, f, \overline{e_2})}
}

\newcommand{\csuperargs} {
\inferrule* [left=(C-SuperArgs)]
{e_1 \to f}
{super.K::m(<\overline{E}>\overline{v}, e_1, \overline{e_2})
\to
super.K::m(<\overline{E}>\overline{v}, f, \overline{e_2})}
}

\newcommand{\cstatictype} {
\inferrule* [left=(C-StaticType)]
{}
{\new{I} \to <I>\new{I}}
}

\newcommand{\cfreduce} {
\inferrule* [left=(C-FReduce)]
{f \to f' \; f \neq \new{I}}
{<I>f \to <I>f'}
}

\newcommand{\cannoreduce} {
\inferrule* [left=(C-AnnoReduce)]
{}
{<I>(<J>e) \to <I>e}
}
%%%%%%%%%%%%%%%%%%%%%%%%%%%%%%%%%%%%%%%%%% syntax.tex end %%%%%%%%%%%%%%%%%%%%%%%%


%%%%%%%%%%%%%%%%%%%%%%%%%%%%%%%%%%%%%%%%%% syntax.tex begin %%%%%%%%%%%%%%%%%%%%%%
\newcommand{\kwinterface}{\keyword{interface}}
\newcommand{\kwextends}{\keyword{extends}}
\newcommand{\kwreturn}{\keyword{return}}
\newcommand{\kwoverride}{\keyword{override}}
\newcommand{\kwsuper}{\keyword{super}}
\newcommand{\kwthis}{\keyword{this}}
\newcommand{\kwnew}{\keyword{new}}
\newcommand{\kwtrue}{\keyword{true}}
\newcommand{\kwfalse}{\keyword{false}}


\newcommand{\mtype}{\keyword{mtype}}
\newcommand{\ext}{\keyword{ext}}
\newcommand{\definedin}{\keyword{definedin}}
\newcommand{\collectMethods}{\keyword{collectMethods}}
\newcommand{\mbody}{\keyword{mbody}}
\newcommand{\Undefined}{\keyword{Undefined}}
\newcommand{\Error}{\keyword{Error}}
\newcommand{\needed}{\keyword{needed}}
\newcommand{\methods}{\keyword{methods}}
\newcommand{\only}{\keyword{only}}
\newcommand{\pathcheck}{\keyword{pathcheck}}

\newcommand{\new}[1]{
    \kwnew \; #1()
}

\newcommand{\interface}[3]{
  \kwinterface \; #1 \; \kwextends \; \overline{#2} \; {\{} \overline{#3} {\}}
}
\newcommand{\method}[6]{
  #1 \; #2 (\overline{#3} \; #4) \; \kwoverride \; #5 \; {\{} \kwreturn \; #6 ; {\}}
}

\newcommand{\subid} {
\inferrule* [right=]
    {}
    {I \subtype I}
}
\newcommand{\subtrans} {
\inferrule* [left=]
    {I \subtype J \\ J \subtype K}
    {I \subtype K}
}

\newcommand{\subextends} {
\inferrule* [right=]
    {\interface{I}{I}{M}}
    {\forall I_i \in \overline{I}, I \subtype I_i}
}

\newcommand{\tvar} {
\inferrule* [left=(T-Var)]
    {}
    {\judgeewf \Gamma x:\Gamma(x)}
}

\newcommand{\tinvk} {
\inferrule* [left=(T-Invk)]
    {  \judgeewf \Gamma {e_0:I_0}
    \\ \mtype(m, I_0) = \overline{J} \to I
    \\ \judgeewf \Gamma \overline{e}:\overline{I}
    \\ \overline{I} \subtype \overline{J}
    }
    {\judgeewf \Gamma e_0.m(\overline{e}):I}
}

\newcommand{\tpathinvk} {
\inferrule* [left=(T-PathInvk)]
    {  \judgeewf \Gamma {e_0:I_0}
    \\ I_0 \subtype J_0
    \\ \mtype(m, J_0) = \overline{J} \to I
    \\ \judgeewf \Gamma \overline{e}:\overline{I}
    \\ \overline{I} \subtype \overline{J}
    }
    {\judgeewf \Gamma e_0.J_0::m(\overline{e}):I}
}

\newcommand{\tsuperinvk} {
\inferrule* [left=(T-SuperInvk)]
    {  \judgeewf \Gamma {this:I_0}
    \\ \ext(I_0, J_0)
    \\ \mtype(m, J_0) = \overline{J} \to I
    \\ \judgeewf \Gamma \overline{e}:\overline{I}
    \\ \overline{I} \subtype \overline{J}
    }
    {\judgeewf \Gamma_{I_0} {\kwsuper.J_0::m(\overline{e})}:I}
}

\newcommand{\tnew} {
\inferrule* [left=(T-New)]
    {}
    {\judgeewf \Gamma \new{I}:I}
}

\newcommand{\tmethod} {
\inferrule* [left=(T-Method)]
    {  \ext(I, J)
    \\ \mtype(m, J) = \overline{I} \to I_0
    \\ \text{If } I=J \text{ then } \only(m, I) = \kwtrue
    %\\ \definedin(m, J) % m is (directly or indirectly) defined in J
    }
    {\method{I_0}{m}{I}{x}{J}{e_0} \text{ OK IN } I}
}

% \newcommand{\tintf} {
% \inferrule* [left=(T-Intf)]
%     {  \overline{I} \text{ OK}
%     \\ \forall m \in \collectMethods(I), \mbody(m, I) \neq \Undefined
%     }
%     { \interface{I}{I}{M} \text{ OK }}
% }
\newcommand{\tintf} {
\inferrule* [left=(T-Intf)]
    {  \overline{I} \text{ OK}
    \\ \forall m \in \collectMethods(I), \red{\pathcheck(I, m)}
    }
    { \interface{I}{I}{M} \text{ OK }}
}


\newcommand{\sinvk} {
\inferrule* [left=(S-Invk)]
{\mbody(m, I, J) = (\overline{X} \; \overline{x}, E' \; e_0)}
{<J>\new{I}.m(<\overline{E}>\overline{e}) \to
    <E'>[<\overline{X}>\overline{e}/\overline{x}, <J>\new{I}/\kwthis]e_0}
}

\newcommand{\spathinvk} {
\inferrule* [left=(S-PathInvk)]
{\mbody(m, I, K) = (\overline{X} \; \overline{x}, E' \; e_0)}
{<J>\new{I}.K::m(<\overline{E}>\overline{e}) \to
    <E'>[<\overline{X}>\overline{e}/\overline{x}, <J>\new{I}/\kwthis]e_0}
}

\newcommand{\ssuperinvk} {
\inferrule* [left=(S-SuperInvk)]
{\mbody(m, K, K) = (\overline{X} \; \overline{x}, E' \; e_0)}
{\kwsuper.K::m(<\overline{E}>\overline{e}) \to
    <E'>[<\overline{X}>\overline{e}/\overline{x}, <J>\new{I}/\kwthis]e_0}
}

\newcommand{\deff} {
\begin{displaymath}
    \begin{array}{l}
        \begin{array}{llrl}
        \text{Annotated Expressions}   & f & \Coloneqq & e \mid <I>e
        \end{array}
    \end{array}
\end{displaymath}
}

\newcommand{\creceiver} {
%----------- v1.0 ---------------
%\inferrule* [left=(C-Receiver)]
%{e \to e'}
%{e.m(\overline{e}) \to e'.m(\overline{e})}
%}
\inferrule* [left=(C-Receiver)]
{f \to f'}
{f.m(\overline{e}) \to f'.m(\overline{e})}
}

\newcommand{\cpathreceiver} {
\inferrule* [left=(C-PathReceiver)]
{f \to f'}
{f.K::m(\overline{e}) \to f'.K::m(\overline{e})}
}

\newcommand{\cargs} {
%---------- v1.0 ---------------
%\inferrule* [left=(C-Args)]
%{e_i \to e_i'}
%{e.m(..., e_i, ...) \to e.m(..., e_i', ...)}
%}
%---------- v2.0 ---------------
%\inferrule* [left=(C-Args)]
%{e_1 \to f}
%{<J>\new{I}.m(<\overline{E}>\overline{e_0}, e_1, \overline{e_2})
%\to
%<J>\new{I}.m(<\overline{E}>\overline{e_0}, f, \overline{e_2})}
%}
%---------- v3.0 ---------------
\inferrule* [left=(C-Args)]
{e_1 \to f}
{<J>\new{I}.m(<\overline{E}>\overline{v}, e_1, \overline{e_2})
\to
<J>\new{I}.m(<\overline{E}>\overline{v}, f, \overline{e_2})}
}

\newcommand{\cpathargs} {
\inferrule* [left=(C-PathArgs)]
{e_1 \to f}
{<J>\new{I}.K::m(<\overline{E}>\overline{v}, e_1, \overline{e_2})
\to
<J>\new{I}.K::m(<\overline{E}>\overline{v}, f, \overline{e_2})}
}

\newcommand{\csuperargs} {
\inferrule* [left=(C-SuperArgs)]
{e_1 \to f}
{super.K::m(<\overline{E}>\overline{v}, e_1, \overline{e_2})
\to
super.K::m(<\overline{E}>\overline{v}, f, \overline{e_2})}
}

\newcommand{\cstatictype} {
\inferrule* [left=(C-StaticType)]
{}
{\new{I} \to <I>\new{I}}
}

\newcommand{\cfreduce} {
\inferrule* [left=(C-FReduce)]
{f \to f' \; f \neq \new{I}}
{<I>f \to <I>f'}
}

\newcommand{\cannoreduce} {
\inferrule* [left=(C-AnnoReduce)]
{}
{<I>(<J>e) \to <I>e}
}
%%%%%%%%%%%%%%%%%%%%%%%%%%%%%%%%%%%%%%%%%% syntax.tex end %%%%%%%%%%%%%%%%%%%%%%%%


\begin{document}

\section{Syntax}

\subsection{Syntax}
\begin{displaymath}
    \begin{array}{l}
        \begin{array}{llrl}
        \text{Interfaces}   & IL & \Coloneqq & \interface{I}{I}{M} \\
        \text{Methods}      & M  & \Coloneqq & \method{I}{m}{I}{x}{J}{e} \\
        \text{Expressions}  & e  & \Coloneqq & x \mid
                                               e.m(\overline{e}) \mid
                                               \new{I} \mid
                                               e.I::m(\overline{e}) \mid
                                               \kwsuper.I::m(e) \\
        \text{Context}  & \Gamma & \Coloneqq & x_1:I_1 ... x_n:I_n
        \end{array}
    \end{array}
\end{displaymath}

\subsection{Subtyping}
\begin{mathpar}
    \subid \\
    \subtrans \\
    \subextends
\end{mathpar}

\subsection{Typing Rules}
\begin{mathpar}
    \tvar \\
    \tinvk \\
    \tpathinvk \\
    \tsuperinvk \\
    \tnew \\
    \tmethod \\
    \tintf
\end{mathpar}

\subsection{Small-step Semantics}
\begin{mathpar}
    \sinvk \\
    \spathinvk \\
    \ssuperinvk
\end{mathpar}


\subsection{Congruence}
\deff
\begin{mathpar}
    \creceiver \\
    \cargs \\
    \cstatictype \\
    \cfreduce \\
    \cannoreduce
\end{mathpar}




\subsection{Auxilary Definitions}


\subsubsection{\mbody}
\begin{mathpar}
\inferrule* [left=]
{C \{ m() \; \kwoverride \; C ... \} }
{\mbody(m, C, A) = (\overline{X} \; \overline{x}, E \; e_0) \text{ IN } C}

\inferrule* [left=]
{C \{ m() \; \kwoverride \; A ... \} }
{\mbody(m, C, A) = (\overline{X} \; \overline{x}, E \; e_0) \text{ IN } C}

\inferrule* [left=]
{   \mbody(m, C) = \{ A.m(), B.m(), ...\}
 \\ \nexists \; C.m()}
{\mbody(m, C, A) = (\overline{X} \; \overline{x}, E \; e_0) \text{ IN } A}
\end{mathpar}

$$\interface{I}{I}{M}$$

$mbody(m, I)$ algorithm:
\begin{itemize}
 \item If m is defined in I directly, then return I.m()
 \item Else, let $\overline{I'} = mdefined(fathers(I))$, all ancestors of $I$ that has directly defined $m()$.
 \item $\overline{I''} = needed(\overline{I'})$, keep only interfaces that are needed, which are not super-interface of others.
 \item If $\overline{I''}$ is unique, then return this unique one. Else if any two I1,I2 in $\overline{I''}$ share a parent in $\overline{I'}$, then diamond conflict is detected, report error. Else return multiple $m()$s.
\end{itemize}



\subsubsection{\mtype}
$\mtype(m, C)$ algorithm:
\begin{itemize}
 \item If the result of $\mbody(m, C, A)$ is a unique method,
       $\method{I_0}{m}{I}{x}{J}{e_0}$, then $\mtype(m, C) = \overline{I} \to I_0$
 \item Else ($\Undefined$ or multiple methods returned), $\mtype(m, C) = \Error$
\end{itemize}



\subsubsection{\ext}
\ext(I,J) means interface $I$ (directly) extends $J$.
\begin{mathpar}
\inferrule* [left=]
{   \interface{I}{I}{M}
 \\ J \in \overline{I} }
{\ext(I, J) = \kwtrue}      \\

\inferrule* [left=]
{}
{\ext(I, J) = \kwfalse}
\end{mathpar}



\subsubsection{\collectMethods}
\[ \collectMethods(I) = \left( \bigcup_{I_i \in \overline{I}} \methods(I_i) \right) \bigcup \methods(I) \]
\[ \methods(I) = \overline{M}, \text{where } IT(I) = \interface{I}{I}{M} \]



\subsubsection{\needed}

\subsubsection{\only}
$\only(m, I)$ is true iff inside $I$ there is only one (direct) methoed $m$ definition.




\end{document}
