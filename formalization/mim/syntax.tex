%%%%%%%%%%%%%%%%%%%%%%%%%%%%%%%%%%%%%%%%%% syntax.tex begin %%%%%%%%%%%%%%%%%%%%%%
\newcommand{\kwinterface}{\keyword{interface}}
\newcommand{\kwextends}{\keyword{extends}}
\newcommand{\kwreturn}{\keyword{return}}
\newcommand{\kwoverride}{\keyword{override}}
\newcommand{\kwsuper}{\keyword{super}}
\newcommand{\kwthis}{\keyword{this}}
\newcommand{\kwnew}{\keyword{new}}
\newcommand{\kwtrue}{\keyword{true}}
\newcommand{\kwfalse}{\keyword{false}}


\newcommand{\mtype}{\keyword{mtype}}
\newcommand{\ext}{\keyword{ext}}
\newcommand{\definedin}{\keyword{definedin}}
\newcommand{\collectMethods}{\keyword{collectMethods}}
\newcommand{\mbody}{\keyword{mbody}}
\newcommand{\Undefined}{\keyword{Undefined}}
\newcommand{\Error}{\keyword{Error}}
\newcommand{\needed}{\keyword{needed}}
\newcommand{\methods}{\keyword{methods}}
\newcommand{\only}{\keyword{only}}

\newcommand{\new}[1]{
    \kwnew \; #1()
}

\newcommand{\interface}[3]{
  \kwinterface \; #1 \; \kwextends \; \overline{#2} \; {\{} \overline{#3} {\}}
}
\newcommand{\method}[6]{
  #1 \; #2 (\overline{#3} \; #4) \; \kwoverride \; #5 \; {\{} \kwreturn \; #6 ; {\}}
}

\newcommand{\subid} {
\inferrule* [right=]
    {}
    {I \subtype I}
}
\newcommand{\subtrans} {
\inferrule* [left=]
    {I \subtype J \\ J \subtype K}
    {I \subtype K}
}

\newcommand{\subextends} {
\inferrule* [right=]
    {\interface{I}{I}{M}}
    {\forall I_i \in \overline{I}, I \subtype I_i}
}

\newcommand{\tvar} {
\inferrule* [left=(T-Var)]
    {}
    {\judgeewf \Gamma x:\Gamma(x)}
}

\newcommand{\tinvk} {
\inferrule* [left=(T-Invk)]
    {  \judgeewf \Gamma {e_0:I_0}
    \\ \mtype(m, I_0) = \overline{J} \to I
    \\ \judgeewf \Gamma \overline{e}:\overline{I}
    \\ \overline{I} \subtype \overline{J}
    }
    {\judgeewf \Gamma e_0.m(\overline{e}):I}
}

\newcommand{\tpathinvk} {
\inferrule* [left=(T-PathInvk)]
    {  \judgeewf \Gamma {e_0:I_0}
    \\ I_0 \subtype J_0
    \\ \mtype(m, J_0) = \overline{J} \to I
    \\ \judgeewf \Gamma \overline{e}:\overline{I}
    \\ \overline{I} \subtype \overline{J}
    }
    {\judgeewf \Gamma e_0.J_0::m(\overline{e}):I}
}

\newcommand{\tsuperinvk} {
\inferrule* [left=(T-SuperInvk)]
    {  \judgeewf \Gamma {this:I_0}
    \\ \ext(I_0, J_0)
    \\ \mtype(m, J_0) = \overline{J} \to I
    \\ \judgeewf \Gamma \overline{e}:\overline{I}
    \\ \overline{I} \subtype \overline{J}
    }
    {\judgeewf \Gamma_{I_0} {\kwsuper.J_0::m(\overline{e})}:I}
}

\newcommand{\tnew} {
\inferrule* [left=(T-New)]
    {}
    {\judgeewf \Gamma \new{I}:I}
}

\newcommand{\tmethod} {
\inferrule* [left=(T-Method)]
    {  \ext(I, J)
    \\ \mtype(m, J) = \overline{I} \to I_0
    \\ \text{If } I=J \text{ then } \only(m, I) = \kwtrue
    %\\ \definedin(m, J) % m is (directly or indirectly) defined in J
    }
    {\method{I_0}{m}{I}{x}{J}{e_0} \text{ OK IN } I}
}

\newcommand{\tintf} {
\inferrule* [left=(T-Intf)]
    {  \overline{I} \text{ OK}
    \\ \forall m \in \collectMethods(I), \mbody(m, I) \neq \Undefined
    }
    { \interface{I}{I}{M} \text{ OK }}
}

\newcommand{\sinvk} {
\inferrule* [left=(S-Invk)]
{\mbody(m, I, J) = (\overline{X} \; \overline{x}, E' \; e_0)}
{<J>\new{I}.m(<\overline{E}>\overline{e}) \to
    <E'>[<\overline{X}>\overline{e}/\overline{x}, <J>\new{I}/\kwthis]e_0}
}

\newcommand{\spathinvk} {
\inferrule* [left=(S-PathInvk)]
{\mbody(m, I, K) = (\overline{X} \; \overline{x}, E' \; e_0)}
{<J>\new{I}.K::m(<\overline{E}>\overline{e}) \to
    <E'>[<\overline{X}>\overline{e}/\overline{x}, <J>\new{I}/\kwthis]e_0}
}

\newcommand{\ssuperinvk} {
\inferrule* [left=(S-SuperInvk)]
{\mbody(m, K, K) = (\overline{X} \; \overline{x}, E' \; e_0)}
{\kwsuper.K::m(<\overline{E}>\overline{e}) \to
    <E'>[<\overline{X}>\overline{e}/\overline{x}, <J>\new{I}/\kwthis]e_0}
}

\newcommand{\deff} {
\begin{displaymath}
    \begin{array}{l}
        \begin{array}{llrl}
        \text{Annotated Expressions}   & f & \Coloneqq & e \mid <I>e
        \end{array}
    \end{array}
\end{displaymath}
}

\newcommand{\creceiver} {
%\inferrule* [left=(C-Receiver)]
%{e \to e'}
%{e.m(\overline{e}) \to e'.m(\overline{e})}
%}
\inferrule* [left=(C-Receiver)]
{f \to f'}
{f.m(\overline{e}) \to f'.m(\overline{e})}
}

\newcommand{\cargs} {
%---------- v1.0 ---------------
%\inferrule* [left=(C-Args)]
%{e_i \to e_i'}
%{e.m(..., e_i, ...) \to e.m(..., e_i', ...)}
%}
%---------- v2.0 ---------------
%\inferrule* [left=(C-Args)]
%{e_1 \to f}
%{<J>\new{I}.m(<\overline{E}>\overline{e_0}, e_1, \overline{e_2})
%\to
%<J>\new{I}.m(<\overline{E}>\overline{e_0}, f, \overline{e_2})}
%}
%---------- v3.0 ---------------
\inferrule* [left=(C-Args)]
{e_1 \to f}
{<J>\new{I}.m(<\overline{E}>\overline{v}, e_1, \overline{e_2})
\to
<J>\new{I}.m(<\overline{E}>\overline{v}, f, \overline{e_2})}
}

\newcommand{\cstatictype} {
\inferrule* [left=(C-StaticType)]
{}
{\new{I} \to <I>\new{I}}
}

\newcommand{\cfreduce} {
\inferrule* [left=(C-FReduce)]
{f \to f' \; f \neq \new{I}}
{<I>f \to <I>f'}
}

\newcommand{\cannoreduce} {
\inferrule* [left=(C-AnnoReduce)]
{}
{<I>(<J>e) \to <I>e}
}
%%%%%%%%%%%%%%%%%%%%%%%%%%%%%%%%%%%%%%%%%% syntax.tex end %%%%%%%%%%%%%%%%%%%%%%%%
