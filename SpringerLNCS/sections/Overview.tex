\section{A Running Example: \lstinline{DrawableDeck}}

This section illustrates the features of our \MIM{} model for resolving unintentional method
conflicts. As mentioned before, such a case arises when two inherited methods happen to have the
same signature, but with different semantics and functionalities. This could be quite troublesome
to programmers that use multiple inheritance. Below we illustrate with a running example called \lstinline|DrawableDeck|.
Note that we use Java-like syntax throughout the paper, and all types are defined with the keyword ``\lstinline|interface|'', which
supports multiple inheritance. Since \MIM{} is designed to be simple, we do not allow abstract methods, that is every method
is required to have a body for its implementation. In that case, interfaces can be directly instantiated by the keyword ``\lstinline|new|''
to create an object.

\subsection{Problem: Unintentional Method Conflicts}

Suppose that two components \lstinline|Drawable| and \lstinline|Deck| have been developed in a system.
\lstinline|Drawable| defines an interface for graphics that can be drawn, which includes a method called \lstinline|draw()|
for visual display. While interface \lstinline|Deck| represents a deck of cards, and supports several operations, like
\lstinline|draw()| for drawing a card from the deck.

\begin{lstlisting}
interface Deck {
  void draw() { // draws a card from the Deck
    Stack<Card> cards = this.getStack();
    Card card = cards.pop();
    ...
  }
}

interface Drawable {
  void draw() { // draws something on the screen
    JFrame frame = new JFrame("Canvas");
    frame.setVisible(true);
    ...
  }
}
\end{lstlisting}
Note that both methods have \lstinline|void| return type (we will not formalize
\lstinline|void| in our calculus afterwards; here is only for illustration). In \lstinline|Deck|, \lstinline|draw()| tries to get the cards as a stack, pops
out the top card, and so on. While in \lstinline|Drawable|, \lstinline|draw()|
creates a blank canvas using \lstinline|JFrame|. Now, a programmer is designing a
card game with GUI. He may want to draw a deck on the screen, so he defines a drawable
deck using multiple inheritance:

\begin{lstlisting}
interface DrawableDeck extends Deck, Drawable {
  ...
} 
\end{lstlisting}
The point of using multiple inheritance is surely for composing the features of
components, achieving great code reuse. It is supported by many mainstream OO
languages. Nevertheless at this point, \lstinline|DrawableDeck| has to throw a compile
error, for the two \lstinline|draw()| methods cause a conflict, though accidentally.

\subsection{More}
\begin{itemize}
	\item Code for the problem (Draw example)
	\item Potential fixes (delegation, renaming, traits, ....)
	\item How it is solved with our language (client code, simple explanation) + show update
\end{itemize}


\subsection{Potential fixes}

For that problem, there are several workarounds that quickly come to our mind:

\paragraph{I. Delegation.}

\paragraph{II. Creating a \lstinline|draw()| method in \lstinline|DrawableDeck|, which explicitly overrides the old ones.}

\paragraph{III. Choosing one of them as the default method, like Mixins.}

\paragraph{IV. Method exclusion like traits.}

\paragraph{V. Method renaming.}
In previous solutions, only renaming can preserve the two different methods m(), 
but it’s cumbersome to do in practice: new names have to be introduced, and (probably) need to prefer one over another. 
Other solutions only preserve one method. Can we keep both methods without renaming?

\subsection{Solution in our language}
\begin{lstlisting}

\end{lstlisting}

\subsection{Method refinement}

