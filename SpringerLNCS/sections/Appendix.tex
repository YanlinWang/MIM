\section{Appendix}

\subsection{A Non-Trivial Example}

\begin{lstlisting}
interface A extends {
A m() override A {return new A(); }
}
interface B extends A {
A m() override B {return new B(); }
}
interface C extends A {
A m() override C {return new C(); }
}
interface D extends B, C {
A m() override B {return new D(); }
}
interface E extends B {
A m() override B {return new E(); }
}
interface F extends D, E {
A m() override B {return new F(); } 
A n(B b) override F {return b.m(); }
}
new F().n(new F())
\end{lstlisting}

\begin{comment}
\[
\begin{array}{l}
\kwinterface \; A \; \kwextends \; \{ \\
\; \; \; \; A \; m() \; \kwoverride \; A \; \{\kwreturn \; \new A;\} \\
\} \\
\\
\kwinterface \; B \; \kwextends \; A \; \{ \\
\; \; \; \; A \; m() \; \kwoverride \; B \; \{\kwreturn \; \new B;\} \\
\} \\
\\
\kwinterface \; C \; \kwextends \; A \; \{ \\
\; \; \; \; A \; m() \; \kwoverride \; C \; \{\kwreturn \; \new C;\} \\
\} \\
\\
\kwinterface \; D \; \kwextends \; B, \; C \; \{ \\
\; \; \; \; A \; m() \; \kwoverride \; B \; \{\kwreturn \; \new D;\} \\
\} \\
\\
\kwinterface \; E \; \kwextends \; B \; \{ \\
\; \; \; \; A \; m() \; \kwoverride \; B \; \{\kwreturn \; \new E;\} \\
\} \\
\\
\kwinterface \; F \; \kwextends \; D, \; E \; \{ \\
\; \; \; \; A \; m() \; \kwoverride \; B \; \{\kwreturn \; \new F;\} \\
\; \; \; \; A \; n(B \; b) \; \kwoverride \; F \; \{\kwreturn \; b.m();\} \\
\} \\
\\
\new F.n(\new F)
\end{array}
\]
\end{comment}

~\red{Unfortunately I think this example shows that it is hard to reuse
	$D.m$ on path $B$ and $E.m$ on path $B$ in $F$?}

\red{We can use the $\kwsuper$ keyword to access the originally defined methods
	in super types, but we cannot access the old updating methods.}

\red{Just like $\kwsuper.I::m()$ is equivalent to $\new I.m()$, maybe we can add
	a degree of freedom to $\kwsuper$, for example, $\kwsuper.D::B::m()$ is equivalent to
	$\new D.B::m()$, so we can use $\kwsuper.D::B::m()$ and $\kwsuper.E::B::m()$ inside interface $F$ for code reuse?}

Interfaces $A,B,C,D,E,F$ OK in type checking.

To type-check $\new F.n(\new F)$:
\begin{itemize}
	\item By (T-INVK), we need to calculate $\mtype(n, F)$.
	\item $\mtype(n, F) = B \to A$. And $\new F : B$.
	\item $\new F.n(\new F) : A$.
\end{itemize}

~

To compute \red{$\new F.n(\new F)$}:
\begin{itemize}
	\item By (C-RECEIVER), we compute \red{$\new F$}:
	\begin{itemize}
		\item By (C-STATICTYPE): \red{$<F> \new F$}.
	\end{itemize}
	\item By (C-ARGS), we compute \red{$\new F$}:
	\begin{itemize}
		\item By (C-STATICTYPE): \red{$<F> \new F$}.
	\end{itemize}
	\item Now we get \red{$(<F> \new F).n(<F> \new F)$}. By (S-INVK):
	\begin{itemize}
		\item Compute $\mbody(n, F, F) = \red{(B \; b, A \; b.m())}$.
		\item Replace \red{$b$} with \red{$<B> \new F$} in \red{$b.m()$}.
		\item Replace \red{$\kwthis$} with \red{$<F> \new F$} in \red{$b.m()$}.
	\end{itemize}
	\item Finally we get \red{$<A>((<B> \new F).m())$}.
	\item By (C-FREDUCE), we first compute \red{$(<B> \new F).m()$}:
	\begin{itemize}
		\item By (S-INVK), we compute \red{$\mbody(m, F, B)$}.
		\item In $\mbody$, we invoke \red{$\mostSpecific(m, F, B)$}.
		\item In $\mostSpecific$, \red{$set = \{B\}$, $\prune(set) = \{B\}$}.
		\item Back to $\mbody$, we invoke \red{$\mostSpecific_3(m, F, B)$}.
		\item In $\mostSpecific_3$, \red{$set = \{B,D,E,F\}$, $\prune(set) = \{F\}$}.
		\item Back to $\mbody$, we check \red{$F.m$} and return \red{$(-,A \; (\new F))$}.
		\item Back to (S-INVK).
		\item Replace \red{$\kwthis$} with \red{$<B> \new F$} in \red{$\new F$}.
		\item Finally we get \red{$<A> \new F$}.
	\end{itemize}
	\item Now we have \red{$<A>(<A> \new F)$}.
	\item By (C-ANNOREDUCE): \red{$<A> \new F$}.
\end{itemize}

\red{TODO: class encapsulation problem: interface A: m; interface B: m update A; interface C: m update B.}