\documentclass[10pt,numbers]{sigplanconf}

% The following \documentclass options may be useful:

% preprint      Remove this option only once the paper is in final form.
% 10pt          To set in 10-point type instead of 9-point.
% 11pt          To set in 11-point type instead of 9-point.
% numbers       To obtain numeric citation style instead of author/year.
% nocopyrightspace ...

\usepackage{listings}
\usepackage{xspace}
\usepackage{multicol}
\usepackage{microtype}%if unwanted, comment out or use option "draft"
\usepackage[table,xcdraw]{xcolor}
\usepackage{color}
\usepackage{amsthm}
\usepackage{amsmath}
\usepackage{stmaryrd}
\usepackage{graphicx}
\usepackage{amssymb}
\usepackage{fancyvrb}
\usepackage{url}
\usepackage{pstricks,pst-node,pst-tree}
\usepackage{bbm}
\usepackage{pgf}
\usepackage{multirow}
\usepackage{enumitem}

\usepackage{listings}
\usepackage{verbatim}
\usepackage{graphicx}
\usepackage{wrapfig}
\usepackage[normalem]{ulem}

\usepackage[T1]{fontenc}
\usepackage[scaled=0.85]{beramono}
\usepackage{mathpartir}
\usepackage[utf8]{inputenc}
\usepackage{flushend}


\setlist[itemize]{noitemsep,nolistsep}
\setlist[enumerate]{noitemsep,nolistsep}
%\let\oldparagraph\paragraph
%\renewcommand{\paragraph}[1]{\vspace{-3pt}\oldparagraph{#1}}

\newcommand{\para}[1]{\vspace{-3pt}\paragraph{#1}}

\usepackage{array}
\newcommand\InTextDef[3]{
\!\!\!\!\begin{array}{c p{#1} c l }
\ensuremath{\!\!\bullet\!\!\!}&\ensuremath{#2} & \ensuremath{=} & \ensuremath{#3}
\end{array}
}
\newcommand\InTextAssert[2]{
\!\!\!\!\begin{array}{c p{#1} c l }
\ensuremath{\!\!\bullet\!\!\!}&\ensuremath{#2} 
\end{array}
}
\newcommand\InTextWith[1]{
{}_{}\quad\quad\mbox{with }#1
}

\newenvironment{grammar}{
$
\begin{array}[t]{lcll}}{\end{array}$
}
\newcommand{\production}[3]{#1&\!{:}{:}{=}\!&#2 & \mbox{{\small{#3}}}}
\newcommand{\productionMore}[2]{&{}&#1 & \mbox{{\small{#2}}}}
%\newcommand{\terminale}[1]{{\text{\tt #1}}}
\newcommand{\nonTerminal}[1]{\mathit{#1}\xspace}
\newcommand\Q\lstinline
\newcommand{\metaVar}[1]{\textit{#1}\xspace}
\newcommand{\ann}{\metaVar{ann}}
\newcommand{\x}{\metaVar{x}}
\newcommand{\e}{\metaVar{e}}
\newcommand{\T}{\metaVar{I}}
\newcommand{\xs}{\overline\x}
\newcommand{\m}{\metaVar{m}}
\newcommand{\es}{\overline\e}
\newcommand{\C}{\metaVar{I}}
\newcommand{\f}{\metaVar{f}}
\newcommand{\MCall}[3]{#1\mbox{\Q@.@}#2\oR#3\cR}
\newcommand{\ctx}{{\cal{E}}}
\newcommand{\emptyctx}{[\ ]}
\newcommand{\val}{\metaVar{v}}
\newcommand{\vals}{\metaVar{vs}}
\newcommand{\Aux}[1]{\textsf{#1}}
\newcommand\QM[1]{\mbox{\Q@#1@}}
\newcommand\oC{\mbox{\Q@\{@}}
\newcommand\cC{\mbox{\Q@\}@}}
\newcommand\oR{\mbox{\Q@(@}}
\newcommand\cR{\mbox{\Q@)@}}
\newcommand{\this}{\mbox{\Q@this@\xspace}}
\newcommand{\mixinAnn}{\mbox{\Q$@Obj$\xspace}}
\newcommand{\weakAnn}{\mbox{\Q$@ObjOf$\xspace}}
\newcommand{\method}{\metaVar{meth}}
\newcommand{\field}{\metaVar{field}}
\newcommand{\fields}{\overline\field}
\newcommand{\mh}{\metaVar{mh}}
\newcommand{\obj}{\metaVar{obj}}

\newcommand{\spc}{\ }
\newcommand\whereNote{\\\quad\quad}

\newcommand\II{{\cal{I}}}

\newcommand\tops{\Aux{needed}}
\newcommand\dom{\Aux{dom}}
\newcommand\override{\Aux{override}}
\newcommand\shadow{\Aux{shadow}}
\newcommand\mBody{\Aux{mbody}}
\newcommand\sigvalid{\Aux{sigvalid}}
\newcommand\alldefined{\Aux{alldefined}}
\newcommand\Cs{\overline\C}
\newcommand\mhs{\overline\mh}
\newcommand\subtype{\,{<}{:}\,}
\newcommand\methods{\overline\method}
\newcommand\mif{\mbox{if}}
\newcommand\mwhere{\mbox{where}}
\newcommand\miff{\mbox{iff}}
\newcommand\mand{\mbox{and}}
\newcommand\mnot{\mbox{not}}
\newcommand\mor{\mbox{or}}
\newcommand\mimply{\mbox{implies}}
\newcommand\conflicted{\Aux{conflicted}}
\newcommand\conflictError{\Aux{error}}
\newcommand\none{\Aux{None}}
\newcommand\error{\Aux{Error}}
\newcommand\ofMethod{\Aux{ofMethod}}
\newcommand\otherMethod{\Aux{refine}}
\newcommand\fieldsFunc{\Aux{fields}}
\newcommand\isField{\Aux{isField}}
\newcommand\isWith{\Aux{isWith}}
\newcommand\isClone{\Aux{isClone}}
\newcommand\isSetter{\Aux{isSetter}}
\newcommand\withMethod{\Aux{withMethod}}
\newcommand\setterMethod{\Aux{setterMethod}}
\newcommand\cloneMethod{\Aux{cloneMethod}}
\newcommand{\rn}[1]{(\textsc{#1})}					% Rule name
\newcommand\isImplemented{\Aux{isImplemented}}
\newcommand\with{\Aux{with}}
\newcommand\clone{\Aux{clone}}
\newcommand\specialName{\Aux{special}}
\newcommand\tab{\ \ \ \ }
\newcommand\valid{\Aux{valid}}
\newcommand\get{\Aux{get}}
\newcommand\super{\Aux{super}}

\newcommand\saveSpaceFig{\vspace{-2ex}}
\newcommand\numOfCaseStudies{4 }
\newcommand\classbased{class-based\xspace}
\newcommand\interfacebased{interface-based\xspace}
\newcommand\InterfaceBased{Interface-Based\xspace}
\newcommand\prototypebased{prototype-based\xspace}
\newcommand\Prototypebased{Prototype-based\xspace}
\newcommand\objectoriented{object-oriented\xspace}
\newcommand\Objectoriented{Object-oriented\xspace}
\newcommand\savespace{\vspace{-2pt}}
\input{predef/java.tex}

\newcommand{\Or}{\ |\ }

\newcommand{\cL}{{\cal L}}
\hyphenation{}

\pagestyle{plain}

\newcommand{\authornote}[3]{{\color{#2} {\sc #1}: #3}}
\newcommand{\authorText}[2]{{\color{#1}#2}}
\newcommand\bruno[1]{\authornote{bruno}{red}{#1}}
\newcommand\yanlin[1]{\authornote{yanlin}{purple}{#1}}
\newcommand\marco[1]{\authornote{marco}{blue}{#1}}
\newcommand\haoyuan[1]{\authornote{haoyuan}{cyan}{#1}}
\newcommand\brunoT[1]{\authorText{red}{#1}}
\newcommand\yanlinT[1]{\authorText{purple}{#1}}
\newcommand\marcoT[1]{\authorText{blue}{#1}}
\newcommand\haoyuanT[1]{\authorText{cyan}{#1}}

\newcommand\delete[1]{\textcolor{red}{\sout{#1}}}

\newcommand\sem[1]{\llbracket #1 \rrbracket_r}
\newcommand\sems[1]{\llbracket #1 \rrbracket_s}
\newcommand\tsem[1]{\llbracket #1 \rrbracket}
\newcommand{\rbm}[1]{\raisebox{-2.0ex}[0.5ex]{#1}}
\newcommand\nat[0]{\mathbb{N}}
\newcommand\unit[0]{\mathbbm{1}}

\newcommand\mixin{\mixinAnn\xspace}
%\renewcommand{\paragraph}[1]{\vspace{5pt}\noindent{\bf #1}}
\newenvironment{listing}{\vspace{-3pt}\begin{lstlisting}}{\end{lstlisting}\vspace{-3pt}}

%\theoremstyle{plain}
\newtheorem{thm}{Theorem}
\newtheorem{lem}{Lemma}
\newtheorem{thm2}{Theorem}
\newtheorem{lem2}{Lemma}

\clubpenalty = 10000
\widowpenalty = 10000
\displaywidowpenalty = 10000

\begin{document}

\toappear{}

\special{papersize=8.5in,11in}
\setlength{\pdfpageheight}{\paperheight}
\setlength{\pdfpagewidth}{\paperwidth}

%\conferenceinfo{CONF 'yy}{Month d--d, 20yy, City, ST, Country}
%\copyrightyear{20yy}
%\copyrightdata{978-1-nnnn-nnnn-n/yy/mm}
%\copyrightdoi{nnnnnnn.nnnnnnn}

% Uncomment the publication rights you want to use.
%\publicationrights{transferred}
%\publicationrights{licensed}     % this is the default
%\publicationrights{author-pays}

% \titlebanner{banner above paper title}        % These are ignored unless
% \preprintfooter{short description of paper}   % 'preprint' option specified.

\title{Classless Java}
%\subtitle{\InterfaceBased Programming for the Masses}

\vspace{-20pt}
\authorinfo{Yanlin Wang\and Haoyuan Zhang \\ Bruno C. d. S. Oliveira}
           {The University of Hong Kong, China}
           {\{ylwang,hyzhang,bruno\}@cs.hku.hk}
\authorinfo{Marco Servetto}
          {Victoria University of Wellington, New Zealand}
          {marco.servetto@ecs.vuw.ac.nz}

\maketitle
\vspace{-20pt}

\begin{abstract}
  This paper presents an OO style without classes, which we call \interfacebased \objectoriented
  programming (IB). IB is a natural extension of closely
  related ideas such as traits.
  \emph{Abstract state
    operations} provide a
  new way to deal with state, which allows for flexibility not
  available in class-based languages.  In IB state
  can be type-refined in subtypes. The combination of a
  purely IB style and type-refinement enables
  powerful idioms using multiple inheritance and state. To introduce
  IB to programmers we created Classless Java: an embedding of IB
  directly into Java. Classless Java uses annotation processing for
  code generation and relies on new features of Java 8 for
  interfaces. The code generation techniques used in Classless Java
  have interesting properties, including guarantees that
  the generated code is type-safe and good integration with IDEs.
  Usefulness of IB and Classless Java is shown with
  examples and case studies.
\end{abstract}

\begin{comment}
\category{CR-number}{subcategory}{third-level}

% general terms are not compulsory anymore,
% you may leave them out
\terms
term1, term2

\keywords
keyword1, keyword2
\end{comment}
\category{D.3.2}{Programming Languages}
                {Language Classifications}
                [Object-Oriented Programming]
\category{F.3.3}{Logics and Meanings of Programs}
                {Studies of Program Constructs}
                []
                
\terms
Languages

\keywords
Interface-based programming, multiple inheritance, code generation

\vspace{-10pt}
\section{Introduction}

\begin{itemize}
	\item problem: Unintended method confliction in Multiple inheritance
	\item Existing approaches or models and Their drawbacks
	\item New features (update)
	\item Our contributions
\end{itemize}

In multiple inheritance, naming conflicts often occurs. Among these conflicts, some are real conflicts which needs 
explicit resolve by programmers, however, there are cases where accidental naming conflicts occurs, where the conflicting
methods have completely different meaning/domain which just share the same name. 

Existing OOP models have taken care of the first case intensively. However, few of them supports 
unintended method confliction well. Trait and other 
mainstream OO models do not allow unintended methods confliction to co-exist. 
SELF~\cite{} uses \emph{sender path tiebreaker rule} to automatically resolve 
ambiguities that are almost certainly caused by accidental naming conflicts. C++ allows methods with the same signature 
co-exist in a class via inheritance and programmers can use $::$ operator to select the method wanted. 
However none of them allows refining these unintended conflicting methods in subclasses. We propose a calculus that 
deals with unintended method confliction and meanwhile allows refining these methods. 

Contributions:
\begin{itemize}
    \item A multiple inheritance model formalized as \MIM.
    \item Novel notion \updates for method path updating.
    \item Implementation of a simple typechecker and evaluator in Scala.
\end{itemize}

% \section{A Running Example: DrawableDeck}

This section illustrates the features of our \MIM{} model for resolving unintentional method
conflicts. As mentioned before, such a case arises when two inherited methods happen to have the
same signature, but with different semantics and functionalities. This could be quite troublesome
to programmers that use multiple inheritance. Below we illustrate with a running example called \lstinline|DrawableDeck|.
Note that we use Java-like syntax throughout the paper, and all types are defined with the keyword ``\lstinline|interface|'', which
supports multiple inheritance. Since \MIM{} is designed to be simple, we do not allow abstract methods, that is every method
is required to have a body for its implementation. In that case, interfaces can be directly instantiated by the keyword ``\lstinline|new|''
to create an object.

\subsection{Problem: Unintentional Method Conflicts}

Suppose that two components \lstinline|Drawable| and \lstinline|Deck| have been developed in a system.
\lstinline|Drawable| defines an interface for graphics that can be drawn, which includes a method called \lstinline|draw()|
for visual display. While interface \lstinline|Deck| represents a deck of cards, and supports several operations, like
\lstinline|draw()| for drawing a card from the deck.

\vspace{3pt}\begin{lstlisting}
interface Deck {
  void draw() { // draws a card from the Deck
    Stack<Card> cards = this.getStack();
    if (!cards.isEmpty()) {
      Card card = cards.pop();
      ...
    }
  }
}

interface Drawable {
  void draw() { // draws something on the screen
    JFrame frame = new JFrame("Canvas");
    frame.setVisible(true);
    ...
  }
}
\end{lstlisting}\vspace{3pt}
Note that both methods have \lstinline|void| return type (we will not formalize
\lstinline|void| in our calculus afterwards; here is only for illustration). In \lstinline|Deck|, \lstinline|draw()| tries to get the cards as a stack, pops
out the top card, and so on. While in \lstinline|Drawable|, \lstinline|draw()|
creates a blank canvas using \lstinline|JFrame|. Now, a programmer is designing a
card game with GUI. He may want to draw a deck on the screen, so he defines a drawable
deck using multiple inheritance:

\vspace{3pt}\begin{lstlisting}
interface DrawableDeck extends Drawable, Deck {
  ...
} 
\end{lstlisting}\vspace{3pt}
The point of using multiple inheritance is surely for composing the features of
components, achieving great code reuse. It is supported by many mainstream OO
languages. Nevertheless at this point, \lstinline|DrawableDeck| has to throw a compile
error, for the two \lstinline|draw()| methods cause a conflict, though accidentally.

\subsection{Potential fixes}

For that problem, there are several workarounds that quickly come to our mind:

\paragraph{I. Delegation.} As an alternative to multiple inheritance, delegation can be used by
introducing two fields with \lstinline|Drawable| type and \lstinline|Deck| type, respectively. This avoids
method conflicts, nevertheless, delegation itself is too restricted in modularity, and meanwhile
introduces a lot of boilerplate.

\paragraph{II. Creating a \lstinline|draw()| method in \lstinline|DrawableDeck|, which explicitly overrides the old ones.}
This is a non-solution. It does not make any sense to override both methods with totally different functionalities, as old
methods have to be hidden.

\paragraph{III. Choosing one of them as the default method, like Mixins.} The mixin model can be applied to choose a
default one based on linearisation. Similarly, we want to preserve both features, rather than keeping only one of them.

\paragraph{IV. Method exclusion like traits.} Same reason as above.

\paragraph{V. Method renaming like traits.} This is probably what people do in most cases, by simply renaming one to avoid conflicts.
It can indeed preserve both features, however, it is cumbersome in practice, as introducing new names can affect other code blocks.
Certainly this is a workaround, not a solution.\\

What we really expect from the language is we keep both methods without renaming, and the type checker does not complain on the
inheritance. But we need to find an approach to disambiguate on method calls statically.

Certainly the compiler can ignore the conflict when \lstinline|DrawableDeck| is declared, but once an object of \lstinline|DrawableDeck| is created, a method call for \lstinline|draw()| on that object is ambiguous, due to dynamic dispatch. Nonetheless, we can adopt static dispatch for disambiguating. Some languages like C++ use qualified names in that way:

\vspace{3pt}\begin{lstlisting}
void func(Drawable obj) {
  obj.draw();
}

DrawableDeck d = new DrawableDeck();
d.Drawable::draw();       // calling draw() in Drawable
((Drawable) d).draw(); // calling draw() in Drawable
func(d);                 // calling draw() in Drawable
\end{lstlisting}\vspace{3pt}
Thus we have: \paragraph{VI. Static dispatch.} Static dispatch finds out and invokes the most specific method ``by need''.

On the other hand, we also need dynamic dispatch as it is essential and widely used in object-oriented programming.
C++ has the flexibility for choosing either way of dispatch by the ``virtual'' keyword.
Unfortunately, this approach is still unsatisfactory regarding code reuse. For instance, here we redefine \lstinline|Deck| to support
both \lstinline|draw()| and another operation called \lstinline|shuffleAndDraw()|:
\vspace{3pt}\begin{lstlisting}
interface Deck {
  void draw() {...}
  void shuffleAndDraw() {
    shuffle();
    draw();
  }
  ...
}
\end{lstlisting}\vspace{3pt}
\lstinline|shuffleAndDraw()| is a representative method that invokes \lstinline|draw()| in its definition. In principle, we want
that invocation to use dynamic dispatch, because a programmer may define a subtype of \lstinline|Deck|, and override \lstinline|draw()|:
\vspace{3pt}\begin{lstlisting}
interface LoggingDeck {
  void draw() { // overriding
    Stack<Card> cards = this.getStack();
    if (!cards.isEmpty()) {
      Card card = cards.pop();
      println("The card is: " + card.toString());
      ...
    } else {
      println("Empty deck.");
    }
  }
}
\end{lstlisting}\vspace{3pt}
Usually we want to reuse the code of \lstinline|shuffleAndDraw()| during inheritance, hence dynamic dispatch is necessary, otherwise
programmers have to override all the other methods that invoke \lstinline|draw()|. However, as seen before, dynamic dispatch can cause
ambiguity if we have:
\vspace{3pt}\begin{lstlisting}
interface DrawableLoggingDeck extends Drawable, LoggingDeck {
  ...
}

DrawableLoggingDeck d = new DrawableLoggingDeck();
d.shuffleAndDraw(); // ambiguous draw()
\end{lstlisting}\vspace{3pt}
Since the dynamic type of the receiver is \lstinline|DrawableLoggingDeck|, calling \lstinline|shuffleAndDraw()| triggers the ambiguity. When \lstinline|shuffleAndDraw()| invokes \lstinline|draw()|, what we really want is \lstinline|LoggingDeck.draw()|, yet
neither static dispatch nor dynamic dispatch in languages like C++ does so.
 Therefore, we need to find another algorithm for method binding.

\subsection{Solution in \MIM: \dispatchnamecaptical}
Our \MIM{} model uses \dispatchnameit{} for method lookup. A qualified method invocation, for instance, \lstinline|e.I::m()|, is read as ``finding the most specific \lstinline|m()| along path \lstinline|I|''. The meaning of ``along path \lstinline|I|'' is that, if the result of \dispatch{} is \lstinline|J.m()| for some \lstinline|J|, then such a \lstinline|J| must be a super type of \lstinline|e|'s dynamic type, and \lstinline|J| has a subtyping relation with \lstinline|I| (either \lstinline|J <: I| or \lstinline|J >: I|). Intuitively, the most specific \lstinline|m()| must be from branch \lstinline|I|, but it can be an overridden version after \lstinline|I| like dynamic dispatch. The formal definition will be introduced later.

On the other hand, \lstinline|((I) e).m()| behaves the same as \lstinline|e.I::m()| in our model. Such a dispatch make uses of both the static type and the dynamic type of the receiver. Intuitively, the static type specifies one branch of the method to avoid ambiguity, and the dynamic type finds the latest version on that branch. It may still introduce ambiguity when there are multiple paths from the static type to the dynamic type, and those paths cause conflicts. To prevent this, we disallow this kind of diamond inheritance to ensure unambiguity. That is to say, we do not allow two conflicted methods to override a same base method. This is natural as it is no longer an ``unintended'' conflict in that way.

With the old example, below code meets our expectation:
\vspace{3pt}\begin{lstlisting}
interface Deck {
  void draw() {...}
  void shuffleAndDraw() {
    this.Deck::shuffle();
    this.Deck::draw();
  }
  ...
}
\end{lstlisting}\vspace{3pt}
It guarantees that \lstinline|shuffleAndDraw()| are calling \lstinline|shuffle()| and \lstinline|draw()| from its own branch, so that the namesakes
from other branches will not cause conflicts. Now \lstinline|d.shuffleAndDraw()| is no longer ambiguous.

Actually in \MIM{}, we can still write ``\lstinline|shuffle();|'' and ``\lstinline|draw();|'',
because the compiler is able to know that the receiver ``\lstinline|this|'' exactly has static type \lstinline|Deck|, hence hierarchical dispatch eliminates ambiguity.
\textcolor{red}{Haoyuan: More. super call. update.}

\subsection{Method refinement}



% \section{Implementation}
The prototype is implemented in Scala.

% \section{Formalization}~\label{sec:formalization}
In this section, we present a formal model called \MIM{} (\emph{\textbf{F}eatherweight \textbf{H}ierarchical \textbf{J}ava}), following a similar style as  
Featherweight Java~\cite{Igarashi01FJ}. \MIM{} is a minimal core calculus that formalizes the core concept of hierarchical dispatching and overriding. The syntax, typing rules and small-step semantics are presented.

% \vspace{-2ex}
\subsection{Syntax}
The abstract syntax of \MIM{} interface declarations, method declarations, and expressions is given in Figure~\ref{fig:syntax}. The multiple
inheritance feature of \MIM{} is inspired by Java 8 interfaces, which supports
method implementations via default methods. This feature is 
closely related to \emph{traits}. To demonstrate how
unintentional method conflicts are untangled in \MIM{}, we only focus on a small subset of the interface model. For example, all methods declared
in an interface are either default methods or abstract methods. Default methods provide default implementations for methods. Abstract methods do not
have a method body. Abstract methods can be overridden with future implementations.

\subsubsection{Notations}
The metavariables $I, J$ range over interface names; $x$ ranges over variables; $m$ ranges over method names; $e$ ranges over expressions; and $M$ ranges over method declarations. Following Featherweight Java, we assume that the set of variables includes the special variable \kwthis, which cannot be used as the name of an argument to a method. We use the same
conventions as FJ; we write $\overline{I}$ as shorthand for a possibly empty sequence $I_1, ..., I_n$, which may be indexed by $I_i$; and write $\overline{M}$ as shorthand for $M_1 .. M_n$ (with no commas). We also abbreviate operations on pairs of sequences in an obvious way, writing $\overline{I} \; \overline{x}$ for $I_1 \; x_1, ..., I_n \; x_n$, where $n$ is the length of $\overline{I}$ and $\overline{x}$.

\subsubsection{Interfaces}
In order to achieve multiple inheritance, an interface can have a set of 
parent interfaces, where such a set can be empty. The interface declaration $\interface{I}{I}{M}$ introduces an interface named $I$ with parent interfaces $\overline{I}$ and a suite of methods $\overline{M}$. The methods of $I$ may either override methods that are already defined in $\overline{I}$ or add new functionality special to $I$, we will illustrate this in more detail later.

\subsubsection{Methods}
Original methods and hierarchically overriding methods share the same syntax in our model for simplicity.
The concrete method declaration $\method{I}{m}{I_x}{x}{J}{e}$ introduces a
method named $m$ with result type $I$, parameters $\overline{x}$ of
type $\overline{I_x}$ and the overriding target $J$. The body of the
method simply includes the returned expression $e$. Notably, we have introduced the
\kwoverride{} keyword for two cases: if the overridden interface is exactly the enclosing
interface itself, then such a method is seen as originally defined; otherwise it is a hierarchical overriding method. 
Note that in an interface $J$, $
I \; m(\overline{I_x} \; \overline{x}) \; {\{} \kwreturn \; e ; {\}} $ is syntactic sugar for $\method{I}{m}{I_x}{x}{J}{e}$, which is the standard way of method definition in Java-like languages. The definition
of abstract methods is written as $\absmethod{I}{m}{I_x}{x}{J}$, which is
similar to a concrete method but without the method body. 
For simplicity, overloading is not modelled for methods, which
implies that we can uniquely identify a method by its name.

\subsubsection{Expressions \& Values}
Expressions can be standard constructs such as variables, method
invocation, object creation, together with cast expressions. 
Object creation is represented by $\new I$\footnote{In Java the corresponding syntax is $\new I\{\}$.}. Fields and primitive types are not modelled in \MIM{}. 
The casts are merely safe upcasts, and in fact, they can be viewed as
annotated expressions, where the annotation indicates its static type.
The coexistence of static and dynamic types is the key to hierarchical dispatch.
A value
``$(I)\new{J}$''
is the final result of multiple reduction steps evaluating an
expression.

For simplicity, \name{} does not formalize statements like assignments and so on because they are orthogonal features to the hierarchical dispatching and overriding feature.
A program in \name{} consists of a list of interface declarations, plus a single expression.

\begin{figure*}[t]
\saveSpaceFig
\begin{displaymath}
\begin{array}{l}
\begin{array}{llrl}
\text{Interfaces}   & IL & \Coloneqq & \interface{I}{I}{M} \\
\text{Methods}      & M  & \Coloneqq & \method{I}{m}{I_x}{x}{J}{e}  \mid
									   \absmethod{I}{m}{I_x}{x}{J} \\
\text{Expressions}  & e  & \Coloneqq & x \mid
e.m(\overline{e}) \mid
\new{I} \mid \; (I)e \\
\text{Context}      & \Gamma & \Coloneqq & \overline{x}:\overline{I} \\
\text{Values}       & v & \Coloneqq & (I) \new{J} \\
%%\\
%%\text{Interface names} & I, J, K & & \\
%%\text{Method names} & m & & \\
%%\text{Variable names} & x & &
\end{array}
\end{array}
\end{displaymath}
\caption{Syntax of \name{}.}\label{fig:syntax}
\saveSpaceFig
\end{figure*}


\begin{figure*}[t]
\saveSpaceFig
\begin{mathpar}
	\framebox{$ I <: J $} \hspace{.5in} \subid \\
	\subtrans \hspace{.5in} \subextends \\
	
	\framebox{$ \judgeewf \Gamma {e:I} $} \hspace{.5in}
	\tvar \\
	\tinvk \\
	% \tpathinvk \\
	% \tsuperinvk \\
	% \tstaticinvk  \\
	\tnew \\
	\tanno \\
	\tmethod \\
	\tabsmethod \\
	\tintf
\end{mathpar}
\saveSpaceFig
\caption{Subtyping and Typing Rules of \name{}.}
\label{fig:typingrules}
\end{figure*}

\subsection{Subtyping and Typing Rules}\label{subsec:typingrules}
\subsubsection{Subtyping}
The subtyping of \MIM{} consists of only a few rules shown at the top of Figure~\ref{fig:typingrules}.
In short, subtyping relations are built from the inheritance in interface
declarations. Subtyping is both reflexive and transitive.

\subsubsection{Type-checking}
Details of type-checking rules are displayed at the bottom of Figure~\ref{fig:typingrules}, including expression
typing, well-formedness of methods and interfaces. As a convention, an environment
$\Gamma$ is maintained to store the types of variables, together with
the self-reference $\kwthis$.
% The three rules for method invocation, \textsc{(T-Invk)}, \textsc{(T-PathInvk)} and \textsc{(T-SuperInvk)}
% are very similar, in the sense that they all check the type of the specific method, by using
% an auxiliary function \mtype. \mtype{} is the function for looking up method types, which we will
% illustrate later in Section~\ref{subsec:auxdefs}. After the method
% type is obtained, they all check that the arguments and the receiver
% have compatible types. Additionally, \textsc{(T-PathInvk)} requires the receiver to be the subtype of the specified
% path type, and \textsc{(T-SuperInvk)} checks if the enclosing type directly extends the specified super type.

\textsc{(T-Invk)} is the typing rule for method invocation.
Naturally, the receiver and the arguments are required to be well-typed.
$\mbody$ is our key function for method lookup that implements the
hierarchical dispatching algorithm. The formal definition will be introduced in Section~\ref{sec:auxdefs}.
Here $\mbody(m, I_0, I_0)$ finds the most specific $m$ above $I_0$. ``Above $I_0$'' specifies
the search space, namely the supertypes of $I_0$ including itself.
For the general case, however, the hierarchical invocation $\mbody(m, I, J)$ finds ``the most specific $m$
above $I$ and along path/branch $J$''. ``Along path $J$'' additionally requires the result to relate to $J$, that is to say,
the most specific interface that has a subtyping relationship with $J$.

In \textsc{(T-Invk)}, as the compilation should not be aware
of the dynamic type, it only requires that invoking $m$ is valid for the static type of the
receiver. The result of $\mbody$ contains the interface that provides the most specific implementation,
the parameters and the return type. We use underscore for the return expression, implying that an empty return expression
from an abstract method is acceptable.
% \textsc{(T-PathInvk)} is the typing judgement for a path invocation. Besides the conditions of \textsc{(T-Invk)}, \textsc{(T-PathInvk)} requires the type of receiver to be the subtype of the specified path type. 
% and \textsc{(T-SuperInvk)} checks if the enclosing type directly extends the specified super type.

\textsc{(T-New)} is the typing rule for object creation $\new{I}$. The
auxiliary function $\canInstantiate(I)$ (see definition in Section~\ref{sec:otherdefs}) checks whether an interface $I$ 
can be instantiated or not. Since \wordfork{} inheritance accepts conflicting branches to coexist, the check requires that the most specific method is concrete for each method on each branch.

\textsc{(T-Method)} is more interesting since a method can either be an original method or a hierarchical overriding, though
they share the same syntax and method typing rule. $\mostSpecific(m, I, J)$ is a fundamental function,
used to find ``the most specific interfaces that are above $I$ and
along path $J$, and originally defines $m$'' (see
Section~\ref{sec:auxdefs} for full definition).
By ``most specific interfaces'',
it implies that the inherited supertypes are excluded. Thus the condition $\mostSpecific(m, I, J) = \{J\}$ indicates a characteristic of a hierarchical overriding: it must override an original method; the overriding is direct and there does not exist any other original method $m$ in between.
Then $\mbody(m, J, J)$ provides the type of the original method, so hierarchical overriding has to preserve the type. Finally the return expression
is type-checked to be subtype of the declared return type. For the definition of an original method, $I$ equals $J$ and the rule is straightforward. \textsc{(T-AbsMethod)} is a similar rule but works on abstract method declarations.

\textsc{(T-Intf)} defines the typing rule on interfaces. The first condition is obvious, namely its methods need to be well checked. The third
condition checks whether the overriding between original methods preserves typing. In this condition we again use some helper functions defined in  Section~\ref{sec:auxdefs}. $I[m\ \kwoverride\ I]$ is defined if $I$ originally defines $m$, and $\canOverride(m, I, J)$ checks whether $I.m$ has the same type as $J.m$. Generally the preservation of method type is required for any supertype $J$ and any method $m$.

The second condition of \textsc{(T-Intf)} is more complex and is the key to type soundness. Unlike C++ which rejects on ambiguous calls,
\MIM{} rejects on the definition of interfaces when they form a diamond. Consider the case when the second condition is broken: $\mbody(m, J, J)$
is defined but $\mbody(m, I, J)$ is undefined for some $J$ and $m$. This indicates that $m$ is available and unambiguous from the perspective of $J$,
but is ambiguous to $I$ on branch $J$. It means that there are multiple overriding paths of $m$ from $J$ to $I$, which form a diamond. Hence rejecting
that case meets our expectation. Below is an example (Figure~\ref{fig:examplesmbody} (e)) that illustrates the reason why this condition is needed:
%\bruno{what is the purpose of this example:
%  state-it upfront please. Is this example meant to ilustrate T-Inf?
%  Then it's better to have the example together with the text
%  explaining T-inf.} \yanlin{revised.please check whether you're happy with it.}
\begin{lstlisting}
interface T                 { T m() override T { return new T(); } }
interface A extends T       { T m() override T { return new A(); } }
interface B extends T       { T m() override T { return new B(); } }
interface C extends A, B {}
((T) new C()).m()
\end{lstlisting}
This program does not compile on interface $C$, because of the second condition in \textsc{(T-Intf)}, where $I$ equals $C$ and $J$ equals $T$.
By the algorithm, $\mbody(m, T, T)$ will refer to $T.m$, but $\mbody(m, C, T)$ is undefined, since both $A.m$ and $B.m$ are most specific
to $C$ along path $T$, which forms a diamond. The expression \lstinline|((T) new C()).m()| is one example of triggering ambiguity, but \MIM{}
simply rejects the definition of $C$. To resolve the issue, the programmer needs to have an overriding method in $C$, to explicitly merge
the conflicting ones.

Finally, rule \textsc{(T-Anno)} is the typing rule for a cast expression. By the rule, only upcasts are valid.

\subsection{Small-step Semantics and Congruence}
Figure~\ref{fig:smallstep} defines the small-step semantics and
congruence rules of \MIM{}. When evaluating an expression, they
are invoked and produce a single value in
the end. %\haoyuan{we need to be consistent on paragraph upper/lower case.}

\subsubsection{Semantic Rules} \textsc{(S-Invk)} is the only computation rule we need for method invocation.
As a small-step rule and by congruence, it assumes that the receiver and the arguments are already values.
Specifically, the receiver $(J)\new{I}$ indicates the dynamic type $I$
together with the static type $J$. Therefore $\mbody(m, I, J)$ carries out hierarchical dispatching, acquires
the types, the return expression $e_0$ and the interface $I_0$ which provides the most specific method.
Here we use $e_0$ to imply that the return expression is forced to be non-empty because it requires a concrete implementation. Now the
rule reduces method invocation to $e_0$ with substitution.
Parameters are substituted with arguments, and the \lstinline|this| reference is substituted with the receiver,
and in the meanwhile the static types are recorded via annotations. Finally, the return type $I_e$ is put in the front as an annotation.
\subsubsection{Congruence Rules} \textsc{(C-Receiver)}, \textsc{(C-Args)} and \textsc{(C-FReduce)} are natural congruence rules
on receivers, arguments, and cast-expressions, respectively. \textsc{(C-StaticType)} automatically adds an annotation $I$ to the new
object $\new{I}$. \textsc{(C-AnnoReduce)} merges nested upcasts into a single upcast with the outermost type.



\begin{comment}
\paragraph{Example} In contrast with the counter-example in Section~\ref{subsec:typingrules}, it is better to understand semantics by
well-compiled examples. Here we abstract a variant of the \lstinline|DrawableDeck| example:

\vspace{3pt}\begin{lstlisting}
interface Void       {}
interface JFrame     {}
interface Deck       { Void draw() override Deck { return new Void(); } }
interface Drawable { JFrame draw() override Drawable; }
interface DrawableDeck extends Drawable, Deck {
  JFrame draw() override Drawable {
    return new JFrame();
  }
}

((Drawable) new DrawableDeck()).draw()
\end{lstlisting}\vspace{3pt}
We put \lstinline|Drawable.draw| as an abstract method instead, but hierarchically override it in \lstinline|DrawableDeck|.
By typing rules, the code is well-compiled. And during runtime,
\begin{align*}
	& ((Drawable) new DrawableDeck()).draw() \\
\rightarrow & (JFrame) new JFrame()
\end{align*}
\end{comment}

% %\input{sections/Translation.tex}


% % \section{Case Study}

% Examples:
% \begin{enumerate}
% \item \yanlin{http://stackoverflow.com/questions/2598009/java-method-name-collision-in-interface-implementation}

% \item \yanlin{https://msdn.microsoft.com/en-us/library/aa288461(v=vs.71).aspx}
% \end{enumerate}

\section{Proof}


% %\input{sections/FutureWork.tex}

% \section{Related Work}

\begin{itemize}
	\item Multiple inheritance models
		\begin{itemize}
			\item Mixin
			\item Scala Mixins
			\item trait
			\item C++ model
			\item Java 8
			\item CZ
			\item \red{C# Explicit method implementation}
		\end{itemize}
	\item Resolve unintended method confliction
		\begin{itemize}
			\item Parents are Shared Parts of Objects: Inheritance and Encapsulation in SELF* (tiebreaker rule)
			\item ???
		\end{itemize}
	\item Static+Dynamic type method lookup (any existing language that supports this?)
	\item Formalization based on FJ (novelty: keep static types <I> in formalization)
		\begin{itemize}
			\item Existing formalizations based on FJ proposed new features and added rules in syntax and semantics. But we not only add rules, but also piggyback static types on almost all semantic rules to model method lookup. 
			\item Featherweight defenders, ...
		\end{itemize}
\end{itemize}


% \section{Conclusion}


This paper proposes \MIM{} as a formalized multiple inheritance model for
unintentional method conflicts. Previous approaches 
either do not support unintentional method conflicts, thus have to
compromise between code reuse and type safety, or do not fully support
overriding in the presence of unintentional conflicts. To deal with unintentional method conflicts we
introduce two key mechanisms: hierarchical dispatching and
hierarchical overriding. Hierarchical dispatching is inspired by the
mechanisms in C++. We provide a minimal formal model of hierarchical
dispatching in \MIM{}. Such an algorithm makes use of both dynamic type
information and static information from either upcasts or parameters'
information. It not only offers great code reuse like
dynamic dispatch but also ensures unambiguity by our algorithm for
method resolution. Additionally we introduce \emph{hierarchical
  overriding} to allow conflicting methods in different branches to be
individually overridden.

\MIM{} is formalized following the style of
Featherweight Java and proved to be sound. A prototype interpreter is
implemented in Scala. We believe that the formalization of
hierarchical dispatching features is general and
can be safely embedded in other OO models, so as to have support for the triangle
inheritance.

Our model can certainly be improved in some aspects. 
As discussed in Section~\ref{sec:discussion}, there are orthogonal and
non-orthogonal features that can potentially be added to the design space. 
The future work relates to loosening the model without giving up its soundness,
together with more exploration on supporting fields in the multiple inheritance setting.


\vspace{-5pt}
\acks
We would like to thank the reviewers for their helpful comments. This work is sponsored by the Hong Kong Research Grant Council Early Career Scheme project number 27200514.
%
% We recommend abbrvnat bibliography style.

\newpage
\clearpage
\bibliographystyle{abbrvnat}
\bibliography{paper}

% The bibliography should be embedded for final submission.

%\begin{thebibliography}{}
%\softraggedright

%\bibitem[Smith et~al.(2009)Smith, Jones]{smith02}
%P. Q. Smith, and X. Y. Jones. ...reference text...

%\end{thebibliography}





% \appendix
% \input{sections/Appendix_Formalization.tex}
% \input{sections/Translation.tex}
% \input{sections/Appendix_Translation.tex}
% \section{Appendix}

\subsection{Proofs}~\label{appendix_proof}
TODO.



\end{document}
